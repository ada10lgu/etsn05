\documentclass[a4paper]{article}
\usepackage[pdftex]{graphicx}
\usepackage{anysize}
\marginsize{3cm}{3cm}{3cm}{3cm}
\usepackage[utf8]{inputenc}
\usepackage[T1]{fontenc}       

\usepackage[swedish]{babel}      
\usepackage{epstopdf}     % För svensk avstavning och svenska
\usepackage[osf]{mathpazo} % Palatino with smallcaps and oldstyle numbers
\usepackage[scaled]{helvet} % Helvetica, scaled 95%

\usepackage{etoolbox}

\newcommand\getcurrentref[1]{%
 \ifnumequal{\value{#1}}{0}
  {??}
  {\the\value{#1}}%
}  
\newcommand\requirement[2]{
	\numberedrow{Krav}{#1}{#2}
}
\newcommand\scenario[2] {
	\numberedrow{Scenario}{#1}{#2}
}
\newcommand\numberedrow[3]{
	\noindent
	\textbf{#1 \getcurrentref{section}.\getcurrentref{subsection}.#2.} #3
	
}

\usepackage{fancyhdr}
\fancyhf{}
\fancyhead[L]{Ansvarig: SG}
\fancyhead[C]{Datum: \today}
\fancyhead[R]{Version: 1.0}


\title{SRS - Software Requirements Specification: NewPussSystem}                  	
\author{Systemarkitektgruppen \\ Lars Gustafsson | Martin Lichota | Marcel Tovar Rascon}
\date{}

\begin{document}

\maketitle
\thispagestyle{fancy}
\tableofcontents
\newpage

\section*{Dokumenthistorik}

\begin{tabular}{ l l l l }
Ver. & Datum & Ansv. & Beskrivning \\\hline
1.0 & \today & SG & Första baseline-versionen

\end{tabular}
\section{Inledning}       


Dokumentet beskriver kraven för <Program X>, ett tidsrapportingssystem för projekt som diverse användare kan logga in på.

\section{Referensdokument}
I denna version används inget referensmaterial.
\section{Bakgrund och mål}   
\subsection{Huvudmål}
Huvudmålet är att tillhandhålla ett system där olika användare, såsom projektledare och övriga projektmedlemmar, ska kunna tidsrapportera och loggföra det fortgående arbetet i sitt projekt. 

\subsection{Aktörer och deras mål}
Följande aktörer kommer att använda systemet:
\begin{itemize}
\item [] \textbf{Vanlig användare (User)} En användare kan logga in i systemet och tidsrapportera. Denne har även tillgång till statistik rörande den egna tidsrapporteringen.
\item [] \textbf{Administrator (Admin)} En administratör är en specifik användare som har privligerade rättigheter. Denne kan lägga till och ta bort andra användare.
\item [] \textbf{Projektledare (PL)} Projektledare är en roll som kan tilldelas till en User vilket ger den administrativa rättigheter för ett givet projekt.
\end{itemize}

\section{Terminologi}
Här följer ord och uttryck som används i rapporten och är till för att öka förståelsen.
\begin{itemize}
\item [Användarnamn] Unik indentifikationsfras för att representera en användare i systemet..
\item [Lösenord] Hemlig fras endast känd för var unik användare samt systemet så användaren kan påvisa sin identitet.
\item [Inloggad] En användare som har identifierat sig mot systemet med användarnamn och lösenord, detta sker genom att användaren loggar in.
\item [Logga in] Se inloggad.
\item [Användarstatus] En indikation på var användare som avgör hur vida den får logga in eller ej.
\item [Projektgrupp] En grupp bestående av vanliga användare och projektledare.
\item [Tidsrapport] En rapport som innehåller arbetsbelasting för en användare under en fix tidsperiod bundet till en specefik projektgrupp.
\end{itemize}
\section{Contextdiagram}
Denna version innehålller inte ett contextdiagram.
\section{Funktionella krav}
\subsection{Generella krav}
 \requirement{1}{Alla scenarion beskrivna under kapitel \ref{scenarion} skall stödjas av systemet.}
 \requirement{2}{Systemet ska inte avbrytas eller låsa sig på grund av att användaren matar in en viss sekvens av data.}
 \requirement{3}{Flera användare ska kunna logga in på systemet samtidigt.}
 
\subsection{Autentisering}
\requirement{1}{Man ska kunna logga in med användarnamn och lösenord.}
\requirement{2}{En inloggad användare skall ha möjligheten att byta sitt lösenord.}
\requirement{3}{Då en användare har varit inaktiv i mer än 15 minuter så är den inte längre inloggad.}
\requirement{4}{Om en användare loggar in från en annan plats ska den första loggas ut samt ges ett meddelande om vad som skett.}

\subsection{Data}
\label{krav-funk-data}
\requirement{1}{Alla uppdatering i systemet ska sparas omedelbart i databasen så att inga ändringar går förlorade vid systemkrasch.}
\requirement{2}{Varje radering av information ska bekräftas av användaren innan den utförs.}
\requirement{3}{Följande information skall vara lagrad om var användare; Användarnamn, lösenord, namn på personen, användarstatus}
\requirement{4}{Följande information skall vara lagrad om var projektgrupp; Gruppnamn, och medlemmar med deras roll}
\requirement{5}{Rätt data ska kunna nås av rätt användare.}
\subsection{Administration}
\requirement{1}{Det skall finnas ett och endast ett adminkonto i systemet.}
\requirement{2}{Admin och endast admin ska kunna skapa nya användare i systemet.}
\requirement{3}{Ny användare måste ha ett unikt användarnamn samt bli tildelad ett slumpmässigt lösenord från systemet.}
\requirement{4}{Om admin försöker att lägga till en användare med ett användarnamn som redan existerar i systemet ska användaren inte läggas till och ett felmeddelande ska visas.}
\requirement{5}{Admin och endast admin ska kunna inaktivera alla användarkonton förutom sitt eget.}
\requirement{6}{Admin och endast admin ska kunna aktivera alla tidigare inaktiverade kontonon.}
\requirement{7}{När en användare inaktiveras ska admin kunna välja mellan alla aktiva användare från en lista genererat av systemet.}
\requirement{8}{När en användare läggs till, inaktiveras eller aktiveras ska en bekräftelseruta visas där admin kan välja att bekräfta och slutföra ändringen eller välja avbryt så att ingen ändring görs.}
\requirement{9}{Admin och endast admin ska kunna skapa projektgrupper i systemet.}
\requirement{10}{Då en ny användare skapas ska admin ange den information som är specifierat av Krav \ref{krav-funk-data}.3.}
\requirement{11}{Då en ny projektgrupp skapas ska admin ange den information som är specifierat av Krav \ref{krav-funk-data}.4.}
\requirement{12}{Om admin försöker skapa en projektgrupp med ett namn som redan existerar ska projektgruppen inte skapas och ett felmeddelande ska visas.}
\requirement{13}{Admin och endast admin ska kunna ta bort projektgrupper i systemet.}
\requirement{14}{När en projektgrupp tas bort ska admin kunna välja projektgrupp ur en lista på alla projektgrupper i systemet.}

\subsection{Tidsraportering}
\requirement{1}{Tidrapporter ska kunna sparas, uppdateras och raderas.}
\subsection{Projektledning}
\requirement{1}{En användare kan vara projektledare för flera olika projekt.}
\requirement{2}{En projektledare har bara rätt att göra ändringar som en projektledare kan för det projektet den är projektledare för.}
\requirement{2}{Projektledaren skall kunna se samtliga tidrapporter i projektgruppen.}
\requirement{3}{Projektledaren skall kunna godkänna ej tidigare godkända tidrapporter.}
\requirement{4}{Projektledaren skall kunna ta tillbaka sitt godkännande från en tidigare godkänd tidsrapport.}
\requirement{5}{Projektledaren skall kunna lista alla projektmedlemmar.}
\requirement{6}{Projektledaren kan hämta ut statestik om tidsrapporterna inom sitt projekt. Rapporterna skall kunnas summera per användare, roll, aktivitet eller vecka.}
\requirement{7}{Projektledaren skall kunna Tilldela roller till medlemmarna i projektgruppen.}
\requirement{8}{Admin kan göra allt som en projektledare kan för varje existerande projekt.}
\section{Funktionella scenarion}
\label{scenarion}
\subsection{Generella scenarion}
I denna version finns inga Generella scenarion.
\subsection{Autentisering}
I denna version finns inga scenarion gällande autentisering.
\subsection{Data}
I denna version finns inga scenarion gällande data.
\subsection{Administration}
I denna version finns inga scenarion gällande administration.
\subsection{Tidsraportering}
I denna version finns inga scenarion gällande tidsrapportering.
\subsection{Projektledning}
I denna version finns inga scenarion gällande projektledning.
\section{Kvalitetskrav}
\subsection{Underhåll}
\requirement{1}{Systemet skall vara väl dokumenterad så det underlättar vidareutvekling av systemet i framtiden.}
\requirement{2}{Förståelse av Java, i nivå med vad som lärs ut i kursen EDA016, samt grundläggande kunskap av SQL skall räcka för att underhålla samt vidareutvekla systemet.}
\subsection{Prestanda}
\requirement{1}{Då systemet används i en av datorsalarna I “E-huset”, LTH, skall svaret på en godtycklig förfrågan i åtminstone 95\% av fallen ges inom 1,0 s.}
\section{Projektkrav}
\subsection{Utvecklingsmiljö}
\requirement{1}{Systemet skall vara utveklat för Apache-Tomcat servern.}
\requirement{2}{Systemet skall vara utveklat i Java.}
\requirement{3}{Databaslösningen MySQL skall användas av programmet för lagring av data mellan sessioner.}
\requirement{4}{Systemet samt projekt- och produktdokumentation ska skrivas på svenska. Java-koden ska följa standarden som finns på http://www.geosoft.no/development/javastyle.html, alla variabelnamn ska vara skrivna på engelska.}

\end{document}