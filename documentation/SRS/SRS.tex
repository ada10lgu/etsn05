
%%%%%%%%%%%%%%%%%%%%%%%%%%%%%%%%%%%%%%%%%%%%%%%%%%%%%%%%%%%%%%%%%%%%%%%%%%%%%%
% Detta är ett exempel på ett latexdokument.
% 
% Alla dokument består av följande delar:
%
%          \documentclass[optioner]{dokumentklass}
%            ...inställningar...
%          \begin{document}
%            ...text...
%          \end{document}
%
% Som ni kanske redan har förstått är används procent (%) för
% kommentarer.
%%%%%%%%%%%%%%%%%%%%%%%%%%%%%%%%%%%%%%%%%%%%%%%%%%%%%%%%%%%%%%%%%%%%%%%%%%%%%%

\documentclass[titlepage,a4paper]{article}
\usepackage[pdftex]{graphicx}
\usepackage{anysize}
\marginsize{3cm}{3cm}{3cm}{3cm}
\usepackage[utf8]{inputenc}
\usepackage[T1]{fontenc}                % För svenska bokstäver
\usepackage[swedish]{babel}      
\usepackage{epstopdf}     % För svensk avstavning och svenska
\usepackage[osf]{mathpazo} % Palatino with smallcaps and oldstyle numbers
\usepackage[scaled]{helvet} % Helvetica, scaled 95%

                  

\begin{document}

\begin{titlepage}
\begin{center}
\includegraphics[width=0.30\textwidth]{./logo}~\\[1cm]
\textsc{\LARGE SRS}\\
\textup{Programvaruutv.}\\[1cm]
Martin Lichota\\


\end{center}
\end{titlepage}
                   % Skriver ut rubriken som vi
                                % deklarerade ovan med \title, \author
                                % och eventuellt \date

\section{Inledning}          % Detta kommando gör en rubrik

Dokumentet beskriver kraven för <Program X>, ett tidsrapportingssystem för projekt som diverse användare kan logga in på.


          % Detta blir en underrubrik
\section{Referensdokument}
< TOMT >
\section{Bakgrund och mål}   
\subsection{Huvudmål}
Huvudmålet är att tillhandhålla ett system där olika användare, såsom projektledare och övriga projektmedlemmar, ska kunna tidsrapportera och loggföra det fortgående arbetet. 

\subsection{Aktörer och deras mål}
Följande aktörer kommer att använda systemet:

\textbf{Vanlig användare (User)} En användare kan logga in i systemet och tidsrapportera. Denne har även tillgång till statistik rörande den egna tidsrapporteringen.

\textbf{Administrator (Admin)} En administratör är en specifik användare som har privligerade rättigheter. Denne kan lägga till och ta bort andra användare.

\textbf{Projektledare (PL)} En administratör är en specifik användare som har privligerade rättigheter. Denne kan lägga till och ta bort andra användare.

%-------------------------------------------------------------------
\pagebreak
\section{Terminologi}
\section{Context diagram}
%----------
\section{Funktionella krav}
\subsection{Autentisering}
\subsection{Data}
\subsection{Administration}
\subsection{Generella krav}
%-----------
\section{Kvalitetskrav}
\subsection{Underhåll}
\subsection{Prestanda}
%----------
\section{Projektkrav}
\subsection{Utvecklingsmiljö}








\end{document}                 % The input file ends with this command.









