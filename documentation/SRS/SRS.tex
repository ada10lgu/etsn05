\documentclass[a4paper]{article}
\usepackage[pdftex]{graphicx}
\usepackage{anysize}
\marginsize{3cm}{3cm}{3cm}{3cm}
\usepackage[utf8]{inputenc}
\usepackage[T1]{fontenc}       

\usepackage[swedish]{babel}      
\usepackage{epstopdf}     % För svensk avstavning och svenska
\usepackage[osf]{mathpazo} % Palatino with smallcaps and oldstyle numbers
\usepackage[scaled]{helvet} % Helvetica, scaled 95%

\usepackage{etoolbox}

\newcommand\getcurrentref[1]{%
 \ifnumequal{\value{#1}}{0}
  {??}
  {\the\value{#1}}%
}  
\newcommand\requirement[2]{
\noindent
\textbf{\getcurrentref{section}.\getcurrentref{subsection}.#1} #2

}

\usepackage{fancyhdr}

\fancyhf{}% Clear header/footer
\fancyhead[L]{Ansvarig: PG}
\fancyhead[C]{Datum: \today}
\fancyhead[R]{Version: 1.0}





\title{SRS - Software Requirements Specification: NewPussSystem}                  	
\author{Systemarkitektgruppen \\ Lars Gustafsson | Martin Lichota | Marcel Tovar Rascon}
\date{}

\begin{document}

\maketitle
\thispagestyle{fancy}
\tableofcontents
\newpage

\section*{Dokumenthistorik}

\begin{tabular}{ l l l l }
Ver. & Datum & Ansv. & Beskrivning \\\hline
1.0 & \today & SG & Första baseline-versionen

\end{tabular}
\section{Inledning}          % Detta kommando gör en rubrik


Dokumentet beskriver kraven för <Program X>, ett tidsrapportingssystem för projekt som diverse användare kan logga in på.


          % Detta blir en underrubrik
\section{Referensdokument}
I denna version används inget referensmaterial.
\section{Bakgrund och mål}   
\subsection{Huvudmål}
Huvudmålet är att tillhandhålla ett system där olika användare, såsom projektledare och övriga projektmedlemmar, ska kunna tidsrapportera och loggföra det fortgående arbetet. 

\subsection{Aktörer och deras mål}
Följande aktörer kommer att använda systemet:
\begin{itemize}
\item [] \textbf{Vanlig användare (User)} En användare kan logga in i systemet och tidsrapportera. Denne har även tillgång till statistik rörande den egna tidsrapporteringen.
\item [] \textbf{Administrator (Admin)} En administratör är en specifik användare som har privligerade rättigheter. Denne kan lägga till och ta bort andra användare.
\item [] \textbf{Projektledare (PL)} Projektledare är en roll som kan tilldelas till en User vilket ger den administrativa rättigheter för ett givet projekt.
\end{itemize}

\section{Terminologi}
Här följer ord och uttryck som används i rapporten och är till för att öka förståelsen.
\begin{itemize}
\item [Random word] Detta betyder något.
\item [Other thing] Betyder inte direkt något men nått gör det nog.
\end{itemize}
\section{Context diagram}
\section{Funktionella krav}
\subsection{Generella krav}
 \requirement{1}{Alla scenarion beskrivna under kapitel \ref{scenarion} skall systemet stödja.}
\subsection{Autentisering}
\subsection{Data}
\subsection{Administration}
\subsection{Tidsraportering}
\subsection{Projektledning}
\section{Funktionella scenarion}
\label{scenarion}
\subsection{Generella scenarion}
\subsection{Autentisering}
\subsection{Data}
\subsection{Administration}
\subsection{Tidsraportering}
\subsection{Projektledning}
\section{Kvalitetskrav}
\subsection{Underhåll}
\requirement{1}{Grundläggande förståelse av Java, nivå av vad som lärs ut i kursen EDA016, samt grundläggande kunskap av SQL skall räcka för att underhålla samt vidareutvekla systemet.}
\subsection{Prestanda}
\requirement{1}{Programmet skall vara snabbast i världen!}
\section{Projektkrav}
\subsection{Utvecklingsmiljö}
\requirement{1}{Systemet skall vara utveklat för Apache-Tomcat servern.}
\requirement{2}{Systemet skall vara utveklat i Java.}
\requirement{3}{Databaslösningen MySQL skall användas av programmet för lagring av data mellan sessioner.}



\end{document}                 % The input file ends with this command.









