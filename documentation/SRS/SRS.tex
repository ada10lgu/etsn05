\documentclass[a4paper]{article}
\usepackage[pdftex]{graphicx}
\usepackage{anysize}
\marginsize{3cm}{3cm}{3cm}{3cm}
\usepackage[utf8]{inputenc}
\usepackage[T1]{fontenc}       
\usepackage[swedish]{babel}      
\usepackage{epstopdf}     % För svensk avstavning och svenska
\usepackage[osf]{mathpazo} % Palatino with smallcaps and oldstyle numbers
\usepackage[scaled]{helvet} % Helvetica, scaled 95%
\usepackage{etoolbox}

\newcommand\getcurrentref[1]{%
 \ifnumequal{\value{#1}}{0}
  {??}
  {\the\value{#1}}%
}   
\newcommand\requirement[2]{
	\textbf{\getcurrentref{section}.\getcurrentref{subsection}.#1} #2

}


\title{SRS - Software Requirements Specification: NewPussSystem}                  	
\author{Systemarkitektgruppen \\ Lars Gustafsson | Martin Lichota | Marcel Tovar Rascon}
\date{}

\begin{document}
\maketitle
\tableofcontents
\newpage

\section*{Dokumenthistorik}
\begin{tabular}{ l l l l }
Ver. & Datum & Ansv. & Beskrivning \\\hline
1.0 & 2014-XX-XX & SG & Första baseline-versionen

\end{tabular}
\section{Inledning}          % Detta kommando gör en rubrik


Dokumentet beskriver kraven för <Program X>, ett tidsrapportingssystem för projekt som diverse användare kan logga in på.


          % Detta blir en underrubrik
\section{Referensdokument}
I denna version används inget referensmaterial.
\section{Bakgrund och mål}   
\subsection{Huvudmål}
Huvudmålet är att tillhandhålla ett system där olika användare, såsom projektledare och övriga projektmedlemmar, ska kunna tidsrapportera och loggföra det fortgående arbetet. 

\subsection{Aktörer och deras mål}
Följande aktörer kommer att använda systemet:
\begin{itemize}
\item [] \textbf{Vanlig användare (User)} En användare kan logga in i systemet och tidsrapportera. Denne har även tillgång till statistik rörande den egna tidsrapporteringen.
\item [] \textbf{Administrator (Admin)} En administratör är en specifik användare som har privligerade rättigheter. Denne kan lägga till och ta bort andra användare.
\item [] \textbf{Projektledare (PL)} En administratör är en specifik användare som har privligerade rättigheter. Denne kan lägga till och ta bort andra användare.
\end{itemize}

\section{Terminologi}
\section{Context diagram}
\section{Funktionella krav}
\subsection{Generella krav}
 \requirement{1}{Alla scenarion beskrivna under kapitel \ref{scenarion} skall systemet stödja.}
\subsection{Autentisering}
\subsection{Data}
\subsection{Administration}
\subsection{Tidsraportering}
\subsection{Projektledning}
\section{Funktionella scenarion}
\label{scenarion}
\subsection{Generella scenarion}
\subsection{Autentisering}
\subsection{Data}
\subsection{Administration}
\subsection{Tidsraportering}
\subsection{Projektledning}
\section{Kvalitetskrav}
\subsection{Underhåll}
\subsection{Prestanda}
\section{Projektkrav}
\subsection{Utvecklingsmiljö}








\end{document}                 % The input file ends with this command.









