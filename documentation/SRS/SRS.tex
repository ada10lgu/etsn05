\documentclass[a4paper]{article}
\usepackage[pdftex]{graphicx}
\usepackage{anysize}
\marginsize{3cm}{3cm}{3cm}{3cm}
\usepackage[utf8]{inputenc}
\usepackage[T1]{fontenc}       

\usepackage[swedish]{babel}      
\usepackage{epstopdf}     % För svensk avstavning och svenska
\usepackage[osf]{mathpazo} % Palatino with smallcaps and oldstyle numbers
\usepackage[scaled]{helvet} % Helvetica, scaled 95%

\usepackage{etoolbox}

\newcommand\getcurrentref[1]{%
 \ifnumequal{\value{#1}}{0}
  {??}
  {\the\value{#1}}%
}  
\newcommand\requirement[2]{
	\numberedrow{Krav}{#1}{#2}
}
\newcommand\scenario[2] {
	\numberedrow{Scenario}{#1}{#2}
}
\newcommand\numberedrow[3]{
	\noindent
	\textbf{#1 \getcurrentref{section}.\getcurrentref{subsection}.#2.} #3
	
}

\usepackage{fancyhdr}
\fancyhf{}
\fancyhead[L]{Ansvarig: SG}
\fancyhead[C]{Datum: \today}
\fancyhead[R]{Version: 0.7}


\title{SRS - Software Requirements Specification: NewPussSystem}                  	
\author{Systemarkitektgruppen \\ Lars Gustafsson | Martin Lichota | Marcel Tovar Rascon}
\date{}

\begin{document}

\maketitle
\thispagestyle{fancy}
\tableofcontents
\newpage

\section*{Dokumenthistorik}

\begin{tabular}{ l l l l }
Ver. & Datum & Ansv. & Beskrivning \\\hline
1.0 & \today & SG & Första baseline-versionen

\end{tabular}
\section{Inledning}       


Dokumentet beskriver kraven för <Program X>, ett tidsrapportingssystem för projekt som diverse användare kan logga in på.

\section{Referensdokument}
I denna version används inget referensmaterial.
\section{Bakgrund och mål}   
\subsection{Huvudmål}
Huvudmålet är att tillhandhålla ett system där olika användare, såsom projektledare och övriga projektmedlemmar, ska kunna tidsrapportera och loggföra det fortgående arbetet i sitt projekt. 

\subsection{Aktörer och deras mål}
Följande aktörer kommer att använda systemet:
\begin{itemize}
\item [] \textbf{Vanlig användare (User)} En användare kan logga in i systemet och tidsrapportera. Denne har även tillgång till statistik rörande den egna tidsrapporteringen.
\item [] \textbf{Administrator (Admin)} En administratör är en specifik användare som har privligerade rättigheter. Denne kan lägga till och ta bort andra användare.
\item [] \textbf{Projektledare (PL)} Projektledare är en roll som kan tilldelas till en User vilket ger den administrativa rättigheter för ett givet projekt.
\end{itemize}

\section{Terminologi}
Här följer ord och uttryck som används i rapporten och är till för att öka förståelsen.
\begin{itemize}
\item [Användarnamn] Unik indentifikationsfras för att representera en användare i systemet..
\item [Lösenord] Hemlig fras endast känd för var unik användare samt systemet så användaren kan påvisa sin identitet.
\item [Inloggad] En användare som har identifierat sig mot systemet med användarnamn och lösenord, detta sker genom att användaren loggar in.
\item [Logga in] Se inloggad.
\item [Användarstatus] En indikation på var användare som avgör hur vida den får logga in eller ej.
\item [Projektgrupp] En grupp bestående av vanliga användare och projektledare.
\item [Tidsrapport] En rapport som innehåller arbetsbelasting för en användare under en fix tidsperiod bundet till en specefik projektgrupp.
\end{itemize}
\section{Kontextdiagram}
Denna version innehålller inte ett kontextdiagram.
\section{Funktionella krav}
\subsection{Generella krav}
\label{krav-funk-gen}

 \requirement{1}{Systemet ska inte avbrytas eller låsa sig på grund av att användaren matar in en viss sekvens av data.}
\requirement{2}{Flera användare ska kunna logga in på systemet samtidigt.}
\requirement{3}{Samtliga användare ska ha tillgång till dessa funktionaliteter: Start, tidrapportering, byta lösenord, hjälp samt utloggning.}
\requirement{4}{Samtliga funktionaliter i krav \ref{krav-funk-gen}.4 ska vara tillgängliga i menyn.}
\requirement{5}{Menyn ska vara tillgänglig på alla sidor som visas av systemet. }
 
\subsection{Autentisering}
\requirement{1}{För varje användare kan loginstatus vara antingen inloggad eller inte inloggad.}
\requirement{2}{Systemet ska hålla loginstatus i en serversession.}
\requirement{3}{När en användare når systemet och inte är inloggad ska denna få en förfrågan om användarnamn och lösenord. Ingen annan information ska vara tillgänglig för användaren.}
\requirement{4}{När en användare skickar ett användarnamn och lösenord ska dessa parametrar jämföras med användarlistan och om användaren ska få tillgång till systemet , så ska serverstatus ändras till 'inloggad' och huvudsidan ska visas.}
\requirement{5}{Om en inloggad användare är inaktiv för längre än 20 minuter ska denna loggas ut och tvingas att logga in igen innan vidare använding av systemet.}
\requirement{6}{Ett användarkonto kan endast vara inloggad på en enhet åt gången.}
\requirement{7}{Om en användare försöker logga in med ett användarkonto som redan är inloggat på en annan enhet, kommer den nya inloggingen att slutföras medan den andra enheten loggas ut.}

\subsection{Data}
\label{krav-funk-data}
\requirement{1}{Användarnamn bör bestå av 5-10 tecken, ascii (decimal) värden 48-57, 65-90, och 97-122 är tillåtna.}
\requirement{2}{Användarnamn ska vara unika.}
\requirement{3}{Lösenord ska bestå av sex tecken, ascii (decimal) värden 97-122 är tillåtna.}

\requirement{4}{Datamodellen för informationslagring om användarna ska ske enligt figur 2.
}

\subsection{Administration}
\requirement{1}{Systemet ska stödja sekvensen för administrören som visas i figur 3}
\requirement{2}{Det skall finnas en och endast en administratör med användarnamnet 'admin' och lösenordet 'adminpw'.}
\requirement{3}{På huvudsidan ska det vara möjligt att välja administrationsvyn.}
\requirement{4}{En ny användare, skapad av administratören, måste ha ett unikt användarnamn samt bli tilldelad ett slumpmässigt lösenord från systemet.}
\requirement{5}{Om administratören försöker att lägga till en användare med ett användarnamn som redan existerar i systemet, ska användaren inte läggas till och ett felmeddelande ska visas.}
\requirement{6}{Om en administratör väljer administrationsvyn ska denne få åtkomst till administationsverktygen, men om samma val görs av en ickeadministratör ska denne istället få åtkomst till huvudsidan.}
\requirement{7}{På administrationsvyn ska alla användare listas med både användarnamn och lösenord.}
\requirement{8}{På administrationsvyn ska det vara möjligt att ta bort vilken användare som helst förutom administratören.}
\requirement{9}{Varje borttagning av en användare ska bekräftas av en dialogruta 'Är du säker på att du vill ta bort användaren X: Ja / Nej'. När administratören väljer 'Ja', tas användaren bort och administratören kommer tillbaka till en uppdaterad lista med användarna. Om 'Nej' väljs, går administratören tillbaka till nuvarande sida.}
\requirement{10}{På administrationsvyn ska det vara möjligt att lägga till en ny användare.}
\requirement{11}{Om en administratör försöker lägga till en ny användare med ett användarnamn som strider mot krav \ref{krav-funk-data}.1-2 ska ett felmeddelande visas och användaren ska inte läggas till.}

\requirement{12}{Endast administratören ska kunna skapa projektgrupper i systemet.}

\requirement{13}{Då en ny projektgrupp skapas ska administratören ange ett gruppnamn, projektledare och eventuella gruppmedlemmar.}

\requirement{14}{Om administratören försöker skapa en projektgrupp med ett namn som redan existerar ska projektgruppen inte skapas och ett felmeddelande ska visas.}
\requirement{15}{Endast administratören ska kunna ta bort projektgrupper i systemet.}
\requirement{16}{När en projektgrupp tas bort ska administratör kunna välja projektgrupp ur en lista på alla projektgrupper i systemet.}



\subsection{Tidrapportering}
\textbf{Förutsättning: Användaren är medlem i ett projekt.}\newline
\requirement{1}{Nygenererade tidrapporter är alltid osignerade.}
\requirement{2}{Användaren ska kunna skapa, uppdatera och ta bort sina egna osignerade tidrapporter.}
\requirement{3}{Användaren kan inte ta bort eller redigera signerade rapporter.}
\scenario{1} <KLIPP IN TASK>
\subsection{Projektledning}
\requirement{1}{Varje projekt ska ha två, och endast två användare, som är projektledare.}
\requirement{2}{Projektledaren skall ha tillgång till samtliga projektmedlemmars tidrapporter i sin projektgrupp.}
\requirement{3}{Projektledaren skall kunna godkänna ej tidigare godkända tidrapporter i från medlemmar sin projektgrupp.}
\requirement{4}{Projektledaren skall kunna ta tillbaka sitt godkännande från en tidigare godkänd gruppmedlems tidsrapport, i sin projektgrupp.}
\requirement{5}{Statistik (TASK)}
\requirement{6}{Projektledaren skall kunna tilldela egenbestämda roller till gruppmedlemmarna i sin projektgrupp.}
\section{Funktionella scenarion}
\label{scenarion}
\subsection{Generella scenarion}
I denna version finns inga Generella scenarion.
\subsection{Autentisering}
I denna version finns inga scenarion gällande autentisering.
\subsection{Data}
I denna version finns inga scenarion gällande data.
\subsection{Administration}
I denna version finns inga scenarion gällande administration.
\subsection{Tidsraportering}
I denna version finns inga scenarion gällande tidsrapportering.
\subsection{Projektledning}
I denna version finns inga scenarion gällande projektledning.
\section{Kvalitetskrav}
\subsection{Underhåll}
\requirement{1}{Systemet skall vara väl dokumenterad så det underlättar vidareutvekling av systemet i framtiden.}
\requirement{2}{Förståelse av Java, i nivå med vad som lärs ut i kursen EDA016, samt grundläggande kunskap av SQL skall räcka för att underhålla samt vidareutvekla systemet.}
\subsection{Prestanda}
\requirement{1}{Då systemet används i en av datorsalarna I “E-huset”, LTH, skall svaret på en godtycklig förfrågan i åtminstone 95\% av fallen ges inom 1,0 s.}
\section{Projektkrav}
\subsection{Utvecklingsmiljö}
\requirement{1}{Systemet skall vara utveklat för Apache-Tomcat servern.}
\requirement{2}{Systemet skall vara utveklat i Java.}
\requirement{3}{Databaslösningen MySQL skall användas av programmet för lagring av data mellan sessioner.}
\requirement{4}{Systemet samt projekt- och produktdokumentation ska skrivas på svenska. Java-koden ska följa standarden som finns på http://www.geosoft.no/development/javastyle.html, alla variabelnamn ska vara skrivna på engelska.}

\end{document}