
%%%%%%%%%%%%%%%%%%%%%%%%%%%%%%%%%%%%%%%%%%%%%%%%%%%%%%%%%%%%%%%%%%%%%%%%%%%%%%
% Detta är ett exempel på ett latexdokument.
% 
% Alla dokument består av följande delar:
%
%          \documentclass[optioner]{dokumentklass}
%            ...inställningar...
%          \begin{document}
%            ...text...
%          \end{document}
%
% Som ni kanske redan har förstått är används procent (%) för
% kommentarer.
%%%%%%%%%%%%%%%%%%%%%%%%%%%%%%%%%%%%%%%%%%%%%%%%%%%%%%%%%%%%%%%%%%%%%%%%%%%%%%

\documentclass[titlepage,a4paper]{article}
\usepackage[pdftex]{graphicx}
\usepackage{anysize}
\marginsize{3cm}{3cm}{3cm}{3cm}
\usepackage[T1]{fontenc}                % För svenska bokstäver
\usepackage[swedish]{babel}      
\usepackage{epstopdf}     % För svensk avstavning och svenska
\usepackage[osf]{mathpazo} % Palatino with smallcaps and oldstyle numbers
\usepackage[scaled]{helvet} % Helvetica, scaled 95%
                  

\begin{document}

\begin{titlepage}
\begin{center}
\includegraphics[width=0.30\textwidth]{./logo}~\\[1cm]
\textsc{\LARGE SRS}\\
\textup{Programvaruutv.}\\[1cm]
Martin Lichota\\


\end{center}
\end{titlepage}
                   % Skriver ut rubriken som vi
                                % deklarerade ovan med \title, \author
                                % och eventuellt \date

\section{Inledning}          % Detta kommando gör en rubrik

Dokumentet beskriver kraven för <Program X>, ett tidsrapportingssystem för projekt som diverse användare kan logga in på.


          % Detta blir en underrubrik
\section{Referensdokument}
< TOMT >
\section{Bakgrund och mål}   
\subsection{Huvudmål}
Huvudmålet är att tillhandhålla ett system där olika användare, såsom projektledare och övriga projektmedlemmar, ska kunna tidsrapportera och loggföra det fortgågna arbetet. 

\subsection{Aktörer och deras mål}
Följande aktörer kommer att använda systemet:

\textbf{Vanlig användare (User)} En användare kan logga in i systemet och tidsrapportera. Denne har även tillgång till statistik rörande den egna tidsrapporteringen.

\textbf{Administrator (Admin)} En administratör är en specifik användare som har privligerade rättigheter. Denne kan lägga till och ta bort andra användare.

\textbf{Projektledare (PL)} En administratör är en specifik användare som har privligerade rättigheter. Denne kan lägga till och ta bort andra användare.

%-------------------------------------------------------------------
\pagebreak
\section{Analys och bearbetning av experimentell data}
%-------------------------------------------------------------------
\subsection{Lysdiod - emission och absorption}
\begin{figure}[htb]
\centering
\includegraphics[scale=.6]{osynligt.eps}
\caption{Absorption (rött) och emission (blått) som funktion av våglängd (nm) - Osynligt ljus.}
\end{figure} 

\begin{figure}[htb]
\centering
\includegraphics[scale=.6]{synligt.eps}
\caption{Emission (blått) som funktion av våglängd (nm) - Synligt ljus.}
\end{figure} 

Det synliga ljuset var blått och har våglängden 
\begin{math} \lambda = 455*10^{-9} m.
\end{math}

Ur formeln 

\begin{equation}
E_{g}= \frac {hc}{\lambda} \label{ekv1} 
\end{equation}

fås $E_{g}=2,7eV$.

Det osynliga ljusets våglängd är enligt mätvärden ca $950*10^{-9}m$.
På samma sätt som ovan fås motsvarande bandgapet till $E_{g}=1,3 eV$.

%-------------------------------------------------------------------

\subsection{Solcellen}
\begin{figure}[htb]
\centering
\includegraphics[scale=.6]{solcell1.eps}
\caption{Ström (mA) som funktion av spänning (V). Kurvan representerar solcellens karakteristik.}
\end{figure} 
\begin{figure}[htb]
\centering
\includegraphics[scale=.6]{solcell2.eps}
\caption{Effekt (mW) som funktion av resistans ($k\Omega)$. Kurvan representerar solcellens karakteristik.}
\end{figure} 
Ur figur 4 avläses maxeffekten till 8,6 mW.

%--------------------------------------------------------------
\pagebreak
\subsection{Tillverkning och analys av en GaP-diod}
Ström utan belysning: 3,5 mA.\\
Ström vid belysning; 4,5 mA.\\
Emitterat ljus: Orange.
\begin{figure}[hb]
\centering
\includegraphics[scale=.5]{opt-GAP.png}
\caption{Diodens karakteristik, ström (mA) som funktion av spänning (V). Pilen till vänster har värdet -4 V och pilen till höger cirka 0,4 V.}
\end{figure} 



\section{Sammanfattning}
Dioders karakteristik stämde till stor del överens med teorin. I det första momentet blev det som förväntat emission och absorption inom en viss våglängd för en given diod. Nyckeln till fenomenen ligger i ämnets bandgap, med vars hjälp man kan identifiera det. Dioden med det osynliga ljuset är gjord av kisel (bandgap på 1,11 eV vid 302K) och den som ger synligt ljus är gjord av zinkselenid (bandgap på 2,7 eV vid 302K).

Från det andra momentet kan det konstateras att man med en lämplig belastning kan utvinna ström och därmed tillverka en solcell. Detta sker då man teoretiskt sätt befinner sig i den fjärde kvadranten vilket medför till negativ ström. Den största arean mellan kurvan och axlarna (produkten av ström och spänning) ger den maximala effekten 8,6 mW.

I det tredje momentet inträffade ett intressant fenomen vid negativ spänning, kallad \textit{break down}. I övrigt stämde diodens karakteristik överens med det förväntade, nämligen tröskelspänning vid 0,6 V (0,4 V på laborationen). Dioden påvisade ökad ström vid belysning och fungerade därför som solcell; den emitterade även orange ljus.

\end{document}                 % The input file ends with this command.