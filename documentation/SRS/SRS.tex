\documentclass[a4paper]{article}
\usepackage[pdftex]{graphicx}
\usepackage{anysize}
\marginsize{3cm}{3cm}{3cm}{3cm}
\usepackage[utf8]{inputenc}
\usepackage[T1]{fontenc}       

\usepackage[swedish]{babel}      
\usepackage{epstopdf}     % För svensk avstavning och svenska
\usepackage[osf]{mathpazo} % Palatino with smallcaps and oldstyle numbers
\usepackage[scaled]{helvet} % Helvetica, scaled 95%

\usepackage{etoolbox}

\newcommand\getcurrentref[1]{%
 \ifnumequal{\value{#1}}{0}
  {??}
  {\the\value{#1}}%
}  
\newcommand\requirement[2]{
	\numberedrow{Krav}{#1}{#2}
}
\newcommand\scenario[2] {
	\numberedrow{Scenario}{#1}{#2}
}
\newcommand\numberedrow[3]{
	\noindent
	\textbf{#1 \getcurrentref{section}.\getcurrentref{subsection}.#2.} #3
	
}

\usepackage{fancyhdr}
\fancyhf{}
\fancyhead[L]{Ansvarig: SG}
\fancyhead[C]{Datum: \today}
\fancyhead[R]{Version: 1.0}


\title{SRS - Software Requirements Specification: NewPussSystem}                  	
\author{Systemarkitektgruppen \\ Lars Gustafsson | Martin Lichota | Marcel Tovar Rascon}
\date{}

\begin{document}

\maketitle
\thispagestyle{fancy}
\tableofcontents
\newpage

\section*{Dokumenthistorik}

\begin{tabular}{ l l l l }
Ver. & Datum & Ansv. & Beskrivning \\\hline
1.0 & \today & SG & Första baseline-versionen

\end{tabular}
\section{Inledning}       


Dokumentet beskriver kraven för <Program X>, ett tidsrapportingssystem för projekt som diverse användare kan logga in på.

\section{Referensdokument}
I denna version används inget referensmaterial.
\section{Bakgrund och mål}   
\subsection{Huvudmål}
Huvudmålet är att tillhandhålla ett system där olika användare, såsom projektledare och övriga projektmedlemmar, ska kunna tidsrapportera och loggföra det fortgående arbetet. 

\subsection{Aktörer och deras mål}
Följande aktörer kommer att använda systemet:
\begin{itemize}
\item [] \textbf{Vanlig användare (User)} En användare kan logga in i systemet och tidsrapportera. Denne har även tillgång till statistik rörande den egna tidsrapporteringen.
\item [] \textbf{Administrator (Admin)} En administratör är en specifik användare som har privligerade rättigheter. Denne kan lägga till och ta bort andra användare.
\item [] \textbf{Projektledare (PL)} Projektledare är en roll som kan tilldelas till en User vilket ger den administrativa rättigheter för ett givet projekt.
\end{itemize}

\section{Terminologi}
Här följer ord och uttryck som används i rapporten och är till för att öka förståelsen.
\begin{itemize}
\item [Random word] Detta betyder något.
\item [Other thing] Betyder inte direkt något men nått gör det nog.
\end{itemize}
\section{Context diagram}
\section{Funktionella krav}
\subsection{Generella krav}
 \requirement{1}{Alla scenarion beskrivna under kapitel \ref{scenarion} skall stödjas av systemet.}
 \requirement{2}{Systemet ska inte krascha eller låsa sig på grund av att användaren matar in en viss sekvens av data.}
 \requirement{3}{Flera användare ska kunna logga in på systemet samtidigt.}
 
\subsection{Autentisering}
\requirement{1}{Man ska kunna logga in med användarnamn och lösenord.}
\requirement{2}{Möjlighet att få reda på sitt lösenord genom att ange sin email (eller generera ett nytt, vilket som känns bäst)}
\requirement{3}{möjlighet att byta lösenord när man är inloggad}
\requirement{4}{Automatisk utloggning om du är inaktiv i 15 min}
\requirement{5}{Användaren ska bara kunna vara inloggad från ett ställe samtidigt (dvs användare A kan inte vara inloggad på både dator A och dator B samtidigt)}

\subsection{Data}
\requirement{1}{Alla uppdatering i systemet ska sparas omedelbart i databasen så att inga ändringar går förlorade vid systemkrasch.}
\requirement{2}{Varje radering av information så som användare, projekt eller tidrapporter ska bekräftas av användaren innan den utförs.}
\requirement{3}{Information om användare ska sparas 
-namn
-lösenord
-roll
-grupptillhörighet
-tidrapporter
(-aktiv/inaktiv kan användas istället för att radera användare)}
\requirement{4}{Information om grupper
- gruppnamn
- medlemmar}
\requirement{5}{Rätt data ska kunna nås av rätt användare.}
\requirement{6}{Tidrapporter ska kunna sparas, uppdateras och raderas.}
\subsection{Administration}
\requirement{1}{Admin ska kunna lägga till användare (förutom sig själv) i systemet.}

\requirement{2}{När en ny användare läggs till ska admin ange användarnamn och ett slumpartat lösenord genereras av systemet.}

\requirement{3}{Om admin försöker att lägga till en användare med ett användarnamn som redan existerar i systemet ska användaren inte läggas till och ett felmeddelande ska visas.}

\requirement{4}{Admin ska kunna ta bort användare (förutom sig själv) i systemet.}

\requirement{5}{När en användare tas bort ska admin kunna välja användare i en lista på användare i systemet.}

\requirement{6}{När en användare läggs till eller tas bort ska en bekräftelseruta visas där admin kan välja att bekräfta och slutföra ändringen eller välja avbryt så att ingen ändring görs.}

\requirement{7}{Admin ska kunna skapa projektgrupper i systemet.}

\requirement{8}{När en ny projektgrupp skapas ska admin ange namn och andra uppgifter för projektgruppen.}

\requirement{9}{Om admin försöker skapa en projektgrupp med ett namn som redan existerar ska projektgruppen inte skapas och ett felmeddelande ska visas.}

\requirement{10}{Admin ska kunna ta bort projektgrupper i systemet.}

\requirement{11}{När en projektgrupp tas bort ska admin kunna välja projektgrupp ur en lista på alla projektgrupper i systemet.}


\subsection{Tidsraportering}
\subsection{Projektledning}
Kunna se samtliga tidrapporter i projektgruppen
Kunna godkänna tidrapporter
Ha tillgång till statistik om tidrapporterna inom projektet. Tiprapporter ska kunna summeras per användare, per roll, per aktivitet, per vecka.
Se alla medlemmar i projektgruppen
Tilldela roller till medlemmarna i projektgruppen



\section{Funktionella scenarion}
\label{scenarion}
\subsection{Generella scenarion}
\subsection{Autentisering}
\subsection{Data}
\subsection{Administration}
\subsection{Tidsraportering}
\subsection{Projektledning}
\section{Kvalitetskrav}
\subsection{Underhåll}
\requirement{1}{Förståelse av Java, i nivå med vad som lärs ut i kursen EDA016, samt grundläggande kunskap av SQL skall räcka för att underhålla samt vidareutvekla systemet.}
\subsection{Prestanda}
\requirement{1}{Då systemet används i en av datorsalarna I “E-huset”, LTH, skall svaret på en godtycklig förfrågan i åtminstone 95\% av fallen ges inom 1,0 s.}
\section{Projektkrav}
\subsection{Utvecklingsmiljö}
\requirement{1}{Systemet skall vara utveklat för Apache-Tomcat servern.}
\requirement{2}{Systemet skall vara utveklat i Java.}
\requirement{3}{Databaslösningen MySQL skall användas av programmet för lagring av data mellan sessioner.}
\requirement{4}{Systemet samt projekt- och produktdokumentation ska skrivas på svenska. Java-koden ska följa standarden som finns på http://www.geosoft.no/development/javastyle.html, alla variabelnamn ska vara skrivna på engelska.}
\requirement{5}{Systemet skall vara väl dokumenterad så det underlättar vidareutvekling av systemet i framtiden.}



\end{document}