\documentclass[a4paper]{article}
\usepackage[pdftex]{graphicx}
\usepackage{anysize}
\marginsize{3cm}{3cm}{3cm}{3cm}
\usepackage[utf8]{inputenc}
\usepackage[T1]{fontenc}

\usepackage[swedish]{babel}      
\usepackage{epstopdf}     % För svensk avstavning och svenska
\usepackage[osf]{mathpazo} % Palatino with smallcaps and oldstyle numbers
\usepackage[scaled]{helvet} % Helvetica, scaled 95%
\usepackage[titletoc]{appendix}

\usepackage{fancyhdr}

\fancyhf{}
\fancyhead[L]{Ansvarig: SG}
\fancyhead[C]{Datum: \today}
\fancyhead[R]{Inofficiell granskning}


\title{Granskningsprotokoll 2 av:\\ SVVS - Software Verification and Validation Specification: NewPussSystem}                  	
\author{Systemarkitektgruppen \\ Lars Gustafsson | Martin Lichota | Marcel Tovar Rascon}
\date{}

\begin{document}

\maketitle
\thispagestyle{fancy}

\section{Inledning}       
Här följer granskning av SVVS - Software Verification and Validation Specification version 0.3, 14 september 2014. Utförd av Systemarkitektgruppen, datum som ovan. Tänkt som en ersättning av den första granskningen då SVVS ändrades innan granskningsmöte skett i fas 1.

\section{Övergripande kritik}
Smärre stavfel här och där. Bra upplägg, överskådbar, och bra svenska.
Författarna är osäkra på vem som är tänkt läsare och var man då skall lägga språket. Behövs till exempel begreppet White-box-testing förklaras? Med en publik av studerande av kursen ETS032 är det troligtvis inte nödvändigt.

Testerna som är specifierade under Appendix A är för generella och författarna ser snarare att de skall testa en sak i taget. Exempel ges under nästa kapitel.

\section{Kapitelspecifik kritik}
\begin{itemize}
\item[2] Ordet projekthandledning används ingen annan stans, bör ersättas med Projektplan.\\
Förutom det saknas både de välkända förkortningarna för dokumenten samt vilken version av Projektplanen som nytjas. Vet man detta är det mycket lättare att granska dokumentet.

\item[3] Första meningen bör avslutas med ett komma. \\
Vad alla olika förkortningar (SVVI, SSD...) står för framgår inte. Kan tänkas vara självklart om man antar att läsaren är studerande av krusen ETS032. \\
SDDD ska inte skapas förrän fas 3 vilket medför det svårt för UG att hålla koll på den i de två första faserna.\\
Referensen till Projektplanen är bra men hade underlättat om man viste vilket kapitel.

\item[4]
Kan tänkas vara övertolkning men denna meningen innefattar även författarnas privata datorer så fort de vistas i huset. Kanske bör omforumleras till något i stil med, ``Någon av datorerna i E-husets allmänna 
datorsalar.''

\item[5] Kan tänkas vara redundant information.

\item[5.5] Osäker på tredje ordet, förutsätter att det borde stå ``kunden''.

\item[6] Förstår inte referensen till Projektplanen, borde referare innom den med inte bara till ett dokument.

\item[Apendix A] 
Bra referering till kraven men kanske bör skriva ``krav'' istället för ``req'' då vi håller rapporterna på svenska.\\

Flertalet krav är på tok för generella och kan skapa fall där vi missar att testa viss funktionallitet.

\begin{itemize}
\item[FT1] En användare ska kunna omplaceras. Det måste förtydligas att man vill ha en användare i två grupper samtidigt.
\item[FT3] Tillgänglig (felstavning)
\item[FT6-9] Även projektledare ska kunna tillsätta dessa roller(krav 6.6.6, som kommer att förtydligas)
\item[FT21] Förtydliga, det handlar om att samma användare inte kan vara inloggad på mer än enhet samtidigt.
\item[FT29] Kanske lägga till ”Testa att spara” , gäller rätt många FT’s. ?
\item[FT30] Samma som ovan
\item[FT47] Formulera om ``typ''
\item[FT49] ``hen'' används, jag föreslår att byta ut mot ``denne'' för att vara konsekvent. (denne används i SRSn samt tidigare FT’s)
\item[FT55-56] - Referera gärna till krav 6.3.14. För omkringliggande FT’s kan man även referera till krav 6.1.9 och krav 6.6.1
\item[FT62] - Otydligt, är det rätt refererat?
\item[FT63 \& FT73] Referera gärna till konkreta scenarios.
\item[FT74] Finns inget krav med givet ID.
\item[FT77] Stavfel, uppdatera.
\item[FT83-90] Kan formuleras bättre.
\item[FT83-90] Ta bort ”tidrapporteringssidan”, det är underförstått. Skriv ``varpå'' tillsammans. Ta bort ``s'' i ``tidSrapporteringssidan''. Kolla igenom språket.
\end{itemize}

\end{itemize}






\end{document}