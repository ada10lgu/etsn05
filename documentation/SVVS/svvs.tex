\documentclass[a4paper]{article}
\usepackage[pdftex]{graphicx}
\usepackage{anysize}
\marginsize{3cm}{3cm}{3cm}{3cm}
\usepackage[utf8]{inputenc}
\usepackage[T1]{fontenc}
\usepackage{enumitem}

\usepackage[swedish]{babel}      
\usepackage{epstopdf}     % För svensk avstavning och svenska
\usepackage[osf]{mathpazo} % Palatino with smallcaps and oldstyle numbers
\usepackage[scaled]{helvet} % Helvetica, scaled 95%
\usepackage[titletoc]{appendix}

\usepackage{fancyhdr}

\fancyhf{}
\fancyhead[L]{Ansvarig: TG}
\fancyhead[C]{Datum: \today}
\fancyhead[R]{Version: 0.1}


\newlist{FT}{enumerate}{1}
\setlist[FT]{label=FT\arabic*:}

\newlist{ST}{enumerate}{1}
\setlist[ST]{label=ST\arabic*:}


\title{SVVS - Software Verification and Validation Specification: NewPussSystem}                  	
\author{Testgruppen \\ Axel Ulmestig | Axel Goteman | Sefik Ceric \\ Victor Johnsson | Johan Kellerth Fredlund}
\date{}

\begin{document}

\maketitle
\thispagestyle{fancy}
\tableofcontents
\newpage

\section*{Dokumenthistorik}

\begin{tabular}{ l l l l }
Ver. & Datum & Ansv. & Beskrivning \\\hline
0.1 & \today & TG & Första baseline-versionen

\end{tabular}
\section{Inledning}       

Detta dokument innehåller information om vilka tester och granskningar som ska genomföras under utvecklingen av NewPussSystem.

\section{Referensdokument}
\begin{enumerate}
\item System Requirements Specification
\item Projekthandledning
\end{enumerate}

\section{Översikt}

Under projektet ska tre formella granskningar hållas

\begin{enumerate}


\item Software Specifikation Review (SSR)
\begin{itemize}
\item [](a) SDP
\item [](b) SRS
\item [](c) SVVS
\end{itemize}


\item Preliminary Design Review (PDR)
\begin{itemize}
\item [](a) SVVI
\item [](b) STLDD
\end{itemize}


\item Product Review (PR)
\begin{itemize}
\item [](a) SVVR
\item [](b) SSD
\item [](c) PFR
\end{itemize}


\end{enumerate}

Dessa formella granskningar utförs på det sätt som anges i Ref. 2. De formella granskningarna föregås av två informella granskningar av de relevanta partierna. I den första sammanställs information och kritik eller förändringar ges, sen ges tid för förändring innan den andra informella granskningen där den slutliga versionen sammanställs, denna version är sedan den som skickas för formell granskning. Inför varje granskning så ska dessutom utvecklarna kolla igenom SDDD ifall den förändrats.


\section{Testmiljöer}

Den testmiljön som används är på en godtycklig dator i E-huset.

\section{Testfall}

Tester utförs på det sätt som anges nedan.

\subsection{Enhetstest}

Utförs av utvecklingsgruppen efter hand som kod tas fram. Här utnyttjas både "white-box"- och "black-box"-testning.

\subsection{Funktionstest}

Där enskilda funktioner testas enligt en funktionstestspecifikation och tillhörande funktionstestinstruktioner. 

\subsection{Systemtest}

Där ett komplett system testas för att se hur flera funktioner och tjänster samverkar enligt en systemtestspecifikation. Testen i enlighet med de framtagna systemtestinstruktionerna. 

\subsection{Regressionstest}

När nya funktioner har införts eller gamla har ändrats behöver "gamla" funktioner testas, för att säkerställa att ändringen inte påverkat något som tidigare godkänts i test.

\subsection{Acceptanstest}

Under vilket kunde utför ett urval av ovanstående tester samt ett par för leverantörer okända testfall, med syfte att validera systemet.

\section{Krav som tas upp vid granskningar}

Vid granskningar ska alla krav granskas.
Följande krav (enligt Ref.2) är endast verifierbara genom granskningar: 8.1.1, 8.1.2, 9.1.1, 9.1.2, 9.1.3, 9.1.4.

\section{Bilagor}

Det finns två bilagor:

\begin{itemize}
\item A: Funktiontestspecifikation
\item B: Systemtestspecifikation

\end{itemize}

\newpage

\begin{appendices}

\section{Funktionstestsspecifikation}


\begin{FT}

\item 

Testa alla scenarion beskriva i kapitel 7 i SRS [SRS req 6.1.1]

\item 

Logga in när annan användare är inloggad [SRS req. 6.1.3]

\item 

Logga in med rätt lösenord och användare name [SRS req. 6.2.1]

\item 

Försök logga in med fel lösenord och användare name [SRS req. 6.2.1]


\item 

Användare kan byta lösenord [SRS req. 6.2.2]

\item 

Användare blir utloggad efter 15 minuters inaktivitet [SRS req. 6.2.3]

\item 

Två användare försöker logga in på samma konto samtidigt [SRS req. 6.2.4]

\item 

Bekräftelseruta om raderande av information [SRS req. 6.3.2]

\item 

Försök skapa användare med för lite data [SRS req. 6.3.3]

\item 

Försök skapa en projektgrupp med för lite data [SRS req. 6.3.4, req. 6.4.11]

\item 

Försök skapa ett adminkonto när det redan finns ett adminkonto [SRS req. 6.4.1]

\item 

Skapa ett adminkonto när det inte finns ett adminkonto [SRS req. 6.4.1]

\item
Skapa ett nytt konto från adminkontot [SRS req. 6.4.2]

\item
Skapa ett nytt konto från ett konto som inte är admin [SRS req. 6.4.2]

\item
Skapa två användare med samma namn [SRS req. 6.4.3]

\item
Skapa två konton och kontrollera att lösenordet är olika [SRS req. 6.4.3]

\item
Admin försöker att lägga till användare med ett upptaget användarnamn [SRS req. 6.4.4]

\item
Admin inaktiverar konton av alla typer [SRS req. 6.4.5]

\item
Alla utom admin försöker inaktivera konton av alla typer [SRS req. 6.4.5]

\item
Admin aktiverar konton av alla typer [SRS req. 6.4.6]

\item
Alla utom admin försöker aktivera konton av alla typer [SRS req. 6.4.6]

\item
Admin kan generera en lista av aktiva användare [SRS req. 6.4.7]

\item
Admin får en bekräftelseruta då hen ska lägga till en användare i systemet [SRS req. 6.4.8]

\item
Admin får en bekräftelseruta då hen ska aktivera en användare i systemet [SRS req. 6.4.8]

\item
Admin får en bekräftelseruta då hen ska avaktivera en användare i systemet [SRS req. 6.4.8]

\item 

Skapa projektgrupp från konton med olika behörighet [SRS req. 6.4.9]

\item 

Försök skapa projektgrupp med ett befintligt namn [SRS req. 6.4.12]

\item 

Ta bort projektgruppen från konton med olika behörighet [SRS req. 6.4.13]

\item 

Admin kommer åt lista på alla projektgrupper [SRS req. 6.4.14]

\item 

Spara, uppdatera och radera tidsrapporter [SRS req. 6.5.1]

\item

Testa att göra en användare till projektledare för två olika projekt [SRS req. 6.6.1]

\item 

Icke projektledare utför operationer reserverade för projektledare [SRS req. 6.6.1]

\item

Antalet tidsrapporter som syns stämmer med antalet i databasen [SRS~req.~6.6.2]

\item 

Projektledare kan godkänna tidrapporter [SRS req. 6.6.3]

\item 

Projektledare kan dra tillbaka tidigare godkännanden av tidrapporter [SRS req. 6.6.4]

\item 

Antalet projektmedlemmer i databasen överenstämmer att antalet projektmedlemmar stämmer med det som projketledaren ser [SRS~req. 6.6.5]

\item 

Kontrollera att tillgänglig statistik om projektet kan ses av projektledaren [SRS req. 6.6.6]

\item 

Kontrollera att projektledaren kan ändra roller hos användare i projketet [SRS req. 6.6.7]






\end{FT}
\end{appendices}

\newpage

\begin{appendices}

\section{Systemtestspecifikation}

\begin{ST}

\item 
Testa input med olika tecken från användaren [SRS req 6.1.2]

\item
Vanliga användare kommer åt rätt funktioner [SRS req 6.3.5]

\item
Admin kommer åt rätt funktioner [SRS req 6.3.5]

\item
Projektledare kommer åt rätt funktioner [SRS req 6.3.5]

\item
Admin kommer åt funktionerna som projektledarna kommer åt i alla projekt [SRS req 6.6.8]

\item
Svaret på en godtycklig förfrågan från en dator i E-huset kommer i 95\% av fallen inom en sekund [SRS req 8.2.1]

\end{ST}






\end{appendices}


\end{document}