\documentclass[a4paper]{article}
\usepackage[pdftex]{graphicx}
\usepackage{anysize}
\marginsize{3cm}{3cm}{3cm}{3cm}
\usepackage[utf8]{inputenc}
\usepackage[T1]{fontenc}
\usepackage{enumitem}

\usepackage[swedish]{babel}      
\usepackage{epstopdf}     % För svensk avstavning och svenska
\usepackage[osf]{mathpazo} % Palatino with smallcaps and oldstyle numbers
\usepackage[scaled]{helvet} % Helvetica, scaled 95%
\usepackage[titletoc]{appendix}

\usepackage{fancyhdr}

\fancyhf{}
\fancyhead[L]{Ansvarig: TG}
\fancyhead[R]{Datum: \today |Version: 0.5 | Dokumentnummer: PUSS144403}


\newlist{FT}{enumerate}{1}
\setlist[FT]{label=FT\arabic*}

\newlist{ST}{enumerate}{1}
\setlist[ST]{label=ST\arabic*}

%\newcommand{\krav}[1]{\[SRS~krav~#1\]}

\title{SVVS - Software Verification and Validation Specification: NewPussSystem}                  	
\author{Testgruppen \\ Axel Ulmestig | Axel Goteman | Sefik Ceric \\ Victor Johnsson | Johan Kellerth Fredlund}
\date{}

\begin{document}

\maketitle
\thispagestyle{fancy}
\tableofcontents
\newpage

\section*{Dokumenthistorik}

\begin{tabular}{ l l l l }
Ver. & Datum & Ansv. & Beskrivning \\\hline
0.1 & 11 september 2014 & TG & Första utkast \\
0.2 & 14 september 2014 & TG & Anpassning efter SRS v0.10, regressionstest inlagda \\
0.3 & 14 september 2014 & TG & Anpassning efter SRS v0.12, referenser uppdaterade\\
0.4 & 16 september 2014 & TG & Anpassning efter SRS v0.14, tillagda FT och ST samt revidering av vissa test\\
0.5 & 16 september 2014 & TG & Anpassning efter SRS v0.15\\


\end{tabular}
\section{Inledning}       

Detta dokument innehåller information om vilka tester och granskningar som ska genomföras under utvecklingen av NewPussSystem.

\section{Referensdokument}
\begin{enumerate}
\item System Requirements Specification v0.14
\item Programvaruutveckling för stora system - Projekthandledning
\end{enumerate}

\section{Definitioner}

\subsection{Dokument}

\begin{enumerate}

\item SDP: Software Development Plan

\item SRS: System Requirements Specification

\item SVVS: Software Verification and Validation Specification

\item STLDD: Software Top Level Design Document

\item SVVR: Software Verification and Validation Report

\item SSD: System Specification Document

\item PFR: Project Final Report


\end{enumerate}

\subsection{White-box test}

White-box testning är när man utgår från kodens interna struktur och försöker testa så att man går igenom alla rader minst en gång eller alla vägval i programflödet.

\subsection{Black-box test}

Black-box testning är där man har ett externt perspektiv d.v.s. testningen är baserad på indata och utdata. Man väljer testfall bland de kombination av indata som kan förekomma. Resultatet av exekveringen jämförs mot en specifikation.


\section{Översikt}

Under projektet ska tre formella granskningar hållas:

\begin{enumerate}


\item Software Specifikation Review (SSR) - 19 september 2014
\begin{itemize}
\item [](a) SDP
\item [](b) SRS
\item [](c) SVVS
\end{itemize}


\item Preliminary Design Review (PDR) - 10 oktober 2014
\begin{itemize}
\item [](a) SVVI
\item [](b) STLDD
\end{itemize}


\item Product Review (PR)
\begin{itemize}
\item [](a) SVVR
\item [](b) SSD
\item [](c) PFR
\end{itemize}


\end{enumerate}

Dessa formella granskningar utförs på det sätt som anges i Ref. 2. kap. 6.2 De formella granskningarna föregås av två informella granskningar av de relevanta partierna. I den första sammanställs information och kritik eller förändringar ges, sen ges tid för förändring innan den andra informella granskningen där den slutliga versionen sammanställs, denna version är sedan den som skickas för formell granskning. Inför varje granskning så ska dessutom utvecklarna kolla igenom SDDD ifall den förändrats.


\section{Testmiljöer}

Den testmiljön som används är på en allmän dator i en godtycklig datorsal i E-huset. För de automatiserade testfallen kommer eclipse och JUnit att användas.

\section{Testfall}

\subsection{Enhetstest}

Utförs av utvecklingsgruppen efter hand som kod tas fram. Här utnyttjas både white-box- och black-box-testning.

\subsection{Funktionstest}

Där enskilda funktioner testas enligt en funktionstestspecifikation och tillhörande funktionstestinstruktioner. 

\subsection{Systemtest}

Där ett komplett system testas för att se hur flera funktioner och tjänster samverkar enligt en systemtestspecifikation. Testen i enlighet med de framtagna systemtestinstruktionerna. 

\subsection{Regressionstest}

När nya funktioner har införts eller gamla har ändrats behöver ``gamla'' funktioner testas, för att säkerställa att ändringen inte påverkat något som tidigare godkänts i test.

\subsection{Acceptanstest}

Under vilket kunden utför ett urval av ovanstående tester samt ett par för leverantörer okända testfall, med syfte att validera systemet.

\section{Krav som tas upp vid granskningar}

Vid granskningar ska alla krav granskas.
Följande krav (i Ref.1) är endast verifierbara genom granskningar: 7.1.1, 7.1.2, 8.1.1, 8.1.2, 8.1.3 och 8.1.4.

\section{Bilagor}

Det finns tre bilagor:

\begin{itemize}
\item A: Funktiontestspecifikation
\item B: Systemtestspecifikation
\item C: Regressionstest

\end{itemize}

\newpage

\begin{appendices}

\section{Funktionstestsspecifikation}
%After each test case the tested requirements according to Ref. 2 are listed.
Efter varje testfall står vilket eller vilka krav som det testar i referens 2.

\begin{FT}


\item
Administratören försöker lägga till en användare ,medlem i en projektgrupp, i en ny projektgrupp utan att ta bort denne från tidigare grupp [SRS krav 6.1.1]

\item
Den funktionalitet som användaren har rätt till finns tillgänglig i menyn [SRS krav 6.1.4]

\item
Menyn finns tillgänglig på samtliga sidor i systemet [SRS krav 6.1.8]

\item
Administratören försöker lägga in en ny användare då det finns 20 användare i en projektgrupp [SRS krav 6.1.9]

\item
Administratören försöker ta bort en användare då det finns 1 användare i en projektgrupp [SRS krav 6.1.9]

\item
Administratören tar bort en tidrapportmall som inte används i ett projekt [SRS krav 6.
%DATA Victor Johnsson
\item
Tidsrapportering sker i antal minuter [SRS krav 6.3.4, 6.5.5]

\item
Tidrapportens utseende, med avseende på vilka aktiviteter man kan föra in tidinformation i, bestäms av administratören innan projektgruppen har skapats [SRS krav 6.3.5, 6.4.16]

\item
Försök att med ett skapat projekt ändra tidrapportens utseende [SRS krav 6.3.5]

\item
Tidrapportens grundmall skall innehålla information om användarnamn, projektgruppsnamn, datum, veckonummer och om den är signerad eller inte [SRS krav 6.3.6]

\item
När en tidsrapportsmall skapas, sparas den på servern [SRS krav 6.3.7]

\item
Spara 20 tidsrapportsmallar, kontrollera att de är sparade [SRS krav 6.3.8]

\item
Försök spara mer än 21 tidsrapportsmallar, kontrollera att den sista inte sparas [SRS krav 6.3.8]

\item
En tidrapportmall består av 1-10 rader samt 1-5 kolumner som det går att mata in tidsinformation i [SRS krav 6.3.9]

\item
Namnet på tidrapportmallen kan endast bestå av 5 tecken, ascii (decimal) värden 48-57 och
97-122 är tillåtna [SRS krav 6.3.10, 6.4.17]

\item
Namnet på aktivitet och subaktivitet kan endast bestå av 3-5 tecken respektive 1 tecken, ascii (decimal) värden 48-57 och 97-122 är tillåtna [SRS krav 6.3.11]

\item
Tidinformationen innehåller endast 1-5 tecken per aktivitetsruta, ascii (decimal) värden 48-57 [SRS krav 6.3.12, 6.5.5, 6.3.19]

\item
Försök skapa en ny tidsrapportsmall med ett namn som redan finns [SRS krav 6.3.13, 6.4.17]

\item
Försök skapa projektgrupp med namn som är kortare än fem tecken eller längre än 10 tecken [SRS krav 6.3.14]

\item
Försök skapa projektgrupp med namn som innehåller tecken som inte innefattas av ascii (decimal) värden 48-57 eller 97-122 [SRS krav 6.3.14]

\item
Försök skapa en ny projektgrupp med ett namn som redan finns [SRS krav 6.3.15]

\item
Försök fylla i ett veckonummer som är kortare än 1 tecken eller längre än 2 teck [SRS krav 6.3.16]

Försök fylla i ett veckonummer som inte enbart innehåller ascii (decimal) värden 48-57 [SRS krav 6.3.16]

\item
Lagra 100 tidsrapporter för en användare, kontrollera att de finns i databasen [SRS krav 6.3.17]

\item
Lagra mer än 100 tidsrapporter för en användare, de skrivs över enligt FIFO-principen [SRS krav 6.3.17]

\item
Admin försöker ta bort en tidrapportmall som används i ett projekt [SRS krav 6.3.18]

\item
Logga in på systemet med user 'admin' och pass 'adminpw' [SRS krav 6.4.2]

\item 
Administrationsvyn finns tillgänglig för administratören [SRS krav 6.4.3, 6.4.6]

\item
Administrationsvyn finns inte tillgänglig för användare som inte är administratörer [SRS krav 6.4.6]

\item
Skapa ett nytt konto från adminkontot med unikt namn och kontrollera att användaren har tilldelats ett slumpmässigt lösenord [SRS krav 6.4.4, 6.4.5]

\item
Administratören försöker att lägga till användare med ett upptaget användarnamn [SRS krav 6.3.1, 6.3.2, 6.4.4, 6.4.5]

\item
Administratören tar bort konton i administratörsvyn som inte är ett administratörskonto [SRS krav 6.4.8]

\item
Administratören försöker i administratörsvyn ta bort administratörskontot [SRS krav 6.4.8]

\item
Administrationsvyn innehåller listas alla användare med både användarnamn och lösenord [SRS krav 6.4.7]

\item
Administratören får en bekräftelseruta då denne ska ta bort till en användare i systemet [SRS krav 6.1.10, 6.4.9]

\item
Administratören lägger till en ny användare med korrekt data för användarnamn och lösenord [SRS krav 6.3.1, 6.3.3, 6.4.10]

\item
Administratören fösöker lägger till en ny användare med korrekt data för användarnamn men fel lösenordsdata, ett felmeddelande visas [SRS krav 6.3.1, 6.3.3, 6.4.10, 6.4.11]

\item
Administratören försöker lägger till en ny användare med fel data för användarnamn och korrekt lösenordsdata, ett felmeddelande visas [SRS krav 6.3.1, 6.3.3, 6.4.5, 6.4.10, 6.4.11]

\item
Administratören försöker lägger till en ny användare med fel data för användarnamn och lösenord, ett felmeddelande visas [SRS krav 6.3.1, 6.3.3, 6.4.5, 6.4.10, 6.4.11]

\item 
Administratören skapar en projektgrupp med gruppnamn och projektledare [SRS krav 6.4.12, 6.4.13, 6.3.14, 6.4.17]

\item 
Administratören försöker skapa en projektgrupp med gruppnamn men utan projektledare [SRS krav 6.4.12, 6.4.13, 6.3.14]

\item 
Administratören försöker skapa en projektgrupp utan gruppnamn fast med projektledare [SRS krav 6.4.12, 6.4.13, 6.3.14]

\item 
Administratören försöker skapa en projektgrupp utan gruppnamn och projektledare [SRS krav 6.4.12, 6.4.13, 6.3.14]

\item
En användare försöker skapa en projektgrupp [SRS krav 6.4.12]

\item 
Administratören lägger till en medlem i en projektgrupp [SRS krav 6.4.14]

\item 
Administratören tar bort en medlem i en projektgrupp [SRS krav 6.4.14]

\item 
Någon som inte är administratör försöker lägga till en medlem i en projektgrupp [SRS krav 6.4.14]

\item 
Någon som inte är administratör försöker ta bort en medlem i en projektgrupp [SRS krav 6.4.14]

\item
Administratören tar bort en projektgrupp [SRS krav 6.4.15, 6.4.16]

\item
En användare försöker ta bort en projektgrupp [SRS krav 6.4.15]

\item 
Administratören utformar en tidsrapportmall [SRS krav 6.4.16]

\item
Administratören ger en användare rollen som projektledare [SRS krav 6.1.11, 6.6.6]

\item
En projektledare ger en användare rollen som projektledare [SRS krav 6.1.11, 6.6.6]

\item
Både adminstratören och projektledaren kan ge en användare rollen t1 [SRS krav 6.1.11, 6.6.6]

\item
Både adminstratören och projektledaren kan ge en användare rollen t2 [SRS krav 6.1.11, 6.6.6]

\item
Både adminstratören och projektledaren kan ge en användare rollen t3 [SRS krav 6.1.11, 6.6.6]

\item
Administratören försöker ge en användare rollen projektledare då det redan finns 2 av dessa i projektgruppen [SRS krav 6.1.12]

Administratören försöker ge en användare rollen t1 då det redan finns 6 av dessa i projektgruppen [SRS krav 6.1.13]

\item
Administratören försöker ge en användare rollen t2 då det redan finns 6 av dessa i projektgruppen [SRS krav 6.1.13]

\item
Administratören försöker ge en användare rollen t3 då det redan finns 6 av dessa i projektgruppen [SRS krav 6.1.13]

\item
En användare loggar in med rätt användarnamn och lösenord [SRS krav 6.2.4, 6.2.9]

\item
En användare försöker logga in med fel användarnamn och rätt lösenord [SRS krav 6.2.4, 6.2.9]

\item
En användare försöker logga in med rätt användarnamn och fel lösenord [SRS krav 6.2.4, 6.2.9]

\item
En användare försöker logga in med fel användarnamn och fel lösenord [SRS krav 6.2.4, 6.2.9]

\item 
Användare blir utloggad efter 20 minuters inaktivitet [SRS krav 6.2.6]

\item 
Logga in med en användare på en enhet och logga in på en annan enhet med samma användare, inloggningen på den andra enheten skall loggas in och inloggning på den första enheten skall loggas ut [SRS krav 6.2.7, 6.2.8]


%TIDSRAPPORTERING Sefik Ceric, Victor Johnsson
\item
Generera en ny tidrappport och kontrollera att den är osignerad [SRS~krav~6.5.1, 6.5.2]

\item
Användare skapar en  egen osignerad tidsrapport [SRS~krav~6.5.2]

\item
Användare uppdaterar egen osignerad tidrapport [SRS~krav~6.5.2]

\item
Användare tar bort en egen osignerad tidrapport [SRS~krav~6.5.2]

\item
Användare försöker ta bort egen signerad tidrapport [SRS~krav~6.5.3]

\item
Användare försöker redigera egen signerad tidrapport [SRS~krav~6.5.3]

\item
Användare som inte är projektledare försöker se en annan användares tidrapporter [SRS~krav~6.5.4]

%PROJEKTLEDNING Victor Johnsson
\item
Ett projekt kan inte ha mer än två eller mindre än en projektledare [SRS~krav~6.6.1]

\item
Projektledare har tillgång till samtliga projektmedlemmars tidrapporter i sin projektgrupp [SRS~krav~6.6.2]

\item
Projektledaren kan godkänna ej tidigare godkända tidrapporter från medlemmar i sin projektgrupp [SRS~krav~6.6.3]

\item
Projektledaren tar tillbaka sitt godkännande från en tidigare godkänd gruppmedlems tidsrapport, i sin projektgrupp [SRS~krav~6.6.4]

\item
Projektledaren kan, inom sitt projekt, summera alla tidsrapporter [SRS~krav~6.6.5]

\item
Projektledaren kan, inom sitt projekt, summera tidsrapporter per användare/roll/aktivitet/vecka [SRS~krav~6.6.5]

\item
Projektledaren kan, inom sitt projekt, ändra och tilldela roller hos användare [SRS~krav~6.6.6]



\item
Projektledaren listar alla ej godkända rapporter [SRS krav 6.6.8]

\item
Projektledaren listar alla godkända rapporter [SRS krav 6.6.9]

\item
Projektledaren listar alla rapporter [SRS krav 6.6.13]






\end{FT}
\end{appendices}



\newpage

\begin{appendices}

\section{Systemtestspecifikation}

Efter varje testfall står vilket eller vilka krav som det testar i referens 2.

\begin{ST}

% \item 
% Testa input med olika tecken från användaren [SRS krav 6.1.1]

\item
Försök logga in med fler än 50 användare samtidigt [SRS krav 6.1.2]

\item
Logga in med 50 användare samtidigt [SRS krav 6.1.2]

\item 
Användare skriver in URL för huvudmenyn. Har tillgång till följande användarfunktionaliteter; tidrapportering, projektinformation, byta lösenord, hjälp samt utloggning. [SRS krav 6.1.3, 6.1.4, 6.1.8, 6.4.1]


\item 
Projektledare skriver in URL för huvudmenyn. Har tillgång till projektadministrationsfunktionaliter och samtliga användarfunktionaliteter i menyn [SRS krav 6.1.5, 6.1.8]

\item 
Administratören skriver in URL för huvudmenyn. Har tillgång till administrationsfunktionaliteter, projektadministrationsfunktionaliter och samtliga användarfunktionaliteter [SRS krav 6.1.6, 6.1.7, 6.1.8]

\item
Administratören försöker lägga till sig själv i en projektgrupp [SRS krav 6.1.14]
krav

\item
En icke inloggad användare försöker använda en URL till en funktionalitet där denne har behörighet, och får en förfrågan om användarnamn och lösenord, ingen annan information utöver inloggningsrutan skall visas [SRS krav 6.2.1, 6.2.3, 6.2.5]

\item
Genomför scenario 6.2.1 [SRS krav 6.2.9]

\item
Genomför scenario 6.4.1 [SRS krav 6.4.17]

\item
Genomför scenario 6.4.2 [SRS krav 6.4.18]

\item
Genomför scenario 6.4.3 [SRS krav 6.4.19]

\item
Genomför scenario 6.4.4 [SRS krav 6.4.20]

\item
Genomför scenario 6.5.1 [SRS krav 6.5.6]

\item
Genomför scenario 6.5.2 [SRS krav 6.5.7]

\item
Genomför scenario 6.6.1 [SRS~ krav~6.6.7]

\item
Genomför scenario 6.6.2 [SRS krav 6.6.10]

\item
Genomför scenario 6.6.3 [SRS krav 6.6.11]

\item
Genomför scenario 6.6.4 [SRS krav 6.6.12]

\item
Projektledaren lyckas sortera projektets all rapporter efter följande attribut i både stigande och fallande ordning; Användare, vecka, och huruvida rapporten är godkänd eller ej [SRS krav 6.6.13]


%flyttade från funktionstest av Axel U
\item
En användare är antingen inloggad eller utloggad [SRS krav 6.2.1]

\item
Systemet håller login status i en server session [SRS krav 6.2.2]

\item
Genomför scenario 6.2.2 [SRS krav 6.2.10]

\item
Genomför scenario 6.2.2 [SRS krav 6.2.10]

\item
Genomför användarscenario för en administratör [SRS krav 6.4.1]

%DATA Victor Johnsson
\item
Tidsrapportering sker i antal minuter [SRS krav 6.3.4, 6.5.5]

\item
Administratören kan skapa en ny tidsrapportmall och ändra den [SRS krav 6.3.5, 6.4.16]

\item
Försök att ändra en tidsrapportsmall som redan används i ett projekt [SRS krav 6.3.5]

\item
Tidrapportens grundmall innehåller information om användarnamn, projektgruppsnamn, datum, veckonummer och om den är signerad eller inte [SRS krav 6.3.6]

\item
När en tidsrapportsmall skapas, sparas den på servern [SRS krav 6.3.7]

\item
Spara 20 tidsrapportsmallar, kontrollera att de är sparade [SRS krav 6.3.8]

\item
Försök spara fler än 20 tidsrapportsmallar, kontrollera att den sista inte sparas [SRS krav 6.3.8]

\item
En tidrapportmall består av 1-2


%PRESTANDA
\item
Svaret på en godtycklig förfrågan från en dator i E-huset kommer i 95\% av fallen tillbaka inom en sekund [SRS~krav~7.2.1]

\item
Genomför 20 stycken inloggningar och få svar under en sekund [SRS krav 7.2.1, 7.2.2]

\end{ST}




\end{appendices}

\newpage

\begin{appendices}

\section{Regressionstest}

Om något ändras inom dessa områden måste följande testfall regressionstestas. Vid varje ändring ska de test rörande de generella kraven också testas.

\subsection{Generella krav}

FT: 1-13, 15, 49, 88\\
ST: 1-7

\subsection{Autentisering}

FT: 14-23, 42, 49, 58, 88\\
ST: 1-7

\subsection{Data}

FT: 24-40, 46, 49, 50-53, 88\\
ST: 1-7

\subsection{Administration}

FT: 1, 3, 4-13, 13-38, 41-74 \\
ST: 1-7

\subsection{Tidsrapportering}

FT: 1-2, 39-53, 68, 75-90\\
ST: 1-7

\subsection{Projektledning}

FT: 91-98\\
ST: 6-7

\end{appendices}


\end{document}