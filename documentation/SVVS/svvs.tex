\documentclass[a4paper]{article}
\usepackage[pdftex]{graphicx}
\usepackage{anysize}
\marginsize{3cm}{3cm}{3cm}{3cm}
\usepackage[utf8]{inputenc}
\usepackage[T1]{fontenc}
\usepackage{enumitem}
\usepackage{titleref}

\usepackage[swedish]{babel}      
\usepackage{epstopdf}     % För svensk avstavning och svenska
\usepackage[osf]{mathpazo} % Palatino with smallcaps and oldstyle numbers
\usepackage[scaled]{helvet} % Helvetica, scaled 95%
\usepackage[titletoc]{appendix}

\usepackage{fancyhdr}

\fancyhf{}
\fancyhead[L]{Ansvarig: TG}
\fancyhead[R]{Datum: \today |Version: 0.14 | Dokumentnummer: PUSS144403}

\newcommand\invisiblesubsubsection[1]{%
  \refstepcounter{subsubsection}%
  \addcontentsline{toc}{subsubsection}{\protect\numberline{\thesubsubsection}#1}%
  \sectionmark{#1}}

\renewcommand{\thesection}{\hspace*{-1.0em}}
\renewcommand{\thesubsection}{\arabic{subsection}}

\newlist{FT}{enumerate}{1}
\setlist[FT]{label=FT \thesubsubsection.\arabic*}

\newlist{ST}{enumerate}{1}
\setlist[ST]{label=ST \thesubsubsection.\arabic*}

\def\reqinside{\hfil\penalty 100 \hfilneg \hbox}
\def \req [#1]{\reqinside{[SRS krav #1]}}

\def\myurl{\hfil\penalty 100 \hfilneg \hbox}

\title{SVVS - Software Verification and Validation Specification: NewPussSystem}                  	
\author{Testgruppen \\ Axel Ulmestig | Axel Goteman | Sefik Ceric \\ Victor Johnsson | Johan Kellerth Fredlund}
\date{}

\begin{document}

\maketitle
\thispagestyle{fancy}
\tableofcontents
\newpage

\section*{Dokumenthistorik}

\begin{tabular}{ l l l p{9cm} }
Ver. & Datum & Ansv. & Beskrivning \\\hline
0.1 & 11 september 2014 & TG & Första utkast \\
0.2 & 14 september 2014 & TG & Anpassning efter SRS v0.10, regressionstest inlagda \\
0.3 & 14 september 2014 & TG & Anpassning efter SRS v0.12, referenser uppdaterade\\
0.4 & 16 september 2014 & TG & Anpassning efter SRS v0.14, tillagda FT och ST samt revidering av vissa test\\
0.5 & 16 september 2014 & TG & Anpassning efter SRS v0.15\\
0.6 & 16 september 2014 & TG & Anpassning efter SRS v0.16\\
0.7 & 16 september 2014 & TG & Små omformuleringar \\
0.8 & 16 september 2014 & TG & Anpassning efter SRS v0.17\\
0.9 & 22 september 2014 & TG & Anpassning efter SRS v0.18 och ändringar efter den formella granskningen \\
v0.10 & 22 september & TG & La till test för två nya krav \\
v0.11 & 23 september & TG & Anpassat efter SRS v0.19 \\
v0.12 & 26 september & TG & Anpassat efter SRS v0.20 och informell granskning, omformuleringar, rättning av stavfel. Ang. informell granskning, "valda tidrapporter" är taget direkt från SRS, datumet för PR är inte utskrivet i SDP\\
v0.13 & 30 september & TG & Tagit bort test som utgått\\
v0.14 & 5 oktober & TG & Anpassad efter SRS v0.21\\

\end{tabular}
\section{Inledning}       

Detta dokument innehåller information om vilka tester och granskningar som ska genomföras under utvecklingen av NewPussSystem.

\section{Referensdokument}
\begin{enumerate}
\item System Requirements Specification v0.21
\item Programvaruutveckling för stora system - Projekthandledning
\item System Requirements Specification: BaseBlockSystem
\end{enumerate}

\section{Definitioner}

\subsection{Dokument}

\begin{enumerate}

\item SDP: Software Development Plan

\item SRS: System Requirements Specification

\item SVVS: Software Verification and Validation Specification

\item STLDD: Software Top Level Design Document

\item SVVR: Software Verification and Validation Report

\item SSD: System Specification Document

\item PFR: Project Final Report


\end{enumerate}

\subsection{White-box test}

White-box testning är när man utgår från kodens interna struktur och försöker testa så att man går igenom alla rader minst en gång eller alla vägval i programflödet.

\subsection{Black-box test}

Black-box testning är där man har ett externt perspektiv d.v.s. testningen är baserad på indata och utdata. Man väljer testfall bland de kombination av indata som kan förekomma. Resultatet av exekveringen jämförs mot en specifikation.


\section{Översikt}

Under projektet ska tre formella granskningar hållas:

\begin{enumerate}


\item Software Specifikation Review (SSR) - 19 september 2014
\begin{itemize}
\item [](a) SDP
\item [](b) SRS
\item [](c) SVVS
\end{itemize}


\item Preliminary Design Review (PDR) - 10 oktober 2014
\begin{itemize}
\item [](a) SVVI
\item [](b) STLDD
\end{itemize}


\item Product Review (PR)
\begin{itemize}
\item [](a) SVVR
\item [](b) SSD
\item [](c) PFR
\end{itemize}


\end{enumerate}

Dessa formella granskningar utförs på det sätt som anges i Ref. 2. kap. 6.2 De formella granskningarna föregås av två informella granskningar av de relevanta partierna. I den första sammanställs information och kritik eller förändringar ges, sen ges tid för förändring innan den andra informella granskningen där den slutliga versionen sammanställs, denna version är sedan den som skickas för formell granskning. Inför varje granskning så ska dessutom utvecklarna kolla igenom SDDD ifall den förändrats.


\section{Testmiljöer}

Den testmiljön som används är på en allmän dator i en godtycklig datorsal i E-huset; 
manuell miljö. För de automatiserade testfallen kommer eclipse och JUnit att användas;
automatisk miljö. Om ingen miljö specificeras i ett testfall är miljön automatisk.

\section{Testfall}

\subsection{Enhetstest}

Utförs av utvecklingsgruppen efter hand som kod tas fram. Här utnyttjas både white-box- och black-box-testning.

\subsection{Funktionstest}

Där enskilda funktioner testas enligt en funktionstestsspecifikation och tillhörande funktionstestinstruktioner. 

\subsection{Systemtest}

Där ett komplett system testas för att se hur flera funktioner och tjänster samverkar enligt en systemtestspecifikation. Testen i enlighet med de framtagna systemtestinstruktionerna. 

\subsection{Regressionstest}

När nya funktioner har införts eller gamla har ändrats behöver ``gamla'' funktioner testas, för att säkerställa att ändringen inte påverkat något som tidigare godkänts i test.

\subsection{Acceptanstest}

Under vilket kunden utför ett urval av ovanstående tester samt ett par för leverantörer okända testfall, med syfte att validera systemet.

\section{Krav som tas upp vid granskningar}


%ändrad 23/9**
Vid granskningar ska alla krav granskas.
Följande krav (i Ref.1) är endast verifierbara genom granskningar: 6.1.5, 6.2.3, 6.2.8, 6.2.15, 6.3.23, 7.1.1, 7.1.2, 7.1.3, 7.3.1, 8.1.1, 8.1.2, 8.1.3 och 8.1.4.

\section{Bilagor}

Det finns tre bilagor:

\begin{itemize}
\item A: Funktiontestspecifikation
\item B: Systemtestspecifikation
\item C: Regressionstest

\end{itemize}

\newpage

\begin{appendices}

\section{Funktionstestsspecifikation}
Efter varje testfall står vilket eller vilka krav som det testar (Ref. 1).

%+++++++++++++++++++++++++++% 
%BÖRJAN på FT:generella krav%
%+++++++++++++++++++++++++++%

\subsection{Generella krav}

\subsubsection{Användare}

\begin{FT}
\item
\emph{Manuell miljö:} Alla typer av inloggade användare har tillgång till menyn på samtliga sidor som visas av systemet \req[6.1.1]

\item
\emph{Manuell miljö:} Menyn ska ge tillgång till de funktionaliteter som en användare besitter \req[6.1.2]

\item 
\emph{Manuell miljö:} Menyns innehåll ska vara samma på varje sida som visas av systemet \req[6.1.3]

\item
Försök ge inkorrekt input till systemet (felaktiga tecken, SQL-injections) \req[6.1.4]

\item
I en projektgupp får det finnas max två stycken projektledare och tre typer av roller: t1, t2, och t3. \req[6.1.6]
\end{FT}

\subsubsection{Projektledare}
\begin{FT}
\item
Varje projekt har minst en och max två användare som besitter rollen som projektledare  \req[6.1.8]
\end{FT}

\subsubsection{Administratör}

\begin{FT}
\item
Försök lägga till administratören i en projektgrupp \req[6.1.11]
\end{FT}

\subsubsection{Data}

\begin{FT}
\item När man tar bort en projektledare bekräftar man bortagningen genom en dialogruta, väljer man ``Ja'' tas projektledaren bort och man dirigeras till en uppdaterad lista av användarna \req[6.1.14]

\item När man tar bort en vanlig användare bekräftar man bortagningen genom en dialogruta, väljer man ``Ja'' tas den vanliga användaren bort och man dirigeras till en uppdaterad lista av användarna \req[6.1.14]

\item När man tar bort en projektledare bekräftar man bortagningen genom en dialogruta, väljer man ``Nej'' tas projektledaren inte bort och man dirigeras till listan av användarna \req[6.1.14]

\item När man tar bort en vanlig användare bekräftar man bortagningen genom en dialogruta, väljer man ``Nej'' tas den vanliga användaren inte bort och man dirigeras till listan av användarna \req[6.1.14]
\end{FT}

%+++++++++++++++++++++++++++% 
%SLUTET på FT:generella krav%
%+++++++++++++++++++++++++++%





%++++++++++++++++++++++++++% 
%BÖRJAN på FT:autentisering%
%++++++++++++++++++++++++++%
%Sektion av Axel G

\subsection{Autentisering}
\subsubsection{Övergripande}

\begin{FT}
\item 
\emph{Manuell miljö:} Logga in med en användare på en terminal och försök logga in på en annan enhet med samma användare, inloggningen kommer då inte genomföras och ett felmeddelande ska visas. \req[6.2.1, 6.2.2]

\item
Systemet håller login status i en server session \req[6.2.4]

%ändrad 23/9
\item
Administratören försöker skapa en användare med användarnamn som är 4 tecken eller mindre långt. Användandes endast tecken definierade i ascii nummer (48-57, 65-90, 97-122)
\req[6.2.5, 6.4.30, 6.1.4]

%ändrad 23/9
\item
Administratören försöker skapa en användare med användarnamn som är 11 tecken eller längre långt.
Användandes endast tecken definierade i ascii nummer (48-57, 65-90, 97-122)
\req[6.2.5, 6.4.30, 6.1.4]

%ändrad 23/9
\item
Administratören försöker skapa en användare med användarnamn innehåller ett eller flera tecken utanför de tecken definierade i ascii nummer (48-57, 65-90, 97-122)
\req[6.2.5, 6.4.30, 6.1.4]

%ändrad 23/9
\item
Administratören försöker skapa en ny användare med ett användarnamn som redan finns registrerat hos systemet.
\req[6.2.6, 6.4.30]

\item
\emph{Manuell miljö:} Försök skapa/byta till ett lösenord som innehåller fler eller färre än 6 tecken. Lösenordet består av tecken definierade i ascii nummer (97-122).
\req[6.2.7, 6.1.4]

\item
\emph{Manuell miljö:} Försök skapa/byta till ett lösenord med tecken som innehåller ett eller flera tecken som är utanför de definierade i ascii nummer (97-122).
\req[6.2.7, 6.1.4]
\end{FT}

%Användare
\subsubsection{Användare}

\begin{FT}
\item
\emph{Manuell miljö:} Gå igenom alla url's en inloggad användare ska ha tillgång till och verifiera att en utloggnings funktionallitet finns synlig och tillgänglig.
\req[6.2.13]

\item
Inaktivitet i 20 minuter gör att inloggnings status förändras till inte inloggad.
\req[6.2.9, 6.2.14]
\end{FT}

%Administratör
\subsubsection{Administratör}

\begin{FT}

%ändrad 23/9
\item
Administratören kan ta bort alla användare förutom sig själv i systemet \req[6.4.29]

%ändrad 23/9
\item
\emph{Manuell miljö:} Administratören försöker ta bort sig själv från systemet \req[6.4.29]

%ändrad 23/9
%\item
%Administratör lyckas ta bort en tidrapportmall som inte används av en projektgrupp \req[6.4.36]
\end{FT}

%Data
\subsubsection{Data}

\begin{FT}
\item
\emph{Manuell miljö:} Ge servern ett användarnamn och lösenord, kontrollera att dessa kontrolleras mot de redan registrerade användarna. Systemet ska sedan dirigera om användaren till en sida med användarfunktioner och sessions statusen ska ändras till inloggad.
\req[6.2.16]
\end{FT}

\subsubsection{Ej inloggad}

\begin{FT}
\item
En icke inloggad användare når systemet och tvingas då lämna inloggningsinformation.
\req[6.2.17]
\item
Användaren kan välja mellan alla befintliga projektgrupper i systemet på inloggningssidan \req[6.2.18]
\item
Användaren skall specificera vilken projektgrupp den vill logga in på \req[6.2.19]
\item
Användare lyckas logga in på de projektgrupp(er) som denne är medlem i \req[6.2.20]
\item
Användare försöker logga in på en projektgrupp som denne inte är medlem i \req[6.2.20]
\item
Administratören lyckas logga in på samtliga projektgrupper \req[6.2.0]
\end{FT}

%++++++++++++++++++++++++++% 
%SLUTET på FT:autentisering%
%++++++++++++++++++++++++++%







%++++++++++++++++++++++++++++% 
%BÖRJAN på FT:tidrapportering%
%++++++++++++++++++++++++++++%
%Johan
\subsection{Tidrapportering}

%tidrapportering
\subsubsection{Projektmedlem}

\begin{FT}

%ändrad 23/9**
\item
Projektmedlem lyckas skapa en egen osignerad tidrapport \req[6.3.1, 6.3.34]
\item
Projektmedlem lyckas uppdatera sin egen osignerade tidrapport \req[6.3.1]
\item
Projektmedlem lyckas ta bort sin egen osignerade tidrapport \req[6.3.1]
\item
\emph{Manuell miljö:} Vid tidrapporteringsfunktionaliten kan en projektmedlem endast se sina egna tidrapporter \req[6.3.2]
\item
Projektmedlem försöker ta bort en av sina signerade tidrapporter \req[6.3.3]
\item
Projektmedlem försöker redigera en signerad tidrapport \req[6.3.4]
\end{FT}

%tidrapportering
\subsubsection{Projektledare}

\begin{FT}

%ändrad 23/9**
\item
\emph{Manuell miljö:} Projektledaren har tillgång till samtliga projektmedlemmars tidrapporter i sin projektgrupp \req[6.3.10]

%ändrad 23/9**
\item
Projektledaren lyckas godkänna en ej tidigare godkänd tidrapport från en medlem i sin projektgrupp \req[6.3.11]

%ändrad 23/9**
\item
Projektledaren lyckas ta tillbaka sitt godkännande från en tidigare godkänd tidrapport \req[6.3.12]

%ändrad 23/9**
\item
Projektledaren lyckas generera statistik i form av tidrapporter per användare för samtliga veckor \req[6.3.13]

%ändrad 23/9**
\item
Projektledaren lyckas generera statistik i form av tidrapporter per roll för samtliga veckor \req[6.3.14]

%ändrad 23/9**
\item
Projektledaren lyckas generera statistik i form av tidrapporter per aktivitet \req[6.3.15]

%ändrad 23/9**
\item
Projektledaren lyckas generera statistik i form av tidrapporter per vecka \req[6.3.16]

%ändrad 23/9**
\item
Projektledaren lyckas generera statistik i form av tidrapporter per användare och aktivitet \req[6.3.17]

%ändrad 23/9**
\item
Projektledaren lyckas generera statistik i form av tidrapporter per användare för utvalda veckor \req[6.3.18]

%ändrad 23/9**
\item
Projektledaren lyckas generera statistik i form av tidrapporter per roll och aktivitet \req[6.3.19]

%ändrad 23/9**
\item
Projektledaren lyckas generera statistik i form av tidrapporter per roll för utvalda veckor \req[6.3.20]

%ändrad 23/9**
\item
Projektledaren lyckas generera statistik i form av tidrapporter per aktivitet och vecka \req[6.3.21]

%ändrad 23/9*
%ändrad 23/9**
\item
\emph{Manuell miljö:} Projektledaren kan se sammalagd arbetstid från valda tidrapporter \req[6.3.13-21, 6.3.6]



%VET INTE VILKA KRAV DESSA TEST SYFTAR PÅ. ÖVERFLÖDIGT?%


% \item
% \emph{Manuell miljö:} Projektledaren kan se sammalagd arbetstid från en vald aktivitet från valda tidrapporter \req[6.3.12-20]

% %ändrad 23/9**
% \item
% \emph{Manuell miljö:} Projektledaren kan se sammalagd arbetstid från en vald subaktivitet från valda tidrapporter \req[6.3.14-21]

% %ändrad 23/9**
% \item
% \emph{Manuell miljö:} Projektledaren kan se hur ofta en vald aktivitet har utförts i valda tidrapporter \req[6.3.14-21]

% %ändrad 23/9**
% \item
% \emph{Manuell miljö:} Projektledaren kan se hur ofta en vald subaktivitet har utförts i valda tidrapporter \req[6.3.14-21]

% %ändrad 23/9**
% \item
% \emph{Manuell miljö:} Projektledaren kan se vilken aktivitet som har tagit mest tid från tidrapportererna \req[6.3.14-21]

% %ändrad 23/9**
% \item
% \emph{Manuell miljö:} Projektledaren kan se vilken aktivitet som är vanligast från tidrapportererna  \req[6.3.14-21]

% %ändrad 23/9**
% \item
% \emph{Manuell miljö:} Projektledaren kan se vilken vecka som har mest rapporterad tid från tidrapportererna \req[6.3.14-21]
\end{FT}


%tidrapportering
\subsubsection{Data}

\begin{FT}

\item
\emph En tidrapport innehåller information om användarnamn \req[6.3.29]


\item
\emph En tidrapport innehåller information om projektgruppsnamn \req[6.3.30]


\item
\emph En tidrapport innehåller information om datum i form av "dag månad år" \req[6.3.31]

\item
\emph En tidrapport innehåller information om veckonummer \req[6.3.32]

%ändrad 23/9**
\item
\emph{Manuell miljö:} Det ska framgå tydligt att en tidrapport är signerad eller inte \req[6.3.33]

%%ändrad 23/9**
%\item
%Generera grundmallen för tidrapporterna automatiskt av systemet \req[6.3.45]

%%ändrad 23/9**
%\item
%Försök skapa två aktiviteter med samma namn \req[6.3.46]


%%ändrad 23/9**
%\item
%Försök skapa två subaktiviteter med samma namn \req[6.3.47]

\end{FT}

%+++++++++++++++++++++++++++%
%SLUT på FT:Tidrapportering+%
%+++++++++++++++++++++++++++%






%+++++++++++++++++++++++++++%
%början på FT:Administration%
%+++++++++++++++++++++++++++%
%------Victor Johnsson------

\subsection{Administration}
\subsubsection{Övergripande}

\begin{FT}
\item Kontrollera att man från huvudsidan kan komma åt administrationssidan \req[6.4.1]

\item
Administratören navigerar till administrationssidan \req[6.4.2]

\item
En vanlig användare försöker navigera till administrationssidan men blir dirigerad till huvudsidan \req[6.4.2]

\item
En projektledare försöker navigera till administrationssidan men blir dirigerad till huvudsidan \req[6.4.2]
\end{FT}

\subsubsection{Projektledare}

\begin{FT}
\item Projektledaren tilldelar en roll till en projektmedlem i sin projektgrupp \req[6.4.3]

\item Projektledaren listar alla ej godkända tidrapporter \req[6.4.4]

\item Projektledaren listar alla godkända tidrapporter \req[6.4.5]

\item Projektledaren listar projektets alla tidrapporter och sorterar dom i stigande ordning efter användare \req[6.4.9]

\item Projektledaren listar projektets alla tidrapporter och sorterar dom i fallande ordning efter användare \req[6.4.9]

\item Projektledaren listar projektets alla tidrapporter och sorterar dom i stigande ordning efter vecka \req[6.4.9]

\item Projektledaren listar projektets alla tidrapporter och sorterar dom i fallande ordning efter vecka \req[6.4.9]

\item Projektledaren listar projektets alla tidrapporter och sorterar dom i stigande ordning efter hurvida rapporten är godkänd eller ej \req[6.4.9]

\item Projektledaren listar projektets alla tidrapporter och sorterar dom i fallande ordning efter hurvida rapporten är godkänd eller ej \req[6.4.9]
\end{FT}

\subsubsection{Administratör}

\begin{FT}
\item Administratören tilldelar en roll till en projektmedlem \req[6.4.3]

\item Administratören listar alla ej godkända tidrapporter \req[6.4.4]

\item Administratören listar alla godkända tidrapporter \req[6.4.5]

\item Administratören listar ett projekts alla tidrapporter och sorterar dom i stigande ordning efter användare \req[6.4.11]

\item Administratören listar ett projekts alla tidrapporter och sorterar dom i fallande ordning efter användare \req[6.4.11]

\item Administratören listar ett projekts alla tidrapporter och sorterar dom i stigande ordning efter vecka \req[6.4.11]

\item Administratören listar ett projekts alla tidrapporter och sorterar dom i fallande ordning efter vecka \req[6.4.11]

\item Administratören listar ett projekts alla tidrapporter och sorterar dom i stigande ordning efter hurvida rapporten är godkänd eller ej \req[6.4.11]

\item Administratören listar ett projekts alla tidrapporter och sorterar dom i fallande ordning efter hurvida rapporten är godkänd eller ej \req[6.4.11]

\item Administratören skapar en projektgrupp \req[6.4.12]

\item En vanlig användare försöker skapa en projektgrupp \req[6.4.12]

\item En projektledare försöker skapa en projektgrupp \req[6.4.12]

\item Administratören lägger till en projektledare i en projektgrupp \req[6.4.13]

\item Administratören tar bort en projektledare i en projektgrupp \req[6.4.13]

\item En vanlig användare försöker lägga till en projektledare i en projektgrupp \req[6.4.13]

\item En vanlig användare försöker ta bort en projektledare i en projektgrupp \req[6.4.13]

\item En projektledare försöker ta bort en projektledare i en projektgrupp \req[6.4.13]

\item Administratören tar bort en projektgrupp \req[6.4.14]

\item En vanlig användare försöker ta bort en projektgrupp \req[6.4.14]

\item En projektledare försöker ta bort en projektgrupp \req[6.4.14]

\item Administratören lyckas ta bort en projektgrupp med medlemmar i \req[6.4.15]

\item På administrationssidan är alla användare listade med både användarnamn och lösenord \req[6.4.22]

\item På administrationssidan kan man ta bort en vanlig användare \req[6.4.23]

\item På administrationssidan kan man ta bort en projektledare \req[6.4.23]

\item På administrationssidan kan man inte ta bort en administratör \req[6.4.23]


\item På administrationssidan kan man lägga till en ny användare \req[6.4.25]

\item Administratören skapar en användare och skriver in användarnamn, användarens lösenord genereras slumpmässigt \req[6.4.26]

\item Administratören försöker skapa en användare med ett upptaget användarnamn, ett felmeddelande visas \req[6.4.27]

\item Administratören försöker skapa en användare med ett ogiltigt användarnamn (giltigt användarnamn: 5-10 tecken, ascii 48-57, 65-90 och 97-122) \req[6.4.28, 6.1.4]

\item Administratören skapar en användare med ett giltigt användarnamn (giltigt användarnamn: 5-10 tecken, ascii 48-57, 65-90 och 97-122) \req[6.4.28, 6.1.4]

%\item Administratören skapar en tidrapportmall \req[6.4.28]

%\item Administratören försöker ändra en tidrapportmall som en projektgrupp använder \req[6.4.28]
\end{FT}

\subsubsection{Data}

\begin{FT}

%ändrad 23/9
\item Administratören försöker skapa en projektgrupp med ett ogiltigt projektgruppsnamn (giltigt projektgruppsnamn: 5-10 tecken, ascii 48-57 och 97-122) \req[6.4.33, 6.1.4]

%ändrad 23/9
\item Administratören försöker skapa en projektgrupp med ett giltigt projektgruppsnamn (giltigt projektgruppsnamn: 5-10 tecken, ascii 48-57 och 97-122) \req[6.4.33, 6.1.4]

%ändrad 23/9
\item Administratören försöker skapa en projektgrupp med ett upptaget projektgruppsnamn \req[6.4.34]

%ändrad 23/9
\item Administratören skapar 5 stycken projektgrupper \req[6.4.35]

%ändrad 23/9
\item Administratören försöker skapa 6 stycken projektgrupper \req[6.4.35]

%ändrad 23/9
\item Administratören lägger till samma användare i flera projektgrupper \req[6.4.36]

%ändrad 23/9
\item Administratören försöker skapa en projektgrupp utan användare \req[6.4.37]

%ändrad 23/9
\item Administratören försöker att, från en projektgrupp som har en användare, ta bort en användare \req[6.4.37]

%ändrad 23/9
\item Administratören skapar en projektgrupp med en användare \req[6.4.37]

%ändrad 23/9
\item Administratören lägger till användare i en projektgrupp så att projektgruppen har 20 användare \req[6.4.37]

%ändrad 23/9
\item Administratören försöker lägga till användare i en projektgrupp så att projektgruppen har 21 användare \req[6.4.37]

%ändrad 23/9
\item Projektledaren tilldelar tre olika roller till projektmedlemmar i sitt projekt \req[6.4.38]

%ändrad 23/9
\item Projektledaren försöker tilldela fyra olika roller till projektmedlemmar i sitt projekt \req[6.4.38]

%ändrad 23/9
\item Projektledaren tilldelar 6 projektmedlemmar i sitt projekt rollen t1 \req[6.4.39]

%ändrad 23/9
\item Projektledaren tilldelar 6 projektmedlemmar i sitt projekt rollen t2 \req[6.4.39]

%ändrad 23/9
\item Projektledaren tilldelar 6 projektmedlemmar i sitt projekt rollen t3 \req[6.4.39]

%ändrad 23/9
\item Projektledaren försöker tilldela 7 projektmedlemmar i sitt projekt rollen t1 \req[6.4.39]

%ändrad 23/9
\item Projektledaren försöker tilldela 7 projektmedlemmar i sitt projekt rollen t2 \req[6.4.39]

%ändrad 23/9
\item Projektledaren försöker tilldela 7 projektmedlemmar i sitt projekt rollen t3 \req[6.4.39]

%ändrad 23/9
\item Administratören har användarnamnet ``admin'' och lösenordet ``adminpw'' \req[6.4.40]
\end{FT}

%+++++++++++++++++++++++++++% 
%slutet på FT:Administration%
%+++++++++++++++++++++++++++%

\end{appendices}


%!!!!!!!!!!!!!
%!!!!!!!!!!!!!
%!!!!!!!!!!!!!
%HÄR BÖRJAR ST
%!!!!!!!!!!!!!
%!!!!!!!!!!!!!
%!!!!!!!!!!!!!

\newpage

\begin{appendices}

\section{Systemtestspecifikation}

Efter varje testfall står vilket eller vilka krav som det testar (Ref. 1).

%+++++++++++++++++++++++++++%
%BÖRJAN på ST:generella krav%
%+++++++++++++++++++++++++++%
\subsection{Generella krav}

\invisiblesubsubsection{Användare}

\subsubsection{Projektmedlem}

\begin{ST}
\item
Genomför flödeschemat i figur 2 i SRS \req[6.1.7]
\end{ST}

\subsubsection{Projektledare}

\begin{ST}

\item
Flödesschemat i figur 3 i SRS (ref 1) stödjs av systemet \req[6.1.10, 6.1.9]

\end{ST}

\subsubsection{Administratör}

\begin{ST}

\item
Alla steg i flödeschemat i figur 4 i SRS stödjs \req[6.1.13, 6.4.21]

\end{ST}

\invisiblesubsubsection{Data}

%+++++++++++++++++++++++++%
%SLUT på ST:generella krav%
%+++++++++++++++++++++++++%


%++++++++++++++++++++++++++%
%BÖRJAN på ST:autentisering%
%++++++++++++++++++++++++++%
\subsection{Autentisering}

\subsubsection{Användare}

\begin{ST}
\item
\emph{Manuell miljö:} Genomför och verifiera utkomst av scenario 6.1.1 ifrån referens 3 (SRS för grundsystem).
\req[6.2.10]
\item
\emph{Manuell miljö:} Genomför och verifiera utkomst av scenario 6.1.2 ifrån referens 3 (SRS för grundsystem).
\req[6.2.11]
\item
\emph{Manuell miljö:} Genomför och verifiera utkomst av scenario 6.1.3 ifån referens 3 (SRS för grundsystem).
\req[6.2.12]
\end{ST}

\subsubsection{Administratör}

\begin{ST}
%ändrad 23/9
%ändrad 23/9**
\item
Genomför scenario 6.4.5 (Ref 1: Skapa projektgrupp) \req[6.4.20]

%ändrad 23/9
\item
Genomför scenario 6.4.6 (Ref 1: Lägga till/ändra roll på projektmedlem i projektgrupp) \req[6.4.31]

%ändrad 23/9
\item
Genomför scenario 6.4.7 (Ref 1: Ta bort projektmedlemmar eller projektgrupp) \req[6.4.32]


%\item
%Genomför scenario 6.4.8 (Ref 1: Skapa tidrapportmall) \req[6.4.3, 6.3.27, 6.3.40]

%ändrad 23/9
%\item
%Genomför scenario 6.4.9 (Ref 1: Ta bort tidrapportmall) \req[6.4.35, 6.3.41]
\end{ST}

%++++++++++++++++++++++++%
%SLUT på ST:autentisering%
%++++++++++++++++++++++++%



%++++++++++++++++++++++++++++%
%BÖRJAN på ST:tidrapportering%
%++++++++++++++++++++++++++++%
%Johan
\subsection{Tidrapportering}
\subsubsection{Projektmedlem}

\begin{ST}

%ändrad 23/9**
\item
Genomför scenario 6.3.1 (Dokumentera arbetstimmar i systemet) \req[6.3.7, 6.3.5]

%ändrad 23/9**
\item
Genomför scenario 6.3.2 (Ta bort/redigera arbetstimmar i systemet) \req[6.3.8]

%ändrad 23/9**
\item
Systemet stödjer stegen i figur 6 \req[6.3.9]
\end{ST}

\subsubsection{Projektledare}

\begin{ST}

%ändrad 23/9**
\item
Systemet stödjer stegen i figur 7 \req[6.3.22]
\end{ST}
%++++++++++++++++++++++++++%
%SLUT på ST:tidrapportering%
%++++++++++++++++++++++++++%




%+++++++++++++++++++++++++++%
%BÖRJAN på ST:Administration%
%+++++++++++++++++++++++++++%
%------Victor Johnsson------

\subsection{Administration}
\invisiblesubsubsection{Övergripande}

\subsubsection{Projektledare}

\begin{ST}
\item Genomför scenario 6.4.1 (Ref 1: Generera statistik från systemet) \req[6.4.6]

\item Genomför scenario 6.4.2 (Ref 1: Godkänna en tidrapport) \req[6.4.7]

\item Genomför scenario 6.4.3 (Ref 1: Ta tillbaka godkännande av en tidrapport) \req[6.4.8]

\item Genomför scenario 6.4.4 (Ref 1: Hitta en rapport genom att sortera dem) \req[6.4.9]
\end{ST}

\subsubsection{Administratör}

\begin{ST}
\item Genomför scenario 6.4.1 som administratör istället för projektledare (Ref 1: Generera rapporter från systemet) \req[6.4.16]

\item Genomför scenario 6.4.2 som administratör istället för projektledare (Ref 1: Godkänna en tidrapport) \req[6.4.17]

\item Genomför scenario 6.4.3 som administratör istället för projektledare (Ref 1: a tillbaka godkännande av en tidrapport) \req[6.4.18]

\item Genomför scenario 6.4.4 som administratör istället för projektledare (Ref 1: itta en rapport genom att sortera dem) \req[6.4.19]

\item \emph{Utgår:} Genomför sekvenserna \req[6.4.21]
\end{ST}

\invisiblesubsubsection{Data}

%+++++++++++++++++++++++++++%
%SLUTET på ST:Administration%
%+++++++++++++++++++++++++++%




\subsection{Kvalitetskrav}

\subsubsection{Prestanda}

% \item 
% Testa input med olika tecken från användaren \req[6.1.1]

\begin{ST}
\item
Försök logga in med fler än 50 användare samtidigt \req[7.2.3]

\item
Logga in med 50 användare samtidigt \req[7.2.3]

\item
Svaret på en godtycklig förfrågan från en dator i E-huset kommer i 95\% av fallen tillbaka inom en sekund \req[7.2.1, 7.2.2]
\end{ST}

% prestanda slut


\invisiblesubsubsection{Användarvänligt}


\end{appendices}

\newpage

\begin{appendices}

\section{Regressionstestspecifikation}

När något ändras ska helst alla tester köras igen. Om så inte är möjligt ska åtminstone de generella kraven och de tester i det område som förändringen påverkade köras.

Om något ändras inom dessa områden måste följande testfall regressionstestas. Vid varje ändring ska de test rörande de generella kraven också testas.

Områden innefattar:

\begin{itemize}
\item
Generella krav
\item
Autentisering
\item
Tidrapportering
\item
Administration
\end{itemize}

\end{appendices}
\end{document}