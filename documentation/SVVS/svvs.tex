\documentclass[a4paper]{article}
\usepackage[pdftex]{graphicx}
\usepackage{anysize}
\marginsize{3cm}{3cm}{3cm}{3cm}
\usepackage[utf8]{inputenc}
\usepackage[T1]{fontenc}
\usepackage{enumitem}

\usepackage[swedish]{babel}      
\usepackage{epstopdf}     % För svensk avstavning och svenska
\usepackage[osf]{mathpazo} % Palatino with smallcaps and oldstyle numbers
\usepackage[scaled]{helvet} % Helvetica, scaled 95%
\usepackage[titletoc]{appendix}

\usepackage{fancyhdr}

\fancyhf{}
\fancyhead[L]{Ansvarig: TG}
\fancyhead[C]{Datum: 14 september 2014}
\fancyhead[R]{Version: 0.3}


\newlist{FT}{enumerate}{1}
\setlist[FT]{label=FT\arabic*}

\newlist{ST}{enumerate}{1}
\setlist[ST]{label=ST\arabic*}

%\newcommand{\req}[1]{\[SRS~req.~#1\]}

\title{SVVS - Software Verification and Validation Specification: NewPussSystem}                  	
\author{Testgruppen \\ Axel Ulmestig | Axel Goteman | Sefik Ceric \\ Victor Johnsson | Johan Kellerth Fredlund}
\date{}

\begin{document}

\maketitle
\thispagestyle{fancy}
\tableofcontents
\newpage

\section*{Dokumenthistorik}

\begin{tabular}{ l l l l }
Ver. & Datum & Ansv. & Beskrivning \\\hline
0.1 & 11 september 2014 & TG & Första utkast \\
0.2 & 14 september 2014 & TG & Anpassning efter SRS v0.10, regressionstest inlagda \\
0.3 & 14 september 2014 & TG & Anpassning efter SRS v0.12, referenser uppdaterade \\

\end{tabular}
\section{Inledning}       

Detta dokument innehåller information om vilka tester och granskningar som ska genomföras under utvecklingen av NewPussSystem.

\section{Referensdokument}
\begin{enumerate}
\item System Requirements Specification v0.12
\item Projekthandledning
\end{enumerate}

\section{Översikt}

Under projektet ska tre formella granskningar hållas

\begin{enumerate}


\item Software Specifikation Review (SSR) - 19 september 2014
\begin{itemize}
\item [](a) SDP
\item [](b) SRS
\item [](c) SVVS
\end{itemize}


\item Preliminary Design Review (PDR)
\begin{itemize}
\item [](a) SVVI
\item [](b) STLDD
\end{itemize}


\item Product Review (PR)
\begin{itemize}
\item [](a) SVVR
\item [](b) SSD
\item [](c) PFR
\end{itemize}


\end{enumerate}

Dessa formella granskningar utförs på det sätt som anges i Ref. 2. De formella granskningarna föregås av två informella granskningar av de relevanta partierna. I den första sammanställs information och kritik eller förändringar ges, sen ges tid för förändring innan den andra informella granskningen där den slutliga versionen sammanställs, denna version är sedan den som skickas för formell granskning. Inför varje granskning så ska dessutom utvecklarna kolla igenom SDDD ifall den förändrats.


\section{Testmiljöer}

Den testmiljön som används är på en godtycklig dator i E-huset. För de automatiserade testfallen kommer eclipse och JUnit att användas.

\section{Testfall}

Tester utförs på det sätt som anges nedan.

\subsection{Enhetstest}

Utförs av utvecklingsgruppen efter hand som kod tas fram. Här utnyttjas både white-box- och black-box-testning.

\subsection{Funktionstest}

Där enskilda funktioner testas enligt en funktionstestspecifikation och tillhörande funktionstestinstruktioner. 

\subsection{Systemtest}

Där ett komplett system testas för att se hur flera funktioner och tjänster samverkar enligt en systemtestspecifikation. Testen i enlighet med de framtagna systemtestinstruktionerna. 

\subsection{Regressionstest}

När nya funktioner har införts eller gamla har ändrats behöver ``gamla'' funktioner testas, för att säkerställa att ändringen inte påverkat något som tidigare godkänts i test.

\subsection{Acceptanstest}

Under vilket kunde utför ett urval av ovanstående tester samt ett par för leverantörer okända testfall, med syfte att validera systemet.

\section{Krav som tas upp vid granskningar}

Vid granskningar ska alla krav granskas.
Följande krav (enligt Ref.1) är endast verifierbara genom granskningar: 7.1.1, 7.1.2, 8.1.1, 8.1.2, 8.1.3 och 8.1.4.

\section{Bilagor}

Det finns tre bilagor:

\begin{itemize}
\item A: Funktiontestspecifikation
\item B: Systemtestspecifikation
\item C: Regressionstest

\end{itemize}

\newpage

\begin{appendices}

\section{Funktionstestsspecifikation}


\begin{FT}


\item
Admin försöker lägga till en användare i en ny projektgrupp då denne är medlem i en annan projektgrupp [SRS req. 6.1.1]

\item
Den funktionalitet som användaren har rätt till finns tillgänglig i menyn [SRS req 6.1.4]

\item
Menyn finns tillgäng på samtliga sidor i systemet [SRS req 6.1.8]

\item
Admin försöker lägga in en ny användare då det finns 20 användare i en projektgrupp [SRS req. 6.1.9]

\item
Admin försöker ta bort en användare då det finns 1 användare i en projektgrupp [SRS req 6.1.9]

\item
Admin ger en användare rollen projektledare [SRS req. 6.1.11]

\item
Admin ger en användare rollen t1 [SRS req. 6.1.11]

\item
Admin ger en användare rollen t2 [SRS req. 6.1.11]

\item
Admin ger en användare rollen t3 [SRS req. 6.1.11]

\item
Admin försöker ge en användare rollen projektledare då det redan finns 2 av dessa i projektgruppen [SRS req. 6.1.12]

\item
Admin försöker ge en användare rollen t1 då det redan finns 6 av dessa i projektgruppen [SRS req. 6.1.13]

\item
Admin försöker ge en användare rollen t2 då det redan finns 6 av dessa i projektgruppen [SRS req. 6.1.13]

\item
Admin försöker ge en användare rollen t3 då det redan finns 6 av dessa i projektgruppen [SRS req. 6.1.13]

\item
En inloggad användare försöker använda en URL till en funktionalitet där denne har behörighet [SRS req. 6.2.1]

\item
Då servern startas om så blir alla användare utloggade [SRS req 6.2.2]

\item
En användare loggar in med rätt användarnamn och lösenord [SRS req. 6.2.4, 6.2.9]

\item
En användare loggar in med fel användarnamn och rätt lösenord [SRS req. 6.2.4, 6.2.9]

\item
En användare loggar in med rätt användarnamn och fel lösenord [SRS req. 6.2.4, 6.2.9]

\item
En användare loggar in med fel användarnamn och fel lösenord [SRS req. 6.2.4, 6.2.9]

\item 
Användare blir utloggad efter 20 minuters inaktivitet [SRS req. 6.2.6]

\item 
Logga in när annan användare är inloggad [SRS req. 6.2.7, 6.2.8]

\item
En inloggad användare loggar ut [SRS req. 6.2.10]

\item
En icke inloggad användare försöker logga ut [SRS req. 6.2.10]

%DATA Victor Johnsson
\item
Tidsrapportering sker i antal minuter [SRS req. 6.3.4, 6.5.5]

\item
Tidrapportens utseende, med avseende på vilka aktiviteter man kan föra in tidinformation i, bestäms av administratören innan projektgruppen har skapats [SRS req. 6.3.5, 6.4.16]

\item
Försök att med ett skapat projekt ändra tidrapportens utseende [SRS req. 6.3.5]

\item
Tidrapportens grundmall skall innehålla information om användarnamn, projektgruppsnamn, datum, veckonummer och om den är signerad eller inte [SRS req. 6.3.6]

\item
När en tidsrapportsmall skapas, sparas den på servern [SRS req. 6.3.7]

\item
Spara 20 tidsrapportsmallar [SRS req. 6.3.8]

\item
Spara mer än 20 tidsrapportsmallar, de skrivs över enligt FIFO-principen [SRS req. 6.3.8]

\item
En tidrapportmall består av 1-10 rader samt 1-5 kolumner som det går att mata in tidsinformation i [SRS req. 6.3.9]

\item
Namnet på tidrapportmallen kan endast bestå av 5 tecken, ascii (decimal) värden 48-57 och
97-122 är tillåtna [SRS req. 6.3.10, 6.4.17]

\item
Namnet på aktivitet och subaktivitet kan endast bestå av 3-5 tecken respektive 1 tecken, ascii (decimal) värden 48-57 och 97-122 är tillåtna [SRS req. 6.3.11]

\item
Tidinformationen får endast innehålla 1-5 tecken per aktivitetsruta, ascii (decimal) värden 48-57 [SRS req. 6.3.12, 6.5.5]

\item
Tidrapportmallarnas namn måste vara unika [SRS req. 6.3.13, 6.4.17]

\item
Ett projektgruppsnamn kan endast bestå av 5-10 tecken, ascii (decimal) värden 48-57 och
97-122 är tillåtna [SRS req. 6.3.14]

\item
Projektgruppsnamn måste vara unika [SRS req. 6.3.15]

\item
Veckonumret kan endast innehålla 1-2 tecken, ascii (decimal) värden 48-57 [SRS req. 6.3.16]

\item
Lagra 100 tidsrapporter för en användare [SRS req. 6.3.17]

\item
Lagra mer än 100 tidsrapporter för en användare, de skrivs över enligt FIFO-principen [SRS req. 6.3.17]

\item
Admin lägger till och/eller tar bort användare och förs sedan tillbaks till listan med användare [SRS req 6.4.1]

\item
Logga in på systemet med user 'admin' och pass 'adminpw' [SRS req. 6.4.2]

\item 
Administrationsvyn finns tillgänglig för administratören [SRS req 6.4.3, SRS req 6.4.6]

\item
Administrationsvyn finns inte tillgänglig för användare som inte är administratörer [SRS req 6.4.6]

\item
Skapa ett nytt konto från adminkontot med unikt namn [SRS req. 6.4.5]

\item
Admin försöker att lägga till användare med ett upptaget användarnamn [SRS req. 6.3.1, 6.3.2, 6.4.4, 6.4.5]

\item
Admin tar bort konton som inte är av admin typ [SRS req. 6.4.8]

\item
Admin får en lista med användare [SRS req. 6.4.7]

\item
Admin får en bekräftelseruta då hen ska ta bort till en användare i systemet [SRS req. 6.1.10, 6.4.9]

\item
Admin lägger till en ny användare med korrekt data för användarnamn och lösenord [SRS req. 6.3.1, 6.3.3, 6.4.10]

\item
Admin fösöker lägger till en ny användare med korrekt data för användarnamn men fel lösenordsdata [SRS req. 6.3.1, 6.3.3, 6.4.10, 6.4.11]

\item
Admin försöker lägger till en ny användare med fel data för användarnamn och korrekt lösenordsdata [SRS req. 6.3.1, 6.3.3, 6.4.10, 6.4.11]

\item
Admin försöker lägger till en ny användare med fel data för användarnamn och lösenord [SRS req. 6.3.1, 6.3.3, 6.4.10, 6.4.11]

\item 
Admin skapar en projektgrupp med gruppnamn och projektledare [SRS req. 6.4.12, 6.4.13, 6.4.17]

\item 
Admin försöker skapa en projektgrupp med gruppnamn men utan projektledare [SRS req. 6.4.12, 6.4.13]

\item 
Admin försöker skapa en projektgrupp utan gruppnamn fast med projektledare [SRS req. 6.4.12, 6.4.13]

\item 
Admin försöker skapa en projektgrupp utan gruppnamn och projektledare [SRS req. 6.4.12, 6.4.13]

\item
En användare försöker skapa en projektgrupp [SRS req. 6.4.12]

\item 
Admin försöker skapa en projektgrupp med ett befintligt namn [SRS req. 6.4.14]

\item
Admin tar bort en projektgrupp [SRS req. 6.4.15, 6.4.16]

\item
En användare försöker ta bort en projektgrupp [SRS req. 6.4.15]

\item 
Admin kommer åt lista på alla projektgrupper [SRS req. 6.4.16]

\item
Admin tar bort en användare från en grupp [SRS req. 6.4.17]

\item
Admin lägger till en användare i en grupp [SRS req. 6.4.17]

\item
Den första användaren som läggs till i en projektgrupp blir projektledare [SRS req. 6.4.17]

\item
Administratören gör en medlem i ett projekt till projektledare [SRS req. 6.4.17]

\item
Admin försöker skapa en projektgrupp med ett upptaget namn [SRS req. 6.4.17]

\item 
Admin försöker skapa lägga till fler medlemmar i en redan full projektgrupp[SRS req. 6.4.17]

\item
Admin kan se användare sorterade efter projektgrupp om det finns medlemmar och grupper[SRS req. 6.4.17]

\item
Admin flyttar användare mellan projektgrupper och roller inom projekt [SRS req. 6.4.17]

\item 
Admin försöker flytta den sista projektledaren i en grupp  till en annan grupp [SRS req. 6.4.17]

\item
Admin försöker radera den sista projektledare i en grupp[SRS req 6.4.17]

\item 
Admin skapar en tidrapportmall [SRS req. 6.4.17]

\item
Admin ändrar en gruppmedlems roll [SRS req. 6.4.18]

%TIDSRAPPORTERING Sefik Ceric, Victor Johnsson
\item
Generera en ny tidrappport och kontrollera att den är osignerad[SRS~req.~6.5.1]

\item
Användare kan skapa egna osignerade tidsrapporter [SRS~req.~6.5.2]

\item
Användare kan updatera egna osignerade tidrapporter [SRS~req.~6.5.2]

\item
Användare kan ta bort egna osignerade tidrapporter [SRS~req.~6.5.2]

\item
Användare kan inte ta bort egna signerade tidrapporter [SRS~req.~6.5.3]

\item
Användare kan inte redigera egna signerade tidrapporter [SRS~req.~6.5.3]

\item
Användare som är inte projektledare kan endast se sina egna tidrapporter [SRS~req.~6.5.4]

\item 
En användare kan välja tidrapportering i menyn [SRS~req.~6.5.6]

\item
När användaren är inne på tidsrapporteringssidan, om hen har gjort en rapportering innan, ser hen vilken vecka denna gjordes. [SRS~req.~6.5.6]

\item
När användaren är inne på tidsrapporteringssidan, om hen inte har gjort en rapportering innan så framgår det. [SRS~req.~6.5.6]

\item
En användare, i tidsrapporteringssidan, fyller i ett ej tidigare använt veckonummer och trycker OK, var på en ny tidsrapport genereras med ifyllt veckonummer [SRS~req.~6.5.6]

\item
En användare, i tidsrapporteringssidan, fyller i ett tidigare använt veckonummer och trycker OK, var på användaren kan redigera tidsrapporten med det veckonummret [SRS~req.~6.5.6]

\item 
När en användare fyller i tidsinformation i en tidsrapport, visas den beräknade totaltiden i realtid [SRS~req.~6.5.6]

\item
När en användare har fyllt i tidsinformation i en tidsrapport och trycker på ``Skicka'', om rapporten sparas i databasen visas en bekräftelse [SRS~req. 6.1.10, 6.5.6]

\item
När en användare har fyllt i tidsinformation i en tidsrapport och trycker på ``Skicka'', om rapporten inte sparas i databasen visas ett felmeddelande [SRS~req.~6.5.6]

\item
En användare kan ta bort och redigera tidsrapporter [SRS req. 6.5.6]


%PROJEKTLEDNING Victor Johnsson
\item
Ett projekt kan inte ha mer än två eller mindre än en projektledare [SRS~req.~6.6.1]

\item
Projektledare har tillgång till samtliga projektmedlemmars tidrapporter i sin projektgrupp [SRS~req.~6.6.2]

\item
Projektledaren kan godkänna ej tidigare godkända tidrapporter från medlemmar i sin projektgrupp [SRS~req.~6.6.3]

\item
Projektledaren tar tillbaka sitt godkännande från en tidigare godkänd gruppmedlems tidsrapport, i sin projektgrupp [SRS~req.~6.6.4]

\item
Projektledaren kan, inom sitt projekt, summera alla tidsrapporter [SRS~req.~6.6.5]

\item
Projektledaren kan, inom sitt projekt, summera tidsrapporter per användare/roll/aktivitet/vecka [SRS~req.~6.6.5]

\item
Projektledaren kan, inom sitt projekt, ändra roller hos användare [SRS~req.~6.6.6]

\item
Projektledaren kan, inom sitt projekt, först klicka på statistik i en meny, sen välja från en lista vilken typ av rapport som ska genereras vart på rapporten genereras [SRS~req.~6.6.7]

\end{FT}
\end{appendices}



\newpage

\begin{appendices}

\section{Systemtestspecifikation}

\begin{ST}


% \item 
% Testa input med olika tecken från användaren [SRS req. 6.1.1]

\item
Försök logga in med fler än 50 användare samtidigt [SRS req 6.1.2]

\item 
Användare skriver in URL för hubudmenyn. Har tillgång till följande funktionaliteter; start, tidrapporte-
ring, projektinformation, byta lösenord, hjälp samt utloggning. [SRS req. 6.1.3, 6.1.4, 6.1.8, 6.4.1]


\item 
Projektledare skriver in URL för huvudmenyn. Har tillgång till projektadministrationsfunktionaliter i menyn [SRS req 6.1.5, 6.1.8]

\item 
Admin skriver in URL för hubudmenyn. Har tillgång till administrationsfunktionaliteter och projektadministrationsfunktionaliter [SRS req 6.1.6, 6.1.7, 6.1.8]

\item
Admin försöker lägga till sig själv i en projektgrupp [SRS req. 6.1.14]

\item
En icke inloggad användare försöker använda en URL till en funktionalitet där denne har behörighet [SRS req. 6.2.1, 6.2.3, 6.2.5]

%PRESTANDA
\item
Svaret på en godtycklig förfrågan från en dator i E-huset kommer i 95\% av fallen tillbaka inom en sekund [SRS~req.~7.2.1]

\end{ST}




\end{appendices}

\newpage

\begin{appendices}

\section{Regressionstest}

Om något ändras inom dessa områden måste följande testfall regressionstestas. Vid varje ändring ska de test rörande de generella kraven också testas.

\subsection{Generella krav}

FT: 1-13, 15, 49, 88\\
ST: 1-7

\subsection{Autentisering}

FT: 14-23, 42, 49, 58, 88\\
ST: 1-7

\subsection{Data}

FT: 24-40, 46, 49, 50-53, 88\\
ST: 1-7

\subsection{Administration}

FT: 1, 3, 4-13, 13-38, 41-74 \\
ST: 1-7

\subsection{Tidsrapportering}

FT: 1-2, 39-53, 68, 75-90\\
ST: 1-7

\subsection{Projektledning}

FT: 91-98\\
ST: 6-7

\end{appendices}


\end{document}