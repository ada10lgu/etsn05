\documentclass[a4paper]{article}
\usepackage[pdftex]{graphicx}
\usepackage{anysize}
\marginsize{3cm}{3cm}{3cm}{3cm}
\usepackage[utf8]{inputenc}
\usepackage[T1]{fontenc}

\usepackage[swedish]{babel}      
\usepackage{epstopdf}     % För svensk avstavning och svenska
\usepackage[osf]{mathpazo} % Palatino with smallcaps and oldstyle numbers
\usepackage[scaled]{helvet} % Helvetica, scaled 95%
\usepackage[titletoc]{appendix}

\usepackage{fancyhdr}

\fancyhf{}
\fancyhead[L]{Ansvarig: SG}
\fancyhead[C]{Datum: \today}
\fancyhead[R]{Inofficiell granskning}


\title{Granskningsprotokoll av:\\ SVVS - Software Verification and Validation Specification: NewPussSystem}                  	
\author{Systemarkitektgruppen \\ Lars Gustafsson | Martin Lichota | Marcel Tovar Rascon}
\date{}

\begin{document}

\maketitle
\thispagestyle{fancy}

\section{Inledning}       
Här följer granskning av SVVS - Software Verification and Validation Specification version 0.1, 11 september 2014. Utförd av Systemarkitektgruppen, datum som ovan. Tänkt som den första inofficiella granskningen i fas 1.

\section{Övergripande kritik}
Smärre stavfel här och där. Bra upplägg, överskådbar, och bra svenska.
Författarna är osäkra på vem som är tänkt läsare och var man då skall lägga språket. Behövs till exempel begreppet White-box-testing förklaras? Med en publik av studerande av kursen ETS032 är det troligtvis inte nödvändigt.

Testerna som är specifierade under Appendix A är för generella och författarna ser snarare att de skall testa en sak i taget. Exempel ges under nästa kapitel.

\section{Kapitelspecifik kritik}
\begin{itemize}
\item[2] Ordet projekthandledning används ingen annan stans, bör ersättas med Projektplan.\\
Förutom det saknas både de välkända förkortningarna för dokumenten samt vilken version av dokumenten som nytjas. Vet man detta är det mycket lättare att granska dokumentet.

\item[3] Första meningen bör avslutas med ett komma. \\
Vad alla olika förkortningar (SVVI, SSD...) står för framgår inte. Kan tänkas vara självklart om man antar att läsaren är studerande av krusen ETS032. \\
SDDD ska inte skapas förrän fas 3 vilket medför det svårt för UG att hålla koll på den i de två första faserna.\\
Referensen till Projektplanen är bra men hade underlättat om man viste vilket kapitel.

\item[4]
Kan tänkas vara övertolkning men denna meningen innefattar även författarnas privata datorer så fort de vistas i huset. Kanske bör omforumleras till något i stil med, ``Någon av datorerna i E-husets allmänna 
datorsalar.''

\item[5] Kan tänkas vara redundant information.

\item[5.1] ``white-box'', felstavning.

\item[5.4] ``gamla funktioner'', mellanslag mellan orden.

\item[5.5] Osäker på tredje ordet, förutsätter att det borde stå ``kunden''.

\item[6] Förstår inte referensen till Projektplanen.

\item[Apendix A] Refereringen till kraven sköts snyggt och tydligt. Däremot, eftersom det inte framgår vilken SRS som används, görs det svårt att kolla så alla krav täcks.\\
\\
Flertalet krav är på tok för generella och kan skapa fall där vi missar att testa viss funktionallitet.

\begin{itemize}
\item[FT32] Bör förslagsvis delas upp till ett test för var sak som man inte bör komma åt.
\item[FT26] Liknar FT32 med samma problem.
\item[FT34] Missar att testa att man inte kan godkänna en redan godkänd rapport. Var noga med att se över alla infallsvinklar.
\end{itemize}

\end{itemize}






\end{document}