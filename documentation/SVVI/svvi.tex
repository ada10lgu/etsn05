\documentclass[a4paper]{article}
\usepackage[pdftex]{graphicx}
\usepackage{anysize}
\marginsize{3cm}{3cm}{3cm}{3cm}
\usepackage[utf8]{inputenc}
\usepackage[T1]{fontenc}
\usepackage{enumitem}
\usepackage{titleref}

\usepackage[swedish]{babel}      
\usepackage{epstopdf}     % För svensk avstavning och svenska
\usepackage[osf]{mathpazo} % Palatino with smallcaps and oldstyle numbers
\usepackage[scaled]{helvet} % Helvetica, scaled 95%
\usepackage[titletoc]{appendix}

\usepackage{fancyhdr}

\fancyhf{}
\fancyhead[L]{Ansvarig: TG}
\fancyhead[R]{Datum: \today |Version: 0.8 | Dokumentnummer: PUSS144403}

\newcommand\invisiblesubsubsection[1]{%
  \refstepcounter{subsubsection}%
  \addcontentsline{toc}{subsubsection}{\protect\numberline{\thesubsubsection}#1}%
  \sectionmark{#1}}

\renewcommand{\thesection}{\hspace*{-1.0em}}
\renewcommand{\thesubsection}{\arabic{subsection}}

\newlist{FT}{enumerate}{1}
\setlist[FT]{label=FT \thesubsubsection.\arabic*}

\newlist{ST}{enumerate}{1}
\setlist[ST]{label=ST \thesubsubsection.\arabic*}

\def\reqinside{\hfil\penalty 100 \hfilneg \hbox}
\def \req [#1]{\reqinside{[SRS krav #1]}}

\def\myurl{\hfil\penalty 100 \hfilneg \hbox}

\title{SVVI - Software Verification and Validation Instructions: NewPussSystem}                  	
\author{Testgruppen \\ Axel Ulmestig | Axel Goteman | Sefik Ceric \\ Victor Johnsson | Johan Kellerth Fredlund}
\date{}

\begin{document}

\maketitle
\thispagestyle{fancy}
\tableofcontents
\newpage

\section*{Dokumenthistorik}

\begin{tabular}{ l l l p{9cm} }
Ver. & Datum & Ansv. & Beskrivning \\\hline
0.1 & 29 september 2014 & TG & Skapande av mall \\
0.2 & 1 oktober 2014 & TG & Kvalitetskrav och regressionstest tillagt \\
0.3 & 1 oktober 2014 & TG & Funktionstester och systemtester för generella krav tillagt\\
0.4 & 1 oktober 2014 & TG & Funktionstester och systemtester för tidrapportering tillagt\\
0.5 & 2 oktober 2014 & TG & Funktionstester för administration tillagt\\
0.6 & 2 oktober 2014 & TG & Ändringar efter intern granskning\\
0.7 & 7 oktober 2014 & TG & Ändringar efter informell granskning och SVVS v0.14\\
0.8 & 10 oktober 2014 & TG & Ändringar efter formell granskning\\

\end{tabular}
\section{1 Inledning}       

Detta dokument innehåller detaljerade testinstruktioner som ska genomföras under utvecklingen av NewPussSystem.

\section{2 Referensdokument}
\begin{enumerate}
\item System Validation and Verification Specification 1.0
\item Software Requirements Specification 0.21
\end{enumerate}



\section{3 Testinstruktioner}
Detaljerade testinstruktioner för testfallen i ref. 1. Testfall dokumenteras i formen av steg som utförs av testaren eller automatiskt av systemet. När steg börjas med ``Kontrollera att'' så beskriver de resultat från systemet, och testaren bör kontrollera att det verkligen har inträffat. Detta kallades i ref. 1 för verklig miljö.

\section{Funktionstest}


%+++++++++++++++++++++++++++% 
%BÖRJAN på FT:generella krav%
%+++++++++++++++++++++++++++%
%Kommentarer:
%Det behövs definieras vilka
%sidor och menyalternativ som
%skall finnas.
%
%"Försök ge inkorrekt input
%till systemet" är väldigt
%luddig.
%
%Duplicerat test, "Adminis-
%tratören försöker lägga 
%till sig själv" finns i 
%FT och ST
%
%Hur relaterar flödesdiagrammen
%till menyval?

\subsection{Generella krav}

\subsubsection{Användare}
\begin{FT}
\item \textbf{Menytillgång}

\emph{Starttillstånd:} Projektmedlem M är inte inloggad. Projektledare L är inte inloggad. Administratören A är inte inloggad.

\emph{Sluttillstånd:} Projektmedlem M är inte inloggad. Projektledare L är inte inloggad. Administratören A är inte inloggad.

\begin{enumerate}
\item Genomför steg 2-4 för M, L och A. Därefter är testet avslutat.
\item Logga in i systemet med rätt lösenord och rätt användarnamn.
\item För alla sidor i 3a-c kontrollera att menyn finns tillgänglig.
\begin{enumerate}
\item För M, navigera till samtliga sidor specificerade i figur 6 (Ref. 2).
\item För L, navigera till samtliga sidor specificerade i figur 7 (Ref. 2).
\item För A, navigera till samtliga sidor specificerade i figur 5 (Ref. 2).
\end{enumerate}
\item Logga ut ur systemet.
\end{enumerate}

\item \textbf{Menyinnehåll}

\emph{Starttillstånd:} Projektmedlem M är inte inloggad. Projektledare L är inte inloggad. Administratören A är inte inloggad.

\emph{Sluttillstånd:} Projektmedlem M är inte inloggad. Projektledare L är inte inloggad. Administratören A är inte inloggad.

\begin{enumerate}
\item Genomför steg 2-4 för M, L och A. Därefter är testet avslutat.
\item Logga in i systemet med rätt lösenord ch rätt användarnamn.
\item Kontrollera att menyn dirigerar användaren till de funktionaliteter som respektive användare besitter i 3a-c.
\begin{enumerate}
\item För M, navigera till samtliga sidor specificerade i figur 6 (Ref. 2) från menyn.
\item För L, navigera till samtliga sidor specificerade i figur 7 (Ref. 2) från menyn.
\item För A, navigera till samtliga sidor specificerade i figur 5 (Ref. 2) från menyn.
\end{enumerate}
\item Logga ut ur systemet.
\end{enumerate}

\item \textbf{Menyn är konsekvent}

\emph{Starttillstånd:} Projektmedlem M är inte inloggad. Projektledare L är inte inloggad. Administratören A är inte inloggad.

\emph{Sluttillstånd:} Projektmedlem M är inte inloggad. Projektledare L är inte inloggad. Administratören A är inte inloggad.

\begin{enumerate}
\item Genomför steg 2-4 för V, M, L och A. Därefter är testet avslutat.
\item Logga in i systemet.
\item Kontrollera att innehållet i menyn är samma för varje specifik användare i 3a-c.
\begin{enumerate}
\item För M, navigera till samtliga sidor specificerade i figur 6 (Ref. 2).
\item För L, navigera till samtliga sidor specificerade i figur 7 (Ref. 2).
\item För A, navigera till samtliga sidor specificerade i figur 5 (Ref. 2).
\end{enumerate}
\item Logga ut ur systemet.
\end{enumerate}

\item \textbf{Skadlig input}

\emph{Starttillstånd:} Administratören A är inte inloggad.

\emph{Sluttillstånd:} Administratören A är inte inloggad.

\begin{enumerate}
\item Logga in A i systemet.
\item Klicka på Administration i menyn.
\item I adressfältet i webbläsaren, lägg till: ``?deletename=\texttt{"}admin\texttt{"}'', och tryck enter.
\item Kontrollera att ett felmeddelande visas som tyder på att det är fel input.
\item Kontrollera i databasen att användaren ``admin'' fortfarande finns kvar.
\item Logga ut A.
\end{enumerate}

\item \textbf{Projektledarantal och rollantal i en projektgrupp}

\emph{Starttillstånd:} Administratören A är inloggad och befinner sig på sidan ``Redigera projektmedlemmar'' (Figur 5, Ref. 2). Det finns en projektgrupp G, två projektledare P1 och P2 och en projektmedlem M (Alla medlemmar i G).

\emph{Sluttillstånd:} oförändrat

\begin{enumerate}
\item Försök tilldela rollen t4 till M
\item Kontrollera att ett felmeddelande visas som visar att denna roll inte finns tillgänglig.
\item Kontrollera i databasen att M inte har fått rollen t4.
\item Försök tilldela rollen ``Projektledare'' till M.
\item Kontrollera att ett felmeddelande visas som visar att det inte kan finnas fler projektledare.
\item Kontrollera i databasen att M inte har fått rollen ``Projektledare''.
\end{enumerate}

%\item \textbf{Projektledarantal och rollantal i en projektgrupp}
%
%\emph{Starttillstånd:} Administratören A är inloggad och befinner sig på sidan ``Redigera projektmedlemmar'' (Figur 5, Ref. 2). Det finns en projektgrupp G, en projektledare P och två projektmedlemmar M1 och M2 i systemet (Alla medlemmar i G).
%
%\emph{Sluttillstånd:} oförändrat
%
%\begin{enumerate}
%\item Försök lägga till V utan att specificera roll.
%\item Kontrollera att ett felmeddelande visas som visar att du måste välja roll.
%\item Kontrollera i databasen att V inte har blivit tillagd.
%\item Försök lägga till V med rollen t4.
%\item Kontrollera att ett felmeddelande visas som visar att denna roll inte finns tillgänglig.
%\item Kontrollera i databasen att V inte har blivit tillagd.
%\item Försök lägga till V som projektledare i G.
%\item Kontrollera att ett felmeddelande visas.
%\item Kontrollera i databasen att V inte har blivit tillagd i G.
%\end{enumerate}
\end{FT}

%\subsubsection{Projektmedlem}
\subsubsection{Projektledare}
\begin{FT}
\item \textbf{Projektledarantal}

\emph{Starttillstånd:} Administratören A är inloggad och befinner sig på sidan ``Projektgrupper'' (Figur 5, Ref. 2). Det finns 3 vanliga användare V1, V2 och V3 i systemet.

\emph{Sluttillstånd:} Administratören A är inloggad och befinner sig på sidan ``Projektgrupper''. Det finns en projektgrupp G där V1 och V2 är projektledare och en vanlig användare V3 i systemet.

\begin{enumerate}
\item Klicka på ``Skapa projektgrupp'' och specificera grupp G och användare V1 som projektledare.
\item Lägg till V2 och V3 i projektgruppen. Klicka ``OK''.
\item Klicka på ``Redigera projektmedlemmar'' för projektet G och ge rollen ``Projektledare'' till V2, klicka ``OK''.
\item Ge rollen ``Projektledare'' till V3 för projektgruppen G, klicka ``OK''.
\item Kontrollera att ett felmeddelande visas som visar att det inte kan finnas fler projektledare.
\item Kontrollera i databasen att V3 inte har fått rollen ``Projektledare''.
\end{enumerate}

\end{FT}

\subsubsection{Administratör}
\begin{FT}
\item \textbf{Administratören får inte vara med i ett projekt}

\emph{Starttillstånd:} Administratören A är inloggad och befinner sig på sidan ``Projektgrupper'' (Figur 5, Ref. 2). Det finns en projektgrupp G.

\emph{Sluttillstånd:} Administratören A är inloggad och befinner sig på sidan ``Projektgrupper''. Det finns en projektgrupp G.

\begin{enumerate}
\item För projektet G klicka på ``Lägg till projektmedlem''.
\item Välj ``admin'' i listan över användare, klicka på ``OK''.
\item Kontrollera att ett felmeddelande visas som visar att Administratören inte får vara med i ett projekt.
\item Kontrollera att ``admin'' inte finns tillagd i G.
\end{enumerate}
\end{FT}

\subsubsection{Data}
\begin{FT}
\item \textbf{Förfrågan innan borttagning av projektledare}

\emph{Starttillstånd:} Administratören A är inloggad i systemet och befinner sig på sidan ``Projektgrupper'' (Figur 5, Ref. 2). Det finns två projektledare i systemet, L1 och L2. L1 och L2 är projektledare för projektgrupp G.

\emph{Sluttillstånd:} Administratören A är inloggad i systemet och befinner sig på sidan ``Projektgrupper''. Projektledare L1 finns inte kvar i systemet. Projektledare L2 är projektledare för projektgrupp G.

\begin{enumerate}
\item Klicka på ``Ta bort användare'' för L1.
\item Kontrollera att en bekräftelseruta visas som frågar om du verkligen vill ta bort L1 från G.
\item Klicka ``Ja''.
\item Kontrollera att en lista över användare visas där L1 inte är med.
\item Kontrollera i databasen att L1 inte finns kvar.
\end{enumerate}

\item \textbf{Förfrågan innan borttagning av vanlig användare}

\emph{Starttillstånd:} Administratören A är inloggad i systemet och befinner sig på sidan ``Projektgrupper'' (Figur 5, Ref. 2). Det finns en vanlig användare V i systemet.

\emph{Sluttillstånd:} Administratören A är inloggad i systemet och befinner sig på sidan ``Projektgrupper''. Vanlig användare L finns inte kvar i systemet.

\begin{enumerate}
\item Klicka på ``Ta bort användare'' för V.
\item Kontrollera att en bekräftelseruta visas som frågar om du verkligen vill ta bort L1 från G.
\item Klicka ``Ja''.
\item Kontrollera att en lista över användare visas där V inte är med.
\item Kontrollera i databasen att V inte finns kvar.
\end{enumerate}

\item \textbf{Förfrågan innan ångrad borttagning av projektledare}

\emph{Starttillstånd:} Administratören A är inloggad i systemet och befinner sig på sidan ``Projektgrupper'' (Figur 5, Ref. 2). Det finns en projektledare L i systemet.

\emph{Sluttillstånd:} Administratören A är inloggad i systemet och befinner sig på sidan ``Projektgrupper''. Det finns en projektledare L i systemet.

\begin{enumerate}
\item Klicka på ``Ta bort användare'' för L.
\item Kontrollera att en bekräftelseruta visas som frågar om du verkligen vill ta bort L1 från G.
\item Klicka ``Nej''.
\item Kontrollera att en lista över användare visas där L är med.
\item Kontrollera i databasen att L finns kvar.
\end{enumerate}

\item \textbf{Förfrågan innan ångrad borttagning av vanlig användare}

\emph{Starttillstånd:} Administratören A är inloggad i systemet och befinner sig på sidan ``Projektgrupper'' (Figur 5, Ref. 2). Det finns en vanlig användare V i systemet.

\emph{Sluttillstånd:} Administratören A är inloggad i systemet och befinner sig på sidan ``Projektgrupper''. Det finns en vanlig användare V i systemet.

\begin{enumerate}
\item Klicka på ``Ta bort användare'' för V.
\item Kontrollera att en bekräftelseruta visas som frågar om du verkligen vill ta bort L1 från G.
\item Klicka ``Nej''.
\item Kontrollera att en lista över användare visas där V är med.
\item Kontrollera i databasen att V finns kvar.
\end{enumerate}
\end{FT}

%\subsubsection{Ej inloggad}

%+++++++++++++++++++++++++++% 
%SLUTET på FT:generella krav%
%+++++++++++++++++++++++++++%


%++++++++++++++++++++++++++% 
%BÖRJAN på FT:autentisering%
%++++++++++++++++++++++++++%
\subsection{Autentisering}

\subsubsection{Övergripande}

\begin{FT}
\item
\textbf{Projektmedlem försöker logga in på dator när den redan är inloggad på en annan dator}

\emph{Starttillstånd:} Projektmedlem A inloggad på dator C, A inte inloggad på dator B.

\emph{Sluttillstånd:} A inloggad på C, A inte inloggad på B.

\begin{enumerate}
\item A fyller i korrekt inloggningsinformation på dator B, och klickar på logga in.
\item Kontrollera att A inte är inloggad på dator B.
\item Kontrollera att A är inloggad på dator C.
\item Kontrollera att felmeddelande visas
\end{enumerate}


%Fix Server session%%%%%%%%%%%%%%%%%%%%%%%%%%%%%%%%%%%%%%%%%%%%%%%%%%%%%%%%%%%%%%%%%%%%%%%%%%%%%%%%%%%
%\item
%\textbf{Loginstatus hålls i en server session}
%
%\emph{Starttillstånd:} Användare A inte inloggad, inloggningssidan visas.
%
%\emph{Sluttillstånd:} A inloggad.
%
%\begin{enumerate}
%\item Kontrollera att inloggningsstatusen i sessionen är "ej inloggad".
%\item A loggar in.
%\item Kontrollera att inloggningsstatusen i sessionen är "inloggad".
%\end{enumerate}

\item
\textbf{Loginstatus hålls i en server session}

\emph{Starttillstånd:} Användare A inte inloggad, inloggningssidan visas.

\emph{Sluttillstånd:} A inloggad.

\begin{enumerate}
\item A loggar in
\item Kontrollera att huvudmenyn visas.
\item Starta om servern.
\item Kontrollera att A är utloggad.
\end{enumerate}



\item
\textbf{Administratören försöker skapa en användare med för kort användarnamn}

\emph{Starttillstånd:} Administratören A inloggad, inne på sidan ''Lista användare '', användare B finns inte i systemet.

\emph{Sluttillstånd:} A inloggad, B finns inte i systemet.

\begin{enumerate}
\item A fyller i användarnamnet på B och försöker ge B användarnamnet ''Tord'', 4 tecken.
\item Kontrollera att B är inte i systemet.
\item Kontrollera att ett felmeddelande visar varför B inte skapades.
\end{enumerate}

\item
\textbf{Administratören försöker skapa en användare med för långt användarnamn}

\emph{Starttillstånd:} Administratören A inloggad, inne på sidan ''Lista användare '', användare B finns inte i systemet.

\emph{Sluttillstånd:} A inloggad, B finns inte i systemet

\begin{enumerate}
\item A fyller i användarnamnet på B och försöker ge B användarnamnet ''Alexandersson'', ett användarnamn längre än 11 tecken.
\item Kontrollera att B inte skapas i systemet.
\item Kontrollera att ett felmeddelande visar varför B inte skapades.
\end{enumerate}

\item
\textbf{Administratören försöker skapa en användare med användarnamn som innehåller icke tillåtna tecken}

\emph{Starttillstånd:} Administratören A inloggad, inne på sidan ''Lista användare '', användare B finns inte i systemet.

\emph{Sluttillstånd:} A inloggad, B finns inte i systemet

\begin{enumerate}
\item A fyller i användarnamnet på B (innehåller minst ett tecken från ascii utanför numren (48-57,65-90,97-122)):
\begin{itemize}
\item [] (a) Knasen? %ইঁদুর
\item [] (b) Knasen,
\item [] (c) Knasen123
\end{itemize}
\item Kontrollera att B inte skapas i systemet.
\item Kontrollera att ett felmeddelande visar varför B inte skapades.
\end{enumerate}

\item
\textbf{Administratören försöker skapa en ny användare med ett användarnamn som redan finns registrerat hos systemet}

\emph{Starttillstånd:} Administratören A inloggad, användare B finns registrerat hos systemet.

\emph{Sluttillstånd:} Administratören A kunde inte skapa användare B, Använda B är registrerat redan hos systemet.

\begin{enumerate}
\item A väljer att skapa B.
\item A kan inte skapa B.
\item Kontrollera att det finns bara ett användare B i systemet.
\end{enumerate}

\item
\textbf{Försök byta lösenord till ett med fler eller färre tecken än 6.}

\emph{Starttillstånd:} Användaren A inloggad, sidan för byta lösenord visas.

\emph{Sluttillstånd:} A inloggad, lösenordet inte förändrat, felmeddelande visas.

\begin{enumerate}
\item A fyller i ett nytt lösenord ''abcdefghijkl'' eller ''abcdf''.
\item Kontrollera att A:s lösenord inte förändrats.
\item Kontrollera att ett felmeddelande visar att lösenordet inte ändrats och varför.
\end{enumerate}

\item
\textbf{Försök byta lösenord till ett med otillåtna tecken.}

\emph{Starttillstånd:} Användaren A inloggad, sidan för byta lösenord visas.

\emph{Sluttillstånd:} A inloggad, lösenordet inte förändrat, felmeddelande visas.

\begin{enumerate}
\item A fyller i ett nytt lösenord ''?aaaaa''.
\item Kontrollera att A:s lösenord inte förändrats.
\item Kontrollera att ett felmeddelande visar att lösenordet inte ändrats och varför.
\end{enumerate}

\end{FT}

\subsubsection{Användare}
\begin{FT}
\item
\textbf{Kontrollera att utloggningsfunktionalitet finns på alla inloggade sidor och fungerar.}

\emph{Starttillstånd:} Användaren A inloggad.

\emph{Sluttillstånd:} A utloggad.

\begin{enumerate}
\item A går till en slumpmässig URL som tillhör NewPussSystem.
\item Kontrollera att en utloggningsknapp finns.
\item Klicka på utloggningsknappen.
\item Kontrollera att A är utloggad.
\end{enumerate}

\item
\textbf{En användare som är inaktiv i 20 min blir utloggad.}

\emph{Starttillstånd:} Användaren A inloggad.

\emph{Sluttillstånd:} A utloggad.

\begin{enumerate}
\item A går till en slumpmässig URL som tillhör NewPussSystem.
\item A rör ingenting i 20 minuter.
\item Försök komma åt någon URL eller funktion som kräver en inloggad användare.
\item Kontrollera att A är utloggad, och funktionallitet som kräver inloggad användare ej är tillgänglig.
\item Kontrollera att ett meddelande visas som informerar om vad som hänt.
\item Kontrollera att A dirigeras om till inloggningssidan.
\end{enumerate}
\end{FT}

\subsubsection{Administratör}
\begin{FT}
\item
\textbf{Administratören kan ta bort alla användare utom sig själv ur systemet.}

\emph{Starttillstånd:} Administratören A inloggad, sidan för att ta bort användare visas, användare B, projektledare C finns i systemet.

\emph{Sluttillstånd:} B finns inte i systemet.

\begin{enumerate}
\item A väljer att ta bor B.
\item A tar bort B.
\item A väljer att ta bor C.
\item A tar bort C.
\item A fösöker att ta bor A.
\item A kan inte ta bort A.
\item Kontrollera att B och C har tagits bort.
\item Kontrollera att A är i systemet.

\end{enumerate}

\item
\textbf{Administratören kan inte ta bort sig själv.}

\emph{Starttillstånd:} Administratören A inloggad, sidan för att ta bort användare visas.

\emph{Sluttillstånd:} A finns kvar i systemet, felmeddelande visas.

\begin{enumerate}
\item A försöker ta bort A.
\item Kontrollera att A inte är borttagen.
\item Kontrollera att ett felmeddelande visas.
\end{enumerate}

\end{FT}

\subsubsection{Data}
\begin{FT}
\item
\textbf{Kontrollera givna användaridentiteter mot de registrerade användare som finns i systemet.}

\emph{Starttillstånd:} Användaren A inte inloggad, sidan för inloggning visas.

\emph{Sluttillstånd:} A inloggad.

\begin{enumerate}
\item A fyller i sitt användarnamn och lösenord sedan klickar den på logga in.
\item Kontrollera att A bara loggas in om A finns registrerad i systemet.
\item Kontrollera att A omdirigerades till den sida som har användarfunktionerna.
\item Kontrollera att A är inloggad.
\item Kontrollera att A är på sidan som har användarfunktionerna.
\end{enumerate}
\end{FT}

\subsubsection{Ej inloggad}
\begin{FT}
\item
\textbf{En inte inloggad användare når systemet och tvingas då lämna inloggninsinformation.}

\emph{Starttillstånd:} Användaren A inte inloggad.

\emph{Sluttillstånd:} A inte inloggad, inloggningssidan visas.

\begin{enumerate}
\item Skriver in URL för ädringa av lösenord till funktionalitet "Ändra lösenord".
\item Kontrollera att A omdirigeras till inloggningssidan.
\item Kontrollera att A inte kommer vidare utan att logga in.
\end{enumerate}

\item
\textbf{En användare kan välja mellan alla befintliga projektgrupper i systemet på inloggningssidan.}

\emph{Starttillstånd:} Användaren A inte inloggad.

\emph{Sluttillstånd:} A inte inloggad.

\begin{enumerate}
\item A går till inloggningssidan.
\item Kontrollera att A kan välja mellan alla befintliga projektgrupper.
\end{enumerate}

\item
\textbf{En användare skall specifiera vilken projektgrupp den vill logga in på.}

\emph{Starttillstånd:} Användaren A inte inloggad.

\emph{Sluttillstånd:} A är inte inloggad.

\begin{enumerate}
\item A går till inloggningssidan.
\item A försöker logga in utan att välja en projektgrupp
\item Kontrollera att A inte är inloggad.
\item Kontrollera att ett felmeddelande visas.
\end{enumerate}

\item
\textbf{En användare lyckas logga in på den/de projektgrupp(er) som den är medlem i.}

\emph{Starttillstånd:} Användaren A inte inloggad, A tillhör projektgrupp B.

\emph{Sluttillstånd:} A är inloggad.

\begin{enumerate}
\item A går till inloggningssidan.
\item A väljer projektgrupp B och loggar in.
\item Kontrollera att A är inloggad.
\end{enumerate}

\item
\textbf{En användare försöker logga in på en projektgrupp som denne inte är medlem i.}

\emph{Starttillstånd:} Användaren A inte inloggad, A är medlem i projektgrupp B, A tillhör inte projektgrupp C.

\emph{Sluttillstånd:} A är inte inloggad.

\begin{enumerate}
\item A går till inloggningssidan.
\item A väljer projektgrupp C och försöker logga in.
\item Kontrollera att A inte är inloggad.
\item Kontrollera att ett felmeddelande visas.
\end{enumerate}

\item
\textbf{Administratören lyckas logga in på samtliga projektgrupper.}

\emph{Starttillstånd:} Administratören A inte inloggad.

\emph{Sluttillstånd:} A är inloggad.

\begin{enumerate}
\item A går till inloggningssidan.
\item A väljer någon projektgrupp och loggar in.
\item Kontrollera att A är inloggad.
\end{enumerate}
\end{FT}








%++++++++++++++++++++++++++% 
%SLUTET på FT:autentisering%
%++++++++++++++++++++++++++%



%Johan
%++++++++++++++++++++++++++++% 
%BÖRJAN på FT:tidrapportering%
%++++++++++++++++++++++++++++%
%+++++++++++++++++++++++++++++
%+++++++++++++++++++++++++++++
%+++++++++++++++++++++++++++++
%+++++++++++++++++++++++++++++
%++++++++++++++++++++++++++++++
%++++++++++++++++++++++++
%++++++++++++++++++++++++++++

%Johan
\subsection{Tidrapportering}

%tidrapportering
\subsubsection{Projektmedlem}

\begin{FT}

\item\textbf{Projektmedlem lyckas skapa en egen osignerad tidrapport}

\emph{Starttillstånd:} Projektmedlem A inloggad, inne på funktionalitetssidan för tidrapportering, se figur 3 i referens 2.

\emph{Sluttillstånd:} Projektmedlem A inloggad, inne på funktionalitetssidan för tidrapportering, ny tidrapport visas.

\begin{enumerate}
\item A trycker på ``Skapa en ny tidrapport''.
\item Kontrollera i databasen att tidrapportern är skapad.
\item Kontrollera i databasen att tidrapporten inte är signerad när den skapas.
\end{enumerate}


\item\textbf{Projektmedlem lyckas uppdatera sin egna osignerade tidrapport}

\emph{Starttillstånd:} Projektmedlem A inloggad, inne på funktionalitetssidan för tidrapportering, se figur 3 i referens 2. En osignerad tidrapport T med 20 minuter registrerat på Aktivitet 1, Subaktivitet 1, finns i systemet.

\emph{Sluttillstånd:} Projektmedlem A inloggad, inne på funktionalitetssidan för tidrapportering, T har nu 30 minuter registrerat på Aktivitet 1, Subaktivitet 1.

\begin{enumerate}
\item A trycker på ``Uppdatera tidrapport''.
\item A väljer en av sina egna osignerade tidrapporter.
\item A skriver in 30 i fältet Aktivitet 1, Subaktivitet 1.
\item A trycker på ``Spara''.
\item Kontrollera i databasen att T har 30 minuter på Aktivitet 1, Subaktivitet 1.
\end{enumerate}


\item\textbf{Projektmedlem lyckas ta bort sin egna osignerade tidrapport}

\emph{Starttillstånd:} Projektmedlem A inloggad, inne på funktionalitetssidan för tidrapportering, se figur 3 i referens 2. En tidrapport finns i systemet.

\emph{Sluttillstånd:} Projektmedlem A inloggad, inne på funktionalitetssidan för tidrapportering. Tidrapport raderad.

\begin{enumerate}
\item A trycker på ``Ta bort tidrapport''.
\item A väljer en av sina egna osignerade tidrapporter.
\item A trycker på knappen ``Radera''.
\item Kontrollera i databasen att tidrapporten är raderad.
\end{enumerate}


\item
\textbf{Vid tidrapporteringsfunktionaliten kan en projektmedlem endast se sina egna tidrapporter} 

\emph{Starttillstånd:} Projektmedlem A inloggad, inne på funktionalitetssidan för tidrapportering, se figur 3 i referens 2. A har två sparade tidrapporter. Projektmedlem B har två sparade tidrapporter.

\emph{Sluttillstånd:} Oförändrat.

\begin{enumerate}
\item A tryck på ``Visa tidrapporter''.
\item Kontrollera att denne endast ser sina egna två tidrapporter.
\end{enumerate}


\item
\textbf{Projektmedlem försöker ta bort en av sina signerade tidrapporter}

\emph{Starttillstånd:} Projektmedlem A inloggad, inne på funktionalitetssidan för tidrapportering, se figur 3 i referens 2. A har en signerad tidrapport i systemet.

\emph{Sluttillstånd:} Oförändrat.

\begin{enumerate}
\item A trycker på ``Ta bort tidrapporter''.
\item A väljer sin signerade tidrapport.
\item Kontrollera att funktionaliteten för att radera tidrapporten inte finns.
\end{enumerate}

\item
\textbf{Projektmedlem försöker redigera en signerad tidrapport}

\emph{Starttillstånd:} Projektmedlem A inloggad, inne på funktionalitetssidan för tidrapportering, se figur 3 i referens 2. A har en signerad tidrapport i systemet.

\emph{Sluttillstånd:} Oförändrat.

\begin{enumerate}
\item A trycker på ``Uppdatera tidrapport''.
\item A väljer sin signerade tidrapport.
\item Kontrollera att funktionaliteten för att redigera tidrapporten inte finns.
\end{enumerate}

\end{FT}


\subsubsection{Projektledare}

\begin{FT}

\item
\textbf{Projektledaren har tillgång till samtliga projektmedlemmars tidrapporter i sin projektgrupp}

\emph{Starttillstånd:} Projektledare är A inloggad, inne på ``Hantera tidrapporter'', se figur 4 i referens 2. A är projektledare i projektgrupp G. M1 och M2 är projektmedlemmar i G. M1 har en tidrapportering T1 för Aktivitet 1, Subaktivitet 1, i vecka 1 på 10 minuter och en T2 i vecka 2 på 20 minuter. M2 har en tidrapportering T3 för Aktivitet 1, Subaktivitet 1, i vecka Aktivitet 1, Subaktivitet 1 på 10 minuter och en T4 i vecka 3 på 20 minuter. 

\emph{Sluttillstånd:} Oförändrat.

\begin{enumerate}
\item A trycker på ``generera tidrapporter''.
\item Kontrollera att A kan se tidrapporterna T1 till T4.
\end{enumerate}


\item
\textbf{Projektledaren lyckas godkänna en ej tidigare signerad tidrapport från en medlem i sin projektgrupp}

\emph{Starttillstånd:} Projektledare är A inloggad, inne på ``Hantera tidrapporter'', se figur 4 i referens 2. A är projektledare i projektgrupp G. M är en projektmedlem i G. M har en tidrapportering T för Aktivitet 1, Subaktivitet 1, i vecka 1 på 10 minuter. T är osignerad.

\emph{Sluttillstånd:} Oförändrat förutom att T är signerat.

\begin{enumerate}
\item A trycker på ``Signera'' vid T.
\item T visas.
\item A trycker på ``Signera''.
\item Kontrollera i databasen att tidrapporten är signerad.
\end{enumerate}


\item
\textbf{Projektledaren lyckas ta tillbaka sitt godkännande från en tidigare godkänd tidrapport}

\emph{Starttillstånd:} Projektledare är A inloggad, inne på ``Hantera tidrapporter'', se figur 4 i referens 2. A är projektledare i projektgrupp G. M är en projektmedlem i G. M har en tidrapportering T för Aktivitet 1, Subaktivitet 1, i vecka 1 på 10 minuter. T är signerad.

\emph{Sluttillstånd:} Oförändrat förutom att T är osignerad.

\begin{enumerate}
\item A trycker på ``Avsignera'' vid T.
\item T visas.
\item A trycker på ``Avsignera''.
\item Kontrollera i databasen att tidrapporten är signerad.
\end{enumerate}



\item
\textbf{Projektledaren lyckas generera statistik i form av tidrapporter per användare för samtliga veckor}

\emph{Starttillstånd:} Projektledare är A inloggad, inne på ``Generera statistik'', se figur 4 i referens 2. A är projektledare i projektgrupp G. M är projektmedlem i G. M har en tidrapportering T1 för Aktivitet 1, Subaktivitet 1, i vecka 2 på 10 minuter, en T2 i vecka 3 på 20 minuter och en T3 i vecka 4 på 30 minuter. 

\emph{Sluttillstånd:} Oförändrat.

\begin{enumerate}
\item A väljer M och ``Samtliga veckor''.
\item A trycker på ``Generera statistik''.
\item Statistiken visas i form av en tidrapport med M och värdet 60 på Aktivitet 1, Subaktivitet 1.
\item Klicka på ``Tillbaka''.
\end{enumerate}



\item
\textbf{Projektledaren lyckas generera statistik i form av tidrapporter per roll för samtliga veckor}

\emph{Starttillstånd:} Projektledare är A inloggad, inne på ``Generera statistik'', se figur 4 i referens 2. A är projektledare i projektgrupp G. M1 och M2 är projektmedlemmar i G, båda har rollen t1. M1 har en tidrapportering T1 för Aktivitet 1, Subaktivitet 1, i vecka 2 på 10 minuter, en T2 i vecka 3 på 20 minuter och en T3 i vecka 4 på 30 minuter. M2 har en tidrapportering T4 för Aktivitet 1, Subaktivitet 1, i vecka 1 på 50 minuter och en T5 i vecka 2 på 20 minuter.

\emph{Sluttillstånd:} Oförändrat.

\begin{enumerate}
\item A väljer t1 och ``Samtliga veckor''.
\item A trycker på ``Generera statistik''.
\item Statistiken visas i form av en tidrapport för rollen t1 och värdet 130 på Aktivitet 1, Subaktivitet 1.
\item Klicka på ``Tillbaka''.
\end{enumerate}


\item
\textbf{Projektledaren lyckas generera statistik i form av tidrapporter per aktivitet}

\emph{Starttillstånd:} Projektledare är A inloggad, inne på ``Generera statistik'', se figur 4 i referens 2. A är projektledare i projektgrupp G. M1 och M2 är projektmedlemmar i G, båda har rollen t1. M1 har en tidrapportering T1 för Aktivitet 1, Subaktivitet 1, i vecka 2 på 10 minuter, en T2 i vecka 3 på 20 minuter och en T3 i vecka 4 på 30 minuter. M2 har en tidrapportering T4 för Aktivitet 1, Subaktivitet 2, i vecka 1 på 50 minuter och en T5 i vecka 2 på 20 minuter.

\emph{Sluttillstånd:} Oförändrat.

\begin{enumerate}
\item A väljer Aktivitet 1 och ``Samtliga veckor''.
\item A trycker på ``Generera statistik''.
\item Statistiken visas i form av en tidrapport för med endast Aktivitet 1 ifylld med värdet 30 på Subaktivitet 1 och 70 på Subaktivitet 2.
\item Klicka på ``Tillbaka''.
\end{enumerate}


\item
\textbf{Projektledaren lyckas generera statistik i form av tidrapporter per vecka}

\emph{Starttillstånd:} Projektledare är A inloggad, inne på ``Generera statistik'', se figur 4 i referens 2. A är projektledare i projektgrupp G. M1 och M2 är projektmedlemmar i G, båda har rollen t1. M1 har en tidrapportering T1 för Aktivitet 1, Subaktivitet 1, i vecka 2 på 10 minuter, en T2 i vecka 3 på 20 minuter och en T3 i vecka 4 på 30 minuter. M2 har en tidrapportering T4 för Aktivitet 1, Subaktivitet 2, i vecka 1 på 50 minuter och en T5 i vecka 2 på 20 minuter.

\emph{Sluttillstånd:} Oförändrat.

\begin{enumerate}
\item A väljer både M1 och M2 och vecka 2.
\item A trycker på ``Generera statistik''.
\item Statistiken visas i form av en tidrapport med M1 och M2 och värdet 10 på Aktivitet 1, Subaktivitet 1 och 20 på Aktivitet 1, Subaktivitet 2.
\item Klicka på ``Tillbaka''.
\end{enumerate}


\item
\textbf{Projektledaren lyckas generera statistik i form av tidrapporter per användare och aktivitet}

\emph{Starttillstånd:} Projektledare är A inloggad, inne på ``Generera statistik'', se figur 4 i referens 2. A är projektledare i projektgrupp G. M1 och M2 är projektmedlemmar i G, båda har rollen t1. M1 har en tidrapportering T1 för Aktivitet 1, Subaktivitet 1, i vecka 2 på 10 minuter, en T2 för Aktivitet 1, Subaktivitet 2, i vecka 3 på 20 minuter och en T3 för Aktivitet 2, Subaktivitet 1 i vecka 4 på 30 minuter. M2 har en tidrapportering T4 för Aktivitet 1, Subaktivitet 2, i vecka 1 på 50 minuter och en T5 i vecka 2 på 20 minuter.

\emph{Sluttillstånd:} Oförändrat.

\begin{enumerate}
\item A väljer M1 och Aktivitet 1.
\item A trycker på ``Generera statistik''.
\item Statistiken visas i form av en tidrapport för M1 med värdet 10 på Aktivitet 1, Subaktivitet 1 och 20 på Aktivitet 1, Subaktivitet 2.
\item Klicka på ``Tillbaka''.
\end{enumerate}


\item
\textbf{Projektledaren lyckas generera statistik i form av tidrapporter per användare för utvalda veckor}

\emph{Starttillstånd:} Projektledare är A inloggad, inne på ``Generera statistik'', se figur 4 i referens 2. A är projektledare i projektgrupp G. M1 och M2 är projektmedlemmar i G, båda har rollen t1. M1 har en tidrapportering T1 för Aktivitet 1, Subaktivitet 1, i vecka 2 på 10 minuter, en T2 för Aktivitet 1, Subaktivitet 2, i vecka 3 på 20 minuter och en T3 för Aktivitet 2, Subaktivitet 1 i vecka 4 på 30 minuter. M2 har en tidrapportering T4 för Aktivitet 1, Subaktivitet 2, i vecka 1 på 50 minuter och en T5 i vecka 2 på 20 minuter.

\emph{Sluttillstånd:} Oförändrat.

\begin{enumerate}
\item A väljer M1 för vecka 1, 2 och 4.
\item A trycker på ``Generera statistik''.
\item Statistiken visas i form av en tidrapport för M1 med värdet 10 på Aktivitet 1, Subaktivitet 1 och 30 på Aktivitet 2, Subaktivitet 1.
\item Klicka på ``Tillbaka''.
\end{enumerate}


\item
\textbf{Projektledaren lyckas generera statistik i form av tidrapporter per roll och aktivitet}

\emph{Starttillstånd:} Projektledare är A inloggad, inne på ``Generera statistik'', se figur 4 i referens 2. A är projektledare i projektgrupp G. M1, M2 och är projektmedlemmar i G, M1 och M2 har rollen t1, M3 har rollen t2. M1 har en tidrapportering T1 för Aktivitet 1, Subaktivitet 1, i vecka 2 på 10 minuter, en T2 för Aktivitet 1, Subaktivitet 2, i vecka 3 på 20 minuter och en T3 för Aktivitet 2, Subaktivitet 1 i vecka 4 på 30 minuter. M2 har en tidrapportering T4 för Aktivitet 1, Subaktivitet 2, i vecka 1 på 50 minuter och en T5 i vecka 2 på 20 minuter. M3 har en tidrapportering T5 för Aktivitet 1, Subaktivitet 2, i vecka 1 på 40 minuter och en T6 i vecka 2 på 20 minuter.

\emph{Sluttillstånd:} Oförändrat.

\begin{enumerate}
\item A väljer t1 och Aktivitet 1.
\item A trycker på ``Generera statistik''.
\item Statistiken visas i form av en tidrapport med värdet 10 på Aktivitet 1, Subaktivitet 1, 90 på Aktivitet 1, Subaktivitet 2 och 30 på Aktivitet 2, Subaktivitet 1.
\item Klicka på ``Tillbaka''.
\end{enumerate}


\item
\textbf{Projektledaren lyckas generera statistik i form av tidrapporter per roll för utvalda veckor}

\emph{Starttillstånd:} Projektledare är A inloggad, inne på ``Generera statistik'', se figur 4 i referens 2. A är projektledare i projektgrupp G. M1, M2 och är projektmedlemmar i G, M1 och M2 har rollen t1, M3 har rollen t2. M1 har en tidrapportering T1 för Aktivitet 1, Subaktivitet 1, i vecka 2 på 10 minuter, en T2 för Aktivitet 1, Subaktivitet 2, i vecka 3 på 20 minuter och en T3 för Aktivitet 2, Subaktivitet 1 i vecka 4 på 30 minuter. M2 har en tidrapportering T4 för Aktivitet 1, Subaktivitet 2, i vecka 1 på 50 minuter och en T5 i vecka 2 på 20 minuter. M3 har en tidrapportering T5 för Aktivitet 1, Subaktivitet 2, i vecka 1 på 40 minuter och en T6 i vecka 2 på 20 minuter.

\emph{Sluttillstånd:} Oförändrat.

\begin{enumerate}
\item A väljer t1 och vecka 1, 2 och 4.
\item A trycker på ``Generera statistik''.
\item Statistiken visas i form av en tidrapport med värdet 10 på Aktivitet 1, Subaktivitet 1, 30 på Aktivitet 2, Subaktivitet 1 och 70 på Aktivitet 1, Subaktivitet 2.
\item Klicka på ``Tillbaka''.
\end{enumerate}


\item
\textbf{Projektledaren lyckas generera statistik i form av tidrapporter per aktivitet och vecka}

\emph{Starttillstånd:} Projektledare är A inloggad, inne på ``Generera statistik'', se figur 4 i referens 2. A är projektledare i projektgrupp G. M1, M2 och är projektmedlemmar i G, M1 och M2 har rollen t1, M3 har rollen t2. M1 har en tidrapportering T1 för Aktivitet 1, Subaktivitet 1, i vecka 2 på 10 minuter, en T2 för Aktivitet 1, Subaktivitet 2, i vecka 3 på 20 minuter och en T3 för Aktivitet 2, Subaktivitet 1 i vecka 4 på 30 minuter. M2 har en tidrapportering T4 för Aktivitet 1, Subaktivitet 2, i vecka 1 på 50 minuter och en T5 i vecka 2 på 20 minuter. M3 har en tidrapportering T5 för Aktivitet 1, Subaktivitet 2, i vecka 1 på 40 minuter och en T6 i vecka 2 på 20 minuter.

\emph{Sluttillstånd:} Oförändrat.

\begin{enumerate}
\item A väljer Aktivitet 1 för vecka 2 och 4.
\item A trycker på ``Generera statistik''.
\item Statistiken visas i form av en tidrapport med värdet 10 på Aktivitet 1, Subaktivitet 1 och 130 på Aktivitet 1, Subaktivitet 2.
\item Klicka på ``Tillbaka''.
\end{enumerate}


\item
\textbf{Projektledaren kan se sammalagd arbetstid från valda tidrapporter}

\emph{Starttillstånd:} Projektledare är A inloggad, inne på ``Generera statistik'', se figur 4 i referens 2. A är projektledare i projektgrupp G. M1 är projektmedlem i G. M har en tidrapportering T i vecka 1 med 20 minuter på Aktivitet 1, Subaktivitet 1, 10 minuter på Aktivitet 2, Subaktivitet 1 och 50 minuter på Aktivitet 2, Subaktivitet 2.

\emph{Sluttillstånd:} Oförändrat.

\begin{enumerate}
\item A väljer M för vecka 1.
\item Kontrollera att A kan se sammanlagd arbetstid från tidrapporten.
\end{enumerate}

% \item
% \emph{Manuell miljö:} Projektledaren kan se sammalagd arbetstid från en vald aktivitet från valda tidrapporter \req[6.3.12-20]

% %ändrad 23/9**
% \item
% \emph{Manuell miljö:} Projektledaren kan se sammalagd arbetstid från en vald subaktivitet från valda tidrapporter \req[6.3.14-21]
% \item
% \emph{Manuell miljö:} Projektledaren kan se hur ofta en vald aktivitet har utförts i valda tidrapporter \req[6.3.14-21]

% %ändrad 23/9**
% \item
% \emph{Manuell miljö:} Projektledaren kan se hur ofta en vald subaktivitet har utförts i valda tidrapporter \req[6.3.14-21]

% %ändrad 23/9**
% \item
% \emph{Manuell miljö:} Projektledaren kan se vilken aktivitet som har tagit mest tid från tidrapportererna \req[6.3.14-21]

% %ändrad 23/9**
% \item
% \emph{Manuell miljö:} Projektledaren kan se vilken aktivitet som är vanligast från tidrapportererna  \req[6.3.14-21]

% %ändrad 23/9**
% \item
% \emph{Manuell miljö:} Projektledaren kan se vilken vecka som har mest rapporterad tid från tidrapportererna \req[6.3.14-21]
\end{FT}


%tidrapportering
\subsubsection{Data}

\begin{FT}

\item Utgått
\item Utgått

\item
\textbf{En tidrapport innehåller information om användarnamn}

\emph{Starttillstånd:} Inloggad på mysql\\
\emph{Sluttillstånd:} Inloggad på mysql. Alla tidrapporter innehåller användarnamn\\

\begin{enumerate}
\item Få tillgång till databasen
\item Visa tidrapporterna
\item Hämta användarnamn från samtliga tidrapporter.
\item Kontrollera att inget av de hämtade värdena är null.
\end{enumerate}

\item
\textbf{En tidrapport innehåller information om projektgruppsnamn}

\emph{Starttillstånd:} Inloggad på mysql\\
\emph{Sluttillstånd:} Inloggad på mysql. Alla tidrapporter innehåller projektgruppsnamn\\

\begin{enumerate}
\item Få tillgång till databasen
\item Visa tidrapporterna
\item Hämta projektgruppsnamn från samtliga tidrapporter.
\item Kontrollera att inget av de hämtade värdena är null.
\end{enumerate}



\item
\textbf{En tidrapport innehåller information om datum i form av "dag månad år"}

\emph{Starttillstånd:} Inloggad på mysql\\
\emph{Sluttillstånd:} Inloggad på mysql. Alla tidrapporter innehåller datum i "dag månad år".\\

\begin{enumerate}
\item Få tillgång till databasen
\item Visa tidrapporterna
\item Hämta datum från samtliga tidrapporter.
\item Kontrollera att inget av de hämtade värdena är null.
\end{enumerate}


\item
\textbf{En tidrapport innehåller information om veckonummer}

\emph{Starttillstånd:} Inloggad på mysql\\
\emph{Sluttillstånd:} Inloggad på mysql. Alla tidrapporter innehåller veckonummer.\\

\begin{enumerate}
\item Få tillgång till databasen
\item Visa tidrapporterna
\item Hämta veckonummer från samtliga tidrapporter.
\item Kontrollera att inget av de hämtade värdena är null.
\end{enumerate}


\item
\textbf{\emph{Manuell miljö:} Det ska framgå tydligt att en tidrapport är signerad eller inte}

\emph{Starttillstånd:} Projektmedlem A inloggad, inne på funktionalitetssidan för tidrapportering.\\
\emph{Sluttillstånd:} Projektmedlem A inloggad, inne på funktionalitetssidan för tidrapportering.

\begin{enumerate}
\item A Tryck på visa tidrapporter
\item A väljer godtycklig tidrapport
\item Kontrollera att det visas tydligt om tidrapporten är godkänd eller ej.
\end{enumerate}

\end{FT}
%-------------------------------
%----------------------------
%-----------------------------
%-------------------------------
%-----------------------------
%+++++++++++++++++++++++++++%
%SLUT på FT:Tidrapportering+%
%+++++++++++++++++++++++++++%






%+++++++++++++++++++++++++++%
%början på FT:Administration%
%+++++++++++++++++++++++++++%
\subsection{Administration}

\subsubsection{Övergripande}
\begin{FT}
\item %FT 4.1.1 
\textbf{Kontrollera att man från ''välj funktion'' sidan kan komma åt ''välj administrationsverktyg'' se figur 5 i ref 2.}

\emph{Starttillstånd:}  Administratör A är inloggad och är på ''välj funktion'' sidan.

\emph{Sluttillstånd:} Oförändrad.

\begin{enumerate}
\item A är på ''välj funktion'' sidan.
\item kontrollera att ''administrationsvyn'' finns tillgänglig som val. 

\end{enumerate}

\item %FT 4.1.2 
\textbf{Administratören navigerar till ''välj administrationsverktyg'' se figur 5 i refenrens 2.}

\emph{Starttillstånd:} Administratören är på ''välj funktion'' sidan se figur 5 i ref 2.

\emph{Sluttillstånd:} Administratör är på ''välj administrationsverktyg''.

\begin{enumerate}
\item Administratören klickar på ''administrationsvyn''.
\item Kontrollera att administratören är inne på ''välj administrationsverktyg''.
\end{enumerate}

\item %FT 4.1.3 
\textbf{En projektmedlem försöker navigera till ''välj administrationsverktyg'' se figur 5 i ref 2.}

\emph{Starttillstånd:} Projektmedlem M är på ''välj funktion'' sidan se figur 3 i ref 2.

\emph{Sluttillstånd:} A är på huvudsidan

\begin{enumerate}
\item M klickar på ''administrationsvyn'' se figur 5 i ref 2.
\item kontrollera att A är på "välj funktion" sidan. 
\end{enumerate}

\item %FT 4.1.4 
\textbf{En projektledare försöker navigera till ''välj administrationsverktyg'' se figur 5 i ref 2.}

\emph{Starttillstånd:} Projektledare L är inloggad

\emph{Sluttillstånd:} A är på huvudsidan 

\begin{enumerate}
\item L klickar på ''administrationsvyn'' se figur 5 i ref 2.
\item Kontrollera att A är kvar på ''välj funktion'' sidan se figur 4 i ref 2.

\end{enumerate}
\end{FT}

\subsubsection{Projektledare}
\begin{FT}

\item %FT 4.2.1 
\textbf{Projektledaren tilldelar en roll till en projektmedlem i sin projektgrupp}

\emph{Starttillstånd:} Projektledaren L är inloggat och har tillgång till sin projektgrupp. En projektmedlem M finns i projektledarens grupp.

\emph{Sluttillstånd:} M har fått en roll t1 i sin projektgrupp.

\begin{enumerate}
\item P tilldelar roll t1 till M.
\item P sparar ändringar.
\item P har fått bekräftelse att M har fått roll som t1.
\item Kontrollera att M har rollen t1 genom att gå till sidan ''Lista projektmedlemmar'' se figur 7 i ref 2.
\end{enumerate}

\item %FT 4.2.2 
\textbf{Projektledaren listar alla osignerade tidrapporter}

\emph{Starttillstånd:} Projektledaren L är inloggad och på sida ''alternativ'' se figur 7 i ref 2. Två osignerade och tre signerade tidrapporter med unika upphovsmän finns i systemet under L:s projektgrupp.

\emph{Sluttillstånd:} Alla ej godkända tidrapporter är listade.                          

\begin{enumerate}
\item L väljer ''redigera tidrapport'' och väljer sedan osignerade rapporter.
\item Kontrollera att två osignerade tidrapporter är listade.
\end{enumerate}

\item %FT 4.2.3
\textbf{Projektledaren listar alla godkända tidrapporter}

\emph{Starttillstånd:} Projektledaren L är inloggad och på sida ''alternativ'' se figur 7 i ref 2. Två osignerade och tre signerade tidrapporter med unika upphovsmän finns i systemet under L:s projektgrupp.

\emph{Sluttillstånd:} Alla signerade tidrapporter är listade.                          

\begin{enumerate}
\item L väljer ''redigera tidrapport'' och väljer sedan signerade rapporter.
\item Kontrollera att tre signerade tidrapporter är listade.
\end{enumerate}

\item %FT 4.2.4 
\textbf{Projektledaren listar projektets alla tidrapporter och sorterar dom i stigande ordning efter användare}

\emph{Starttillstånd:} Projektledare L är inloggad och på sidan ''alternativ'' se figur 7 ref 2. Två osignerade och tre signerade tidrapporter med unika upphovsmän finns i systemet under L:s projektgrupp.

\emph{Sluttillstånd:} Projektets alla tidrapporter är sorterade i stigande ordning efter användare.

\begin{enumerate}
\item L väljer ''redigera tidrapport''.
\item L sorterar i stigande ordning efter användare.
\item Kontrollera att 5 tidrapporter är sorterade i stigande ordning efter användare.
\end{enumerate}

\item %FT 4.2.5 
\textbf{Projektledaren listar projektets alla tidrapporter och sorterar dom i fallande ordning efter användare}

\emph{Starttillstånd:} Projektledare L är inloggad och på sidan ''alternativ'' se figur 7 ref 2. Två osignerade och tre signerade tidrapporter med unika upphovsmän finns i systemet under P:s projektgrupp.

\emph{Sluttillstånd:} Projektets alla tidrapporter är sorterade i fallande ordning efter användare.

\begin{enumerate}
\item L väljer ''redigera tidrapport''.
\item L sorterar i fallande ordning efter användare.
\item Kontrollera att 5 tidrapporter är sorterade i fallande ordning efter användare.
\end{enumerate}

\item %FT 4.2.6 
\textbf{Projektledaren listar projektets alla tidrapporter och sorterar dem i stigande ordning efter vecka}

\emph{Starttillstånd:} Projektledare P är inloggad och på sidan ''alternativ'' se figur 7 ref 2. Två osignerade och tre signerade tidrapporter med unika upphovsmän finns i systemet under P:s projektgrupp.

\emph{Sluttillstånd:} Projektet alla tidrapporter är sorterade i stigande ordning efter vecka.

\begin{enumerate}
\item L väljer ''redigera tidrapport''.
\item L sorterar i stigande ordning efter vecka.
\item Kontrollera att 5 tidrapporter är sorterade i stigande ordning efter vecka.
\end{enumerate}

%kontrollläst från FT 4.1.1 till hit 2/10 

\item %FT 4.2.7 
\textbf{Projektledaren listar projektets alla tidrapporter och sorterar dom i fallande ordning efter vecka} 

\emph{Starttillstånd:} Projektledare P är inloggad och på sidan ''alternativ'' se figur 7 ref 2. Två osignerade och tre signerade tidrapporter med unika upphovsmän finns i systemet under P:s projektgrupp.

\emph{Sluttillstånd:} Projektet alla tidrapporter är sorterade i fallande ordning efter vecka.

\begin{enumerate}
\item L väljer ''redigera tidrapport''.
\item L sorterar i fallande ordning efter vecka.
\item Kontrollera att 5 tidrapporter är sorterade i fallande ordning efter vecka.
\end{enumerate}

\item %FT 4.2.8 
\textbf{Projektledaren listar projektets alla tidrapporter och sorterar dom i stigande ordning efter hurvida rapporten är godkänd eller ej}

\emph{Start tillstånd:} Projektledare L är inloggad och på sidan ''alternativ'' se figur 7 ref 2. Två osignerade och tre signerade tidrapporter med unika upphovsmän finns i systemet under P:s projektgrupp.

\emph{Slut tillstånd:} Projektets alla tidrapporter är sorterade i stigande ordning efter hurvida rapporten är signerad eller ej.

\begin{enumerate}
\item L väljer ''redigera tidrapport''.
\item L sorterar i stigande ordning efter hurvida rapporten är signerad eller ej.
\item Kontrollera att alla projektets tidrapporter är sorterade i stigande ordning efter hurvida rapporten är signerad eller ej.
\end{enumerate}

\item %FT 4.2.9 
\textbf{Projektledaren listar projektets alla tidrapporter och sorterar dom i fallande ordning efter hurvida rapporten är godkänd eller ej}

\emph{Start tillstånd:} Projektledare L är inloggad och på sidan ''alternativ'' se figur 7 ref 2. Två osignerade och tre signerade tidrapporter med unika upphovsmän finns i systemet under P:s projektgrupp.

\emph{Slut tillstånd:} Projektets alla tidrapporter är sorterade i fallande ordning efter hurvida rapporten är signerad eller ej.

\begin{enumerate}
\item L väljer ''redigera tidrapport''.
\item L sorterar i fallande ordning efter hurvida rapporten är signerad eller ej.
\item Kontrollera att alla projektets tidrapporter är sorterade i fallande ordning efter hurvida rapporten är signerad eller ej.
\end{enumerate}

\end{FT}

\subsubsection{Administratör}
\begin{FT}
%FT 4.3.1
\item
\textbf{Administratören tilldelar en roll till en projektmedlem.}

\emph{Starttillstånd:} Administratör A är inloggad och på sidan ''välj administrationsverktyg'' se figure 5 i ref 2, det finns en projektmedlem M.

\emph{Sluttillstånd:} A har tilldelat roll ''t1'' till M.

\begin{enumerate}
\item A klickar på ''redigera projektmedlemmar'' se figur 5 i ref 2.
\item A väljer M.
\item A tilldela roll ''t1'' till M.
\item A får bekräftelse att M har fått roll som ''t1''.
\item Kontrollera att M har rollen ''t1'' genom att gå til sidan ''Lista projektgrupper och användare'' se figur 5 i ref 2.
\end{enumerate}

\item %FT 4.3.2 
\textbf{Administratören listar alla osignerade tidrapporter}

\emph{Starttillstånd:} Administratör A är inloggat och är inne på ''välj funktion'' se figur 7 ref 2.

\emph{Sluttillstånd:} Alla ej godkända tidrapporter är listade.

\begin{enumerate}
\item A klickar på ''tidrapportering'' se figur 7 ref 2.
\item A väljer ''redigera tidrapport'' se figur 7 ref 2.
\item A väljer att visa osignerade rapporter.
\item Kontrollera att alla osignerade tidrapporter är listade.
\end{enumerate}

\item %FT 4.3.3 
\textbf{Administratören listar alla signerade tidrapporter}

\emph{Starttillstånd:} Administratör A är inloggat och är inne på ''välj funktion'' se figur 7 ref 2.

\emph{Sluttillstånd:} Alla godkända tidrapporter är listade.

\begin{enumerate}
\item A klickar på ''tidrapportering'' se figur 7 ref 2.
\item A väljer ''redigera tidrapport'' se figur 7 ref 2.
\item A väljer att visa signerade rapporter.
\item Kontrollera att alla signerade tidrapporter är listade.
\end{enumerate}

\item %FT 4.3.4 
\textbf{Administratören listar ett projekts alla tidrapporter och sorterar dom i stigande ordning efter användare}

\emph{Starttillstånd:} Administratör A är inloggad och på sidan ''alternativ'' se figur 7 ref 2. Två osignerade och tre signerade tidrapporter med unika upphovsmän finns i systemet.

\emph{Sluttillstånd:} Alla tidrapporter är sorterade i stigande ordning efter användare.

\begin{enumerate}
\item A väljer ''redigera tidrapport''. se figur 7 i ref 2.
\item A sorterar i stigande ordning efter användare.
\item Kontrollera att alla projektets tidrapporter är sorterade i stigande ordning efter användare
\end{enumerate}

\item %FT 4.3.5 
\textbf{Administratören listar ett projekts alla tidrapporter och sorterar dom i fallande ordning efter användare}

\emph{Starttillstånd:} Administratör A är inloggad och på sidan ''alternativ'' se figur 7 i ref 2. Två osignerade och tre signerade tidrapporter med unika upphovsmän finns i systemet.

\emph{Sluttillstånd:} Alla tidrapporter är sorterade i fallande ordning efter användare.

\begin{enumerate}
\item A väljer ''redigera tidrapport''. se figur 7 i ref 2.
\item A sorterar i fallande ordning efter användare.
\item Kontrollera att alla projektets tidrapporter är sorterade i fallande ordning efter användare
\end{enumerate}

\item %FT 4.3.6 
\textbf{Administratören listar ett projekts alla tidrapporter och sorterar dom i stigande ordning efter vecka}

\emph{Starttillstånd:} Administratör A är inloggad och på sidan ''alternativ'' se figur 7 ref 2. Två osignerade och tre signerade tidrapporter med unika upphovsmän finns i systemet.

\emph{Sluttillstånd:} Alla tidrapporter är sorterade i stigande ordning efter vecka.

\begin{enumerate}
\item A väljer ''redigera tidrapport''. se figur 7 i ref 2.
\item A sorterar i stigande ordning efter vecka.
\item Kontrollera att alla projektets tidrapporter är sorterade i stigande ordning efter vecka
\end{enumerate}

\item %FT 4.3.7 
\textbf{Administratören listar ett projekts alla tidrapporter och sorterar dom i fallande ordning efter vecka}

\emph{Starttillstånd:} Administratör A är inloggad och på sidan ''alternativ'' se figur 7 i ref 2. Två osignerade och tre signerade tidrapporter med unika upphovsmän finns i systemet.

\emph{Sluttillstånd:} Alla tidrapporter är sorterade i fallande ordning efter vecka.

\begin{enumerate}
\item A väljer ''redigera tidrapport'' se figur 7 i ref 2.
\item A sorterar i fallande ordning efter vecka.
\item Kontrollera att alla projektets tidrapporter är sorterade i fallande ordning efter vecka
\end{enumerate}

\item %FT 4.3.8 
\textbf{Administratören listar ett projekts alla tidrapporter och sorterar dom i stigande ordning efter hurvida rapporten är signerad eller osignerad}

\emph{Starttillstånd:} Administratör A är inloggad och på sidan ''alternativ'' se figur 7 i ref 2. Två osignerade och tre signerade tidrapporter med unika upphovsmän finns i systemet.

\emph{Sluttillstånd:} Alla tidrapporter är sorterade i stigande ordning efter om de är signerade eller osignerade.

\begin{enumerate}
\item A väljer ''redigera tidrapport'' se figur 7 i ref 2.
\item A sorterar i stigande ordning efter om de är signerade eller osignerade.
\item Kontrollera att alla projektets tidrapporter är sorterade i stigande ordning efter om de är signerade eller osignerade.
\end{enumerate}

\item %FT 4.3.9 
\textbf{Administratören listar ett projekts alla tidrapporter och sorterar dom i fallande ordning efter hurvida rapporten är signerad eller ej}

\emph{Starttillstånd:} Administratör A är inloggad och på sidan ''alternativ'' se figur 7 i ref 2. Två osignerade och tre signerade tidrapporter med unika upphovsmän finns i systemet.

\emph{Sluttillstånd:} Alla tidrapporter är sorterade i fallande ordning efter om de är signerade eller osignerade.

\begin{enumerate}
\item A väljer ''redigera tidrapport'' se figur 7 i ref 2.
\item A sorterar i fallande ordning efter om de är signerade eller osignerade.
\item Kontrollera att alla projektets tidrapporter är sorterade i fallande ordning efter om de är signerade eller osignerade.
\end{enumerate}

\item %FT 4.3.10 
\textbf{Administratören skapar en projektgrupp}

\emph{Starttillstånd:} Administratör A är inloggad och på sida ''välj administrationsverktyg'' se figur 5 i ref 2.

\emph{Sluttillstånd:} Administratör A har skapat projektgrupp ''grupp''.

\begin{enumerate}
\item A väljer ''Projektgrupper'' sedan ''skapa projektgrupp''.
\item A ger gruppen namnet ''grupp''
\item A får bekräftelse att projektgrupp ''grupp'' har skapats.
\item Kontrollera att projektgrupp ''grupp'' existerar i systemet genom att gå till sidan ''Lista projektgrupper och användare'' se figur 5 i ref 2.
\end{enumerate}

\item %FT 4.3.11 
\textbf{En vanlig användare försöker skapa en projektgrupp}

\emph{Starttillstånd:} Användare U är inloggat och på sidan ''välj funktion'' se figur 3 i ref 2.

\emph{Sluttillstånd:} U kunde inte skappa projektgrupp ''grupp''.

\begin{enumerate}
\item U försöker nå ''välj administrationsverktyg'' se figur 5 i ref 2.
\item U kan inte nå ''välj administrationsverktyg''.
\item U ges ett meddelande om att den saknar administrativa rättigheter.
\item Kontrollera att U kan inte nå ''välj administrationsverktyg''.
\end{enumerate}

\item %FT 4.3.12 
\textbf{En projektledare försöker skapa en projektgrupp}

\emph{Starttillstånd:} projektledare L är inloggat och på sidan ''välj funktion'' se figur 4 i ref 2.

\emph{Sluttillstånd:} L kunde inte skappa projektgrupp ''grupp''.

\begin{enumerate}
\item L försöker nå ''välj administrationsverktyg'' se figur 5 i ref 2.
\item L kan inte nå ''välj administrationsverktyg''.
\item L ges ett meddelande om att den saknar administrativa rättigheter.
\item Kontrollera att L kan inte nå ''välj administrationsverktyg''.
\end{enumerate}

\item %FT 4.3.13 
\textbf{Administratören lägger till en projektledare i en projektgrupp}

\emph{Starttillstånd:} Administratör A är inloggat, på sidan ''Lista med projektgrupper och användare'' se figur 5 i ref 2. Projektgrupp G finns och innehåller endast en projektledaren L1.

\emph{Sluttillstånd:}G har projektledare L1 och L2.

\begin{enumerate}
\item A lägger till L2 i G.
\item A utser L2 till projektledare i G.
\item Kontrollera att L2 är projektledare i G genom att gå till sidan ''Lista med projektgrupper och användare'' se figur 5 i ref 2.
\end{enumerate}

\item %FT 4.3.14
\textbf{Administratören tar bort en projektledare i en projektgrupp}

\emph{Starttillstånd:} Administratör A är inloggat, på sidan ''Lista med projektgrupper och användare'' se figur 5 i ref 2. Projektgrupp G finns och innehåller två projektledare L1 och L2.

\emph{Sluttillstånd:} G har projektledare L1.

\begin{enumerate}
\item A väljer projektgrupp G.
\item A tar bort L2 från G.
\item Kontrollera att L1 är ensam projektledare i G genom att gå till sidan ''Lista med projektgrupper och användare'' se figur 5 i ref 2.
\end{enumerate}

\item %FT 4.3.15
\textbf{En vanlig användare försöker lägga till en projektledare i en projektgrupp} 

\emph{Starttillstånd:} Användare U är inloggad och inne på sidan ''välj funktion'' se figur 3 i ref 2, Projektgrupp G finns, med endast en projektledare.

\emph{Sluttillstånd:} Användare U kunde inte lägga in projektledare L i projektgrupp G.

\begin{enumerate}
\item U försöker nå URL för sidan ''Lista med projektgrupper och användare'' se figur 5 i ref 2.
\item U får ett felmeddelande att den inte har rättigheter för den sidan. 
\item Kontrollera att G inte har ny projektledare L genom att gå till sidan ''Lista med projektgrupper och användare'' se figur 5 i ref 2 som en administratör.
\end{enumerate}

\item %FT 4.3.16
\textbf{En vanlig användare försöker ta bort en projektledare i en projektgrupp}

\emph{Starttillstånd:} Användare U är inloggad och inne på sidan ''välj funktion'' se figur 3 i ref 2, Projektgrupp G finns, med projektledare L1 och L2.

\emph{Sluttillstånd:} Användare U kunde inte ta bort L från projektgrupp G.

\begin{enumerate}
\item U försöker nå URL för sidan ''Lista med projektgrupper och användare'' se figur 5 i ref 2.
\item U får ett felmeddelande att den inte har rättigheter för den sidan. 
\item Kontrollera att G fortfarande har projektledare L1 och L2 genom att gå till sidan ''Lista med projektgrupper och användare'' se figur 5 i ref 2 som en administratör.
\end{enumerate}

\item %FT 4.3.17 
\textbf{En projektledare försöker ta bort en projektledare i en projektgrupp}

\emph{Starttillstånd:} Projektledare L1 är inloggad och på sidan för ''välj funktion'' se figur 7 i ref 2. Det finns en projektgrupp G med projektledare L1, L2.

\emph{Sluttillstånd:} G har fortfarande projektledare L1 och L2.

\begin{enumerate}
\item L1 försöker navigera till URL för ''Lista med projektgrupper och användare'' se figur 5 i ref 2.
\item L1 får ett felmeddelande om att den inte har rättigheter för den sidan.
\item Kontrollera att G fortfarande har projektledare L1 och L2 genom att gå till sidan ''Lista med projektgrupper och användare'' se figur 5 i ref 2 som en administratör.
\end{enumerate}


\item %FT 4.3.18 
\textbf{Administratören tar bort en projektgrupp}

\emph{Starttillstånd:} Administratör A är inloggat, det finns en projektgrupp G.

\emph{Sluttillstånd:} G finns inte längre i systemet.

\begin{enumerate}
\item A går till URL för sidan ''Projektgrupper'' se figur 5 i ref 2.
\item A får en bekräftelseruta se krav 6.1.14 i ref 2, och klickar ''ja''.
\item Kontrollera att G inte längre finns i databasen genom att gå till sidan ''Lista med projektgrupper och användare'' se figur 5 i ref 2.
\end{enumerate}

\item %FT 4.3.19
\textbf{En vanlig användare försöker ta bort en projektgrupp}

\emph{Starttillstånd:} Projektmedlem M är inloggad, det finns en projektgrupp G.

\emph{Sluttillstånd:} G finns kvar i databasen.

\begin{enumerate}
\item M försöker nå URL för sidan ''Projektgrupper'' se figur 5 i ref 2.
\item M får ett meddelande om att M inte har behörighet för den sidan.
\item Kontrollera att M kan inte nå sidan.
\item Kontrollera att projektgrupp G finns kvar i databasen genom att gå till sidan ''Lista med projektgrupper och användare'' se figur 5 i ref 2 som en administratör.
\end{enumerate}

\item %FT 4.3.20
\textbf{En projektledare försöker ta bort en projektgrupp}

\emph{Starttillstånd:} Projektledare L är inloggad och det finns en projektgrupp G.

\emph{Sluttillstånd:} G finns kvar i databasen.

\begin{enumerate}
\item L försöker gå till URL för sidan ''Projektgrupper'' se figur 5 i ref 2.
\item A får ett meddelande om att A inte har behörighet för den sidan.
\item Kontrollera att A kan inte nå sidan.
\item Kontrollera att G finns kvar i databasen genom att gå till sidan ''Lista med projektgrupper och användare'' se figur 5 i ref 2 som en administratör.
\end{enumerate}

\item
\textbf{Administratören kan ta bort en projektgrupp med medlemmar i}

\emph{Starttillstånd:} Administratören A är inloggad och det finns en projektgrupp G med projektledare L och medlemmar M1 och M2.

\emph{Sluttilstånd:} G finns inte i databasen längre.

\begin{enumerate}
\item A går till URL för sidan ''Projektgrupper'' se figur 5 i ref 2. 
\item A väljer att ta bort G och klickar ''ja'' på den dialogruta som definieras i krav  6.1.14 i ref 2.
\item Kontrollera att G inte längre finns i databasen genom att gå till sidan ''Lista med projektgrupper och användare'' se figur 5 i ref 2.
\end{enumerate}

\item %FT 4.3.22
\textbf{På sidan ''Lista användare med lösenord'' är alla användare listade med både användarnamn och lösenord}

\emph{Starttillstånd:} Administratör A är inloggad. I systemet finns det användare U1, U2 och projektmedlemmar M1, M2 och M3.

\emph{Sluttillstånd:} A kan se alla användare och respektive användares lösenord.

\begin{enumerate}
\item A går till URL för ''Lista användare och lösenord''. 
\item Kontrollera att A har tillgång till U1, U2, M1, M2 och M3:s användarnamn och lösenord i en lista.
\end{enumerate}

\item %FT 4.3.23 
\textbf{På administrationssidan kan man ta bort en vanlig användare}

\emph{Starttillstånd:} Administratör A är inloggad, det finns en användare U i systemet.

\emph{Sluttillstånd:} U finns inte längre i systemet.

\begin{enumerate}
\item A går till URL för ''Lista användare och lösenord'' se figur 5 i ref 2. 
\item A väljer att ta bort U, och klickar ''ja'' på den dialogruta som definieras i krav 6.1.14 i ref 2.
\item Kontrollera att U inte finns i databasen längre genom att gå till sida ''Lista användare med lösenord'' se figur 5 i ref 2.
\end{enumerate}

\item %FT 4.3.23 
\textbf{På administrationssidan kan man ta bort en projektledare}

\emph{Starttillstånd:} Administratör A är inloggad, och det finns en projektledare L i systemet.

\emph{Sluttillstånd:} L finns inte längre i systemet.

\begin{enumerate}
\item A går till URL för ''Lista användare och lösenord'' se figur 5 i ref 2.
\item A väljer att ta bort U, och klickar ''ja'' på den dialogruta som definieras i krav 6.1.14 i ref 2.
\item Kontrollera att L inte längre finns i databasen genom att gå till sida ''Lista användare med lösenord'' se figur 5 i ref 2..
\end{enumerate}

\item %FT 4.3.24
\textbf{En administratör kan inte ta bort en administratör}

\emph{Starttillstånd:} Administratör A är inloggad.

\emph{Sluttillstånd:} Administratör A finns kvar i systemet.

\begin{enumerate}
\item A går till URL för ''Lista användare och lösenord'' se figur 5 i ref 2.
\item A väljer att ta bort U, och klickar ''ja'' på den dialogruta som definieras i krav 6.1.14 i ref 2.
\item A får ett felmeddelande om att den inte kan ta bort en administratör.
\item Kontrollera att A finns kvar i databasen genom att gå in på ''Lista användare och lösenord igen.
\end{enumerate}

\item %FT 4.3.25 
\textbf{På administrationssidan kan man lägga till en ny användare}

\emph{Starttillstånd:} Administratör A är inloggad, det finns ingen användare ''Kurtan'' i systemet.

\emph{Sluttillstånd:} Det finns en användare ''Kurtan'' i systemet.

\begin{enumerate}
\item A går till URL för ''Lista användare och lösenord'' se figur 5 i ref 2.
\item  A väljer att lägga till användare och anger namnet ''Kurtan''.
\item Kontrollera genom att gå till ''Lista användare och lösenord'' att ''Kurtan'' finns i databasen.
\end{enumerate}

\item %FT 4.3.26
\textbf{Administratören skapar en användare och skriver in användarnamn, användarens lösenord genereras slumpmässigt}

\emph{Starttillstånd:} Administratör A är inloggad, det finns ingen användare ''Kurtan'' i systemet.

\emph{Sluttillstånd:} ''Kurtan'' finns i systemet med ett unikt lösenord.

\begin{enumerate}
\item A går till URL för ''Lista användare och lösenord'' se figur 5 i ref 2.
\item  A väljer att lägga till användare och anger namnet ''Kurtan''.
\item Kontrollera genom att gå till ''Lista användare och lösenord'' att ''Kurtan'' finns i databasen.
\item Kontrollera att ''Kurtan'' har ett unikt lösenord som inte angivits av A.
\end{enumerate}

\item %FT 4.3.27 
\textbf{Administratören försöker skapa en användare med ett upptaget användarnamn, ett felmeddelande visas}

\emph{Starttillstånd:} Administratör A är inloggad, det finns en användare ''Kurtan'' i systemet med lösenord x.

\emph{Sluttillstånd:} Det finns bara en användare ''Kurtan'' i systemet, med oförändrat lösenord x.

\begin{enumerate}
\item A går till URL för ''Lista användare och lösenord'' se figur 5 i ref 2.
\item  A väljer att lägga till användare och anger namnet ''Kurtan''.
\item A får ett felmeddelande om att namnet ''Kurtan'' redan är upptaget.
\item Kontrollera att det endast finns en användare ''Kurtan'' i databasen genom att gå till sidan ''Lista användare och lösenord''.
\item Kontrollera att det lösenord som ''Kurtan'' har är oförändrat mot innan A försökte lägga in ''Kurtan'' i systemet.
\end{enumerate}

\item %FT 4.3.28
\textbf{Administratören försöker skapa en användare med ett ogiltigt användarnamn} 

\emph{Starttillstånd:} Administratör A är inloggad.

\emph{Sluttillstånd:} Oförändrat.

\begin{enumerate}
\item A går till URL för ''Lista användare och lösenord'' se figur 5 i ref 2.
\item A väljer lägg till användare och anger namnet ''Kurtan£''.
\item A får ett felmeddelande om att ogiltigt namn angivits.
\item Kontrollera att ''Kurtan£'' inte finns i systemet genom att gå till sidan ''Lista användare och lösenord''.
\item A väljer lägg till användare och anger namnet ''Kurt''.
\item A får ett felmeddelande om att ogiltigt namn angivits.
\item Kontrollera att ''Kurt'' inte finns i systemet genom att gå till sidan ''Lista användare och lösenord''.
\end{enumerate}

\item %FT 4.3.29 
\textbf{Administratören skapar en användare med ett giltigt användarnamn}
\begin{enumerate}
\item Gör test beskrivet i FT 4.3.26
\end{enumerate}

\item %FT 4.3.30 
\textbf{Administratören skapar en tidrapportmall}

\emph{Utgår}
%\emph{Starttillstånd:} Administratör A är inloggat.

%\emph{Sluttillstånd:} Administratör A har skapat tidrapportmall.

%\begin{enumerate}
%\item Administratör A skapar ny tidrapportmall.
%\item Administratör A får bekräftelse att ny tidrappormall har sakpat.
%\end{enumerate}

\item %FT 4.3.31 
\textbf{Administratören försöker ändra en tidrapportmall som en projektgrupp använder}

\emph{Utgår}
%\emph{Starttillstånd:} Administratör A är inloggat.

%\emph{Sluttillstånd:} Administratör A kunde inte ändra  tidrapportmall som projektgruppB använder.

%\begin{enumerate}
%\item Administratör A försäker ändra tidrapportmall som projektgrupp B använder.
%\item Administratör A får bekräftelse att  den kan inte användas pga projektgrupp B använder tidrappormall. 
%\end{enumerate}

\end{FT}
\subsubsection{Data}

%FT 4.4.1 
\begin{FT}
\item
\textbf{Administratören försöker skapa en projektgrupp med ett ogiltigt projektgruppsnamn (gil- tigt projektgruppsnamn: 5-10 tecken, ascii 48-57 och 97-122)}

\emph{Starttillstånd} Administratör A  är inloggad. En användare U finns i databasen.

\emph{Sluttillstånd:} Oförändrat.

\begin{enumerate}
\item A navigerar till administrationsvyn och sedan till projektgrupper (Figur 5, Ref. 2).
\item A försöker skapa en projektgrupp med namnet ''brie'' och med U som projektledare.
\item Kontrollera att ett felmeddelande visas som informerar om att namnet var för kort.
\item A försöker skapa en projektgrupp med namnet ''creamcheese''.
\item Kontrollera att ett felmeddelande visas som informerar om att namnet var för långt.
\item A försöker skapa en projektgrupp med namnet ''cheddar?'' med U som projektledare.
\item Kontrollera att ett felmeddelande visas som informerar om att namnet innehöll ett olämpligt tecken.
\item Kontrollera att det inte finns några projektgrupper i databasen.

\end{enumerate}

%FT 4.4.2 
\item
\textbf{Administratören försöker skapa en projektgrupp med ett giltigt projektgruppsnamn(giltigt projektgruppsnamn: 5-10 tecken, ascii 48-57 och 97-122)}

\emph{Starttillstånd} Administratör A  är inloggat. En användare U finns i databasen.

\emph{Sluttillstånd:} U och två projektgrupper med namnen ''gouda'' och ''mozzarella'' finns i databasen.

\begin{enumerate}
\item A navigerar till administrationsvyn och sedan till projektgrupper (Figur 5, Ref. 2).
\item A skapar en projektgrupp med namnet ''gouda'' med U som projektledare.
\item A navigerar till projektgrupper.
\item A skapar en projektgrupp med namnet ''mozzarella'' med U som projektledare.
\item Kontrollera att det finns två projektgrupper i databsen och att de har namnen ''gouda'' respektive ''mozzarella''.
\end{enumerate}

%FT 4.4.3 
\item
\textbf{Administratören försöker skapa en projektgrupp med ett upptaget projektgruppsnamn}

\emph{Starttillstånd} Administratör A är inloggat. En användare U finns i systemet och en projektgrupp med namnet ''getost'' finns i systemet.

\emph{Sluttillstånd:} Oförändrat.

\begin{enumerate}
\item A navigerar till administrationsvyn och sedan till projektgrupper (Figur 5, Ref. 2).
\item A försöker skapa en projektgrupp med namnet ''getost'' med U som projektledare.
\item Kontrollera att ett felmeddelande visas som informerar om att namnet var upptaget.
\item Kontrollera att det bara finns en projektgrupp i databasen.
\end{enumerate}

%FT 4.4.4 
\item
\textbf{Administratören skapar 5 stycken projektgrupper}

\emph{Starttillstånd} Administratör A är inloggad. En användare U finns i systemet. 

\emph{Sluttillstånd:} U och fem stycken projektgrupper finns i systemet.

\begin{enumerate}
\item A navigerar till administrationsvyn och sedan till projektgrupper (Figur 5, Ref. 2).
\item A skapar en ny projektgrupp med namnet ''grupp1'' med U som projektledare.
\item A navigerar till projektgrupper.
\item A skapar en ny projektgrupp med namnet ''grupp2'' med U som projektledare.
\item A navigerar till projektgrupper.
\item A skapar en ny projektgrupp med namnet ''grupp3'' med U som projektledare.
\item A navigerar till projektgrupper.
\item A skapar en ny projektgrupp med namnet ''grupp4'' med U som projektledare.
\item A navigerar till projektgrupper.
\item A skapar en ny projektgrupp med namnet ''grupp5'' med U som projektledare.
\item Kontrollera att det finns fem stycken projektgrupper i databasen.
\end{enumerate}

%FT 4.4.5 
\item
\textbf{Administratören försöker skapa 6 stycken projektgrupper}

\emph{Starttillstånd} Administratör A är inloggat. En användare U finns i systemet.

\emph{Sluttillstånd:} U och fem projektgrupper finns i systemet.

\begin{enumerate}
\item A navigerar till administrationsvyn och sedan till projektgrupper (Figur 5, Ref. 2).
\item A skapar en ny projektgrupp med namnet ''grupp1'' med U som projektledare.
\item A navigerar till projektgrupper.
\item A skapar en ny projektgrupp med namnet ''grupp2'' med U som projektledare.
\item A navigerar till projektgrupper.
\item A skapar en ny projektgrupp med namnet ''grupp3'' med U som projektledare.
\item A navigerar till projektgrupper.
\item A skapar en ny projektgrupp med namnet ''grupp4'' med U som projektledare.
\item A navigerar till projektgrupper.
\item A skapar en ny projektgrupp med namnet ''grupp5'' med U som projektledare.
\item A navigerar till projektgrupper.
\item A skapar en ny projektgrupp med namnet ''grupp6'' med U som projektledare.
\item Kontrollera att ett felmeddelande visas som informerar om att det inte går att lägga till fler projektgrupper.
\item Kontrollera att det finns exakt fem stycken projektgrupper i databasen.
\item Kontrollera att det inte finns en projektgrupp med namnet ''grupp6'' i databasen.
\end{enumerate}

%FT 4.4.6 
\item
\textbf{Administratören lägger till samma användare i flera projektgrupper}

\emph{Starttillstånd} Administratör A är inloggad. Två användare U1, U2 finns i systemet. Två projektgrupper G1,G2 finns i systemet. U1 är projektledare i G1,G2.

\emph{Sluttillstånd:} Två användare U1, U2 finns i systemet. Två projektgrupper G1,G2 finns i systemet. U1 är projektledare i G1,G2. U2 är medlem i G1 och G2.

\begin{enumerate}
\item A navigerar till administrationsvyn och sedan till projektgrupper (Figur 5, Ref. 2).
\item A lägger till U2 i G1.
\item A lägger till U2 i G2.
\item Kontrollera att U2 är medlem i G1 och G2 i databasen.

\end{enumerate}

%FT 4.4.7 
\item
\textbf{Administratören försöker skapa en projektgrupp utan användare}

\emph{Starttillstånd} Administratör A är inloggad.

\emph{Sluttillstånd:} Oförändrad.

\begin{enumerate}

\item A navigerar till administrationsvyn och sedan till projektgrupper (Figur 5, Ref. 2).
\item A försöker skapa en projektgrupp med namnet ''edamer'' utan projektledare.
\item Kontrollera att ett felmeddelande visas som informerar om att det inte går att skapa en projektgrupp utan projektledare.
\item Kontrollera att det inte finns några projektgrupper i databasen.
\end{enumerate}

%FT 4.4.8 
\item
\textbf{Administratören försöker att från en projektgrupp som har en (1) användare, ta bort en användare.}

\emph{Starttillstånd} Administratör A är inloggad. En projektgrupp G och en användare U finns i systemet. U är medlem i G.

\emph{Sluttillstånd:} Oförändrat.

\begin{enumerate}
\item A navigerar till administrationsvyn och sedan till projektgrupper (Figur 5, Ref. 2).
\item A försöker ta bort C ur projektgrupp B
\item Kontrollera att ett felmeddelande visas som informerar om att det inte går att ta bort U ur G.
\item Kontrollera att U är medlem i G.
\end{enumerate}

%FT 4.4.9 
\item
\textbf{Administratören skapar en projektgrupp med en användare}

\emph{Starttillstånd} Administratör A är inloggad. En användare U finns i systemet. 

\emph{Sluttillstånd:} U och en projektgrupp G finns i systemet.

\begin{enumerate}
\item A navigerar till administrationsvyn och sedan till projektgrupper (Figur 5, Ref. 2).
\item A skapar projektgrupp en projektgrupp G med U som projektledare.
\item Kontrollera att G finns i databasen.
\item Kontrollera att U är projektledare i G.
\end{enumerate}

%FT 4.4.10 
\item
\textbf{Administratören lägger till en användare i en projektgrupp så att projektgruppen har 20 användare.}

\emph{Starttillstånd} Administratör A är inloggad. Användare U1-u20 finns i systemet. En projektgrupp G finns i systemet. U1-U19 är medlemmar i G.

\emph{Sluttillstånd:} Användare U1-U20 finns i systemet. En projektgrupp G finns i systemet. U1-U20 är medlemmar i G.

\begin{enumerate}
\item A navigerar till administrationsvyn och sedan till projektgrupper (Figur 5, Ref. 2).
\item A lägger till U20 som medlem i G.
\item Kontrollera att G har 20 medlemmar.
\item Kontrollera att U20 är medlem i G.
\end{enumerate}

%FT 4.4.11 
\item
\textbf{Administratören försöker lägga till användare i en projektgrupp så att projektgruppen har 21 användare.}

\emph{Starttillstånd} Administratör A är inloggad. En projektgrupp G och användare U1-U21 finns i systemet. U1-U20 är medlemmar i G.

\emph{Sluttillstånd:} Oförändrat.

\begin{enumerate}
\item A navigerar till administrationsvyn och sedan till projektgrupper (Figur 5, Ref. 2).
\item A lägger till U21 som medlem i G.
\item Kontrollera att ett felmeddelande visas som informerar om att det inte går att lägga till fler medlemmar i G.
\item Kontrollera att G har 20 medlemmar.
\item Kontrollera att U21 inte är medlem i G.
\end{enumerate}

%FT 4.4.12
\item 
\textbf{Projektledaren tilldelar tre olika roller till projektmedlemmar i sitt projekt}

\emph{Starttillstånd} Projektledare A är inloggad. En projektgrupp G och tre användare U1,U2,U3 finns i systemet. U1,U2, U3 är medlemmar i G.

\emph{Sluttillstånd:} En projektgrupp G och tre användare U1,U2,U3 finns i systemet. U1,U2, U3 är medlemmar i G. U1 har rollen ''t1', U2 har rollen ''t2'' och U3 har rollen ''t3''.

\begin{enumerate}
\item A navigerar till administrationsvyn och sedan till redigera projektmedlemmar (Figur 5, Ref. 2).
\item A tilldelar rollen ''t1'' till U1.
\item A tilldelar rollen ''t2'' till U2.
\item A tilldelar rollen ''t3'' till U3.
\item Kontrollera att U1 har rollen ”t1” i G.
\item Kontrollera att U2 har rollen ”t1” i G.
\item Kontrollera att U3 har rollen ”t3” i G.
\end{enumerate}

%FT 4.4.13 
\item
\textbf{Projektledaren försöker tilldela fyra olika roller till projektmedlemmar i sitt projekt}

\emph{Starttillstånd} Projektledare A är inloggad. En projektgrupp G och tre användare U1,U2,U3,U4 finns i systemet. U1,U2,U3,U4 är medlemmar i G.

\emph{Sluttillstånd:} En projektgrupp G och tre användare U1,U2,U3,U4 finns i systemet. U1,U2, U3,U4 är medlemmar i G. U1 har rollen ''t1', U2 har rollen ''t2'' och U3 har rollen ''t3''.

\begin{enumerate}
\item A navigerar till administrationsvyn och sedan till redigera projektmedlemmar (Figur 5, Ref. 2).
\item A tilldelar rollen ''t1'' till U1.
\item A tilldelar rollen ''t2'' till U2.
\item A tilldelar rollen ''t3'' till U3.
\item Kontrollera att det inte går att tilldela rollen ''t4'' till U4.
\item Kontrollera att U1 har rollen ”t1” i G.
\item Kontrollera att U2 har rollen ”t1” i G.
\item Kontrollera att U3 har rollen ”t3” i G.
\end{enumerate}

%FT 4.4.14 
\item
\textbf{Projektledaren tilldelar 6 projektmedlemmar i sitt projekt rollen ''t1''}

\emph{Starttillstånd} Projektledare A är inloggad. Användare U1-U6 och en projektgrupp G finns i systemet. U1-U6 är medlemmar i G.

\emph{Sluttillstånd:} Projektledare A är inloggad. Användare U1-U6 och en projektgrupp G finns i systemet. U1-U6 är medlemmar i G. U1-U6 har rollen ''t1''.

\begin{enumerate}
\item A navigerar till administrationsvyn och sedan till redigera projektmedlemmar (Figur 5, Ref. 2).
\item A tilldelar rollen ''t1'' till U1.
\item A tilldelar rollen ''t1'' till U2.
\item A tilldelar rollen ''t1'' till U3.
\item A tilldelar rollen ''t1'' till U4.
\item A tilldelar rollen ''t1'' till U5.
\item A tilldelar rollen ''t1'' till U6.
\item Kontrollera att U1-U6 har rollen ''t1'' i G.
\end{enumerate}

%FT 4.4.15 
\item
\textbf{Projektledaren tilldelar 6 projektmedlemmar i sitt projekt rollen ''t2''}

\emph{Starttillstånd} Projektledare A är inloggad. Användare U1-U6 och en projektgrupp G finns i systemet. U1-U6 är medlemmar i G.

\emph{Sluttillstånd:} Projektledare A är inloggad. Användare U1-U6 och en projektgrupp G finns i systemet. U1-U6 är medlemmar i G. U1-U6 har rollen ''t2''.

\begin{enumerate}
\item A navigerar till administrationsvyn och sedan till redigera projektmedlemmar (Figur 5, Ref. 2).
\item A tilldelar rollen ''t2'' till U1.
\item A tilldelar rollen ''t2'' till U2.
\item A tilldelar rollen ''t2'' till U3.
\item A tilldelar rollen ''t2'' till U4.
\item A tilldelar rollen ''t2'' till U5.
\item A tilldelar rollen ''t2'' till U6.
\item Kontrollera att U1-U6 har rollen ''t2'' i G.
\end{enumerate}

%FT 4.4.16 
\item
\textbf{Projektledaren tilldelar 6 projektmedlemmar i sitt projekt rollen ''t3''}

\emph{Starttillstånd} Projektledare A är inloggad. Användare U1-U6 och en projektgrupp G finns i systemet. U1-U6 är medlemmar i G.

\emph{Sluttillstånd:} Projektledare A är inloggad. Användare U1-U6 och en projektgrupp G finns i systemet. U1-U6 är medlemmar i G. U1-U6 har rollen ''t3''.

\begin{enumerate}
\item A navigerar till administrationsvyn och sedan till redigera projektmedlemmar (Figur 5, Ref. 2).
\item A tilldelar rollen ''t3'' till U1.
\item A tilldelar rollen ''t3'' till U2.
\item A tilldelar rollen ''t3'' till U3.
\item A tilldelar rollen ''t3'' till U4.
\item A tilldelar rollen ''t3'' till U5.
\item A tilldelar rollen ''t3'' till U6.
\item Kontrollera att U1-U6 har rollen ''t3'' i G.
\end{enumerate}

%FT 4.4.17 
\textbf{Projektledaren försöker tilldela 7 projektmedlemmar i sitt projekt rollen ''t1''}

\emph{Starttillstånd} Projektledare A är inloggad. Användare U1-U7 och en projektgrupp G finns i systemet. U1-U7 är medlemmar i G.

\emph{Sluttillstånd:} Användare U1-U7 och en projektgrupp G finns i systemet. U1-U7 är medlemmar i G. U1-U6 har rollen ''t1''.

\begin{enumerate}
\item A navigerar till administrationsvyn och sedan till redigera projektmedlemmar (Figur 5, Ref. 2).
\item A tilldelar rollen ''t1'' till U1.
\item A tilldelar rollen ''t1'' till U2.
\item A tilldelar rollen ''t1'' till U3.
\item A tilldelar rollen ''t1'' till U4.
\item A tilldelar rollen ''t1'' till U5.
\item A tilldelar rollen ''t1'' till U6.
\item A tilldelar rollen ''t1'' till U7.
\item Kontrollera att ett felmeddelande visas som informerar om att det inte går att ha fler medlemmar med rollen ''t1''.
\item Kontrollera att U1-U6 har rollen ''t1'' i G.
\item Kontrollera att U7 inte har rollen ''t1'' i G.
\end{enumerate}

%FT 4.4.18 
\textbf{Projektledaren försöker tilldela 7 projektmedlemmar i sitt projekt rollen ''t2''}

\emph{Starttillstånd} Projektledare A är inloggad. Användare U1-U7 och en projektgrupp G finns i systemet. U1-U7 är medlemmar i G.

\emph{Sluttillstånd:} Användare U1-U7 och en projektgrupp G finns i systemet. U1-U7 är medlemmar i G. U1-U6 har rollen ''t2''.

\begin{enumerate}
\item A navigerar till administrationsvyn och sedan till redigera projektmedlemmar (Figur 5, Ref. 2).
\item A tilldelar rollen ''t2'' till U1.
\item A tilldelar rollen ''t2'' till U2.
\item A tilldelar rollen ''t2'' till U3.
\item A tilldelar rollen ''t2'' till U4.
\item A tilldelar rollen ''t2'' till U5.
\item A tilldelar rollen ''t2'' till U6.
\item A tilldelar rollen ''t2'' till U7.
\item Kontrollera att ett felmeddelande visas som informerar om att det inte går att ha fler medlemmar med rollen ''t2''.
\item Kontrollera att U1-U6 har rollen ''t2'' i G.
\item Kontrollera att U7 inte har rollen ''t2'' i G.
\end{enumerate}

%FT 4.4.19 
\textbf{Projektledaren försöker tilldela 7 projektmedlemmar i sitt projekt rollen ''t3''}

\emph{Starttillstånd} Projektledare A är inloggad. Användare U1-U7 och en projektgrupp G finns i systemet. U1-U7 är medlemmar i G.

\emph{Sluttillstånd:} Användare U1-U7 och en projektgrupp G finns i systemet. U1-U7 är medlemmar i G. U1-U6 har rollen ''t3''.

\begin{enumerate}
\item A navigerar till administrationsvyn och sedan till redigera projektmedlemmar (Figur 5, Ref. 2).
\item A tilldelar rollen ''t3'' till U1.
\item A tilldelar rollen ''t3'' till U2.
\item A tilldelar rollen ''t3'' till U3.
\item A tilldelar rollen ''t3'' till U4.
\item A tilldelar rollen ''t3'' till U5.
\item A tilldelar rollen ''t3'' till U6.
\item A tilldelar rollen ''t3'' till U7.
\item Kontrollera att ett felmeddelande visas som informerar om att det inte går att ha fler medlemmar med rollen ''t3''.
\item Kontrollera att U1-U6 har rollen ''t3'' i G.
\item Kontrollera att U7 inte har rollen ''t3'' i G.
\end{enumerate}

%FT 4.4.20 
\item
\textbf{Administratören har användarnamnet ''admin'' och lösenordet ''adminpw''}

\emph{Starttillstånd} Administratör A är på inloggningssidan.

\emph{Sluttillstånd:} A är inloggad.

\begin{enumerate}
\item A loggar in med användarenamnet ''admin'' och lösenordet ''adminpw''.
\item A navigerar till administrationvyn (Figur 5, Ref. 2).
\item Kontrollera att A är på administrationsvyn.
\end{enumerate}
\end{FT}

%+++++++++++++++++++++++++++% 
%slutet på FT:Administration%
%+++++++++++++++++++++++++++%




%!!!!!!!!!!!!!
%!!!!!!!!!!!!!
%!!!!!!!!!!!!!
%HÄR BÖRJAR ST
%!!!!!!!!!!!!!
%!!!!!!!!!!!!!
%!!!!!!!!!!!!!

\newpage


\section{Systemtest}
%+++++++++++++++++++++++++++%
%BÖRJAN på ST:generella krav%
%+++++++++++++++++++++++++++%
\subsection{Generella krav}
\invisiblesubsubsection{Användare}
\subsubsection{Projektmedlem}
\begin{ST}
\item \textbf{Projektmedlem använder systemet}

\emph{Starttillstånd:} Projektmedlemmen M är inte inloggad i systemet. M och Projektledaren L är båda medlemmar i projektgruppen G.

\emph{Sluttillstånd:} Projektmedlemmen M är inte inloggad i systemet. M och Projektledaren L är båda medlemmar i projektgruppen G. En osignerad tidsrapport T för projekt G, projektmedlem M och vecka 1 finns i systemet med 20 minuter på möte registrerat. M har lösenordet ``abcdefgh''.

\begin{enumerate}
\item Skriv rätt användarnamn och felaktigt lösenord för M.
\item Klicka på ``Logga in''.
\item Kontrollera att M fortfarande befinner sig på inloggningssidan.
\item Skriv in rätt användarnamn och rätt lösenord för M.
\item Klicka på ``Logga in''.
\item Klicka på ``Tidrapportering''.
\item Klicka på projekt G.
\item Klicka på lägg till tidrapport.
\item Skriv 1 i fältet för vecka.
\item Skriv 30 i fältet för möte.
\item Klicka på ``Lägg till'', tidsrapporten som skapades kallas för T.
\item Klicka på ``Visa'' för tidrapport för vecka 1.
\item Kontrollera att det i fältet ``möte'' står 30.
\item Klicka på ``Tillbaka''.
\item Klicka på ``Redigera rapport'' för tidrapport T.
\item Ändra fältet ``möte'' till 20.
\item Klicka på ``Redigera rapport''.
\item Kontrollera i databasen att det i tidsrapporten T står 20 för möte.
\item Klicka på ``Visa projektinformation''.
\item Kontrollera att projektnamnet för G syns och att M och L står med som medlemmar i projektet.
\item Klicka på ``Ändra lösenord''.
\item Skriv det nya lösenordet ``k''.
\item Kontrollera att du fortfarande befinner dig på samma sida.
\item Kontrollera i databasen att lösenordet inte är ``k''.
\item Skriv in det nya lösenordet ``abcdefgh''.
\item Kontrollera i databasen att lösenordet är ``abcdefgh''.
\item Klicka på ``Logga ut''
\item Kontrollera att du befinner dig på inloggningssidan.

\end{enumerate}

\end{ST}

\subsubsection{Projektledare}
\begin{ST}
\item \textbf{Projektledaren har tillgång till projektadministrationsfunktionaliteter}

\emph{Starttillstånd:} Projektledaren L är projektledare för projektgruppen G och är inloggad och befinner sig på sidan ``Projektgruppsadministration''. M1, M2 och M3 är projektmedlemmar i G, de har rollerna t1, t2 och t3 respektive. M1, M2 och M3 har tidsrapporteringar i vecka 1 T1, T2 och T3 respektive. T1, T2 och T3 innehåller tidrapportering för 10, 20 och 30 minuter, de är alla signerade.

\emph{Sluttillstånd:} Oförändrat.

\begin{enumerate}
\item Klicka på ``Generera statistik'' för G.
\item Kontrollera att statistik visas för G, sammanlagd tid rapporterad skall vara 60 minuter.
\item Gå tillbaka till projektgruppsadmininstrationssidan.
\item Klicka på ``Hantera tidsrapporter''.
\item Avsignera tidsrapport T1.
\item Kontrollera att tidsrapporten T1 inte är signerad i databasen.
\item Signera tidsrapport T1.
\item Kontrollera att tidsrapporten T1 är signerad i databasen.
\item Gå tillbaka till projektgruppsadmininstrationssidan.
\item Klicka på ``Hantera användare'' för projektet G.
\item Kontrollera att alla projektmedlemmar visas.
\item För projektmedlem M1, klicka på ``Byt roll''.
\item Välj t2.
\item Kontrollera i databasen att rollen för M1 i grupp G har ändrats i databasen.
\end{enumerate}
\end{ST}

\subsubsection{Administratör}
\begin{ST}

%\item \textbf{Administratören har tillgång administratörsfunktionaliteter} 
%
%\emph{Starttillstånd:} Administratör Ad är inte inloggad.
%
%\emph{Sluttillstånd:} Administratör Ad är inte inloggad.
%
%\begin{enumerate}
%
%\item Ad skriver URL till inloggningssidan.
%\item Ad skriver felaktigt lösenord, sidan laddas om.
%\item Ad skriver rätt lösenord, inloggad.
%\item Ad väljer administrationsvyn.
%\item Ad väljer funk. lista användare.
%\item Kontrollera att Ad kan se alla användare med lösenord.
%\item Ad lägger till användare, rätt input, lyckas.
%\item Ad försöker lägga till användare, fel input, misslyckas.
%\item Ad tar bort användare, lyckas.
%\item Ad väljer administrationsvyn.
%\item Ad väljer funk. Projektgrupper.
%\item Ad lägger till användare i godtycklig projektgrupp, lyckas.
%\item Ad tar bort användare från projektgrupp, lyckas.
%\item Ad försöker ta bort projektgrupp, finns projektmedlemmar i, misslyckas.
%\item Ad tar bort projektgrupp, tom projektgrupp, lyckas.
%\item Ad skapar projektgrupp.
%\item Ad väljer tidrapportmall.
%\item Ad väljer användare till projektgruppen från en lista.
%\item Ad väljer projektledare, trycker på "Skapa".
%\item Kontrollera att Ad skickas tillbaka till administrationsvyn.
%\item Steg 11-20 upprepas, men i steg 17 skapar Ad en ny tidrapportmall.
%\item Ad väljer funk. Redigera projektmedlemmar.
%\item Kontrollera att samtliga projektmedlemmar listas.
%\item Ad utser andra projektledare, lyckas.
%\item Ad tilldelar roller, lyckas.
%\item Ad byter grupp på användare, lyckas.
%\item Ad väljer väljer administrationsvyn.
%\item Ad väljer funk. Ta bort tidrapportmall.
%\item Kontrollera att tidrapportmallarna listas.
%\item Ad försöker ta bort tidrapportmall som används, misslyckas.
%\item Ad tar bort tidrapportmall som inte används, lyckas.
%\item Ad väljer administrationsvyn.
%\item Ad trycker på "Logga ut".
%\item Kontrollera att Ad är utloggad.
%
%\end{enumerate}

\item \textbf{Administratören använder systemet}

\emph{Starttillstånd:} Administratören A är inte inloggad i systemet och befinner sig på inloggningssidan. De vanliga användarna V1 och V2, finns i systemet. Projektmedlem M och projektledare L finns i systemet, de tillhör båda projektgrupp G1.

\emph{Sluttillstånd:} Administratören A är inte inloggad i systemet och befinner sig på inloggningssidan. De vanliga användarna V1 och V2, finns i systemet. Den vanliga användaren M och projektledare L finns i systemet, endast L tillhör projektgrupp G1.

\begin{enumerate}
\item Skriv in rätt användarnamn för A och fel lösenord.
\item Klicka på ``Logga in''.
\item Kontrollera att du fortfarande befinner dig på inloggningssidan.
\item Skriv in rätt användarnamn för A och rätt lösenord.
\item Klicka på ``Logga in''.
\item Klicka på ``Administrationsvyn''.
\item Klicka på ``Lista användare''.
\item Kontrollera att en lista med användarnamnen och lösenorden för A, V1 och V2 visas.
\item Klicka på ``Lägg till användare''.
\item Skriv in ``Kalle''.
\item Klicka på ``Lägg till'', användaren som skapas refererars till som V3.
\item Kontrollera att en lista med användarnamnen och lösenorden för A, V1, V2 och V3 visas.
\item Klicka på ``Ta bort användare'' brevid V3:s rad i listan.
\item Klicka på ``OK''.
\item Kontrollera att en lista med användarnamnen och lösenorden för A, V1 och V2 visas.
\item Klicka på ``Administrationsvyn''.
\item Klicka på ``Projektgrupper''.
\item Klicka på ``Skapa projektgrupp''.
\item Skriv in projektgruppsnamnet ``grupp123'' och V1s användarnamn som projektledare.
\item Klicka på ``Skapa projektgrupp'', projektgruppen refereras till som G2.
\item Klicka på ``Lägg till användare i projektgrupp''.
\item Välj V2:s användarnamn i listan.
\item Klicka på ``OK''.
\item Klicka på ``Projektgrupper''. %Ta bort projektgrupp
\item Klicka på ``Ta bort användare i projektgrupper''.
\item Klicka på ``Ta bort'' vid V2.
\item Klicka på ``Redigera projektmedlemmar''.
\item Kontrollera att en lista med G1 och G2 visas samt V1, V2, M och L.
\item Klicka på ``Tilldela roll'' för V1, sätt V1:s roll till t1.
\item Klicka på ``Utse projektledare''.
\item Välj V2:s namn i listan.
\item Klicka på ``OK''.
\item Klicka på ``Byta grupp på användare''.
\item Välj M i listan över användare.
\item Välj G2 i listan över projektgrupper.
\item Klicka på ``OK''.
\item Klicka på ``Administrationsvyn''.
\item Klicka på ``Projektgrupper''.
\item Klicka på ``Ta bort projektgrupper''.
\item Välj G2 i listan.
\item Klicka på ``OK''.
\item Klicka på ``Logga ut''. 
\end{enumerate}
\end{ST}


%+++++++++++++++++++++++++%
%SLUT på ST:generella krav%
%+++++++++++++++++++++++++%


%++++++++++++++++++++++++++%
%BÖRJAN på ST:autentisering%
%++++++++++++++++++++++++++%
\subsection{Autentisering}

subsubsection{Användare}

\begin{ST}

\item
\textbf{Användare kan logga in.}

\emph{Starttillstånd:} Användaren A är inte inloggad.

\emph{Sluttillstånd:} A är inloggad.

\begin{enumerate}
\item A når systemet.
\item A ombedes ange användarnamn och lösenord på inloggningssidan.
\item A anger korrekt användarnamn och lösenord.
\item Kontrollera att A är inloggad
\item Kontrollera att A vidarebefodras till en sida där funktionaliteten för en inloggad användare finns tillgänglig.
\end{enumerate}

\item
\textbf{Användare kan logga ut.}

\emph{Starttillstånd:} Användaren A är inloggad.

\emph{Sluttillstånd:} A är inte inloggad.

\begin{enumerate}
\item A når systemet.
\item A når en sida där det finns en utloggningslänk.
\item A klickar på denna utloggningslänk.
\item Kontrollera att A inte är inloggad
\item Kontrollera att ett meddelande som anger detta visas.
\end{enumerate}

\item
\textbf{Användare misslyckas med inloggning.}

\emph{Starttillstånd:} Användaren A är inte inloggad.

\emph{Sluttillstånd:} A är inte inloggad.

\begin{enumerate}
\item A når systemet.
\item A ombedes ange användarnamn och lösenord på inloggningssidan.
\item A anger användarnamn och lösenord som inte finns i systemet.
\item Kontrollera att A inte är inloggad
\item Kontrollera att ett felmeddelande visas
\item Kontrollera att A ombeds ange användarnamn och lösenord igen.
\end{enumerate}

\end{ST}

\subsubsection{Administratör}

\begin{ST}

\item
\textbf{Administratören kan skapa projektgrupp.}

\emph{Starttillstånd:} Administratören A är inloggad, sidan för att redigera projekgrupper visas.

\emph{Sluttillstånd:} A är inloggad.

\begin{enumerate}
\item A väljer att skapa projektgrupp.
\item A får nu fylla i ett projektgruppsnamn och tycker därefter på ''ok''.
\item 
\begin{enumerate}
\item A får nu en lista över alla användare i systemet, första användare som läggs till tilldelas automatiskt projektledarrollen.
\item Projektgruppsnamnet existerar redan, A informeras och skickas tillbaka till steg 3.
\item A får ett felmeddelande om inga fler användare existerar, sedan omdirigeras A till huvudsidan.
\item A får ett felmeddelande om inga fler projektgrupper kan skapas eftersom maxantalet redan är uppnått.
\end{enumerate}
\item A kan därefter lägga till fler projektgruppsmedlemmar. A kan dessutom lägga till ytterliggare en projektledare. Sedan klickar A på ''ok''.
\item Kontrollera att en projektgrupp skapats
\item Kontrollera att A omdirigeras till en sida som meddelar att en projektgrupp skapats.
\end{enumerate}

\item
\textbf{Administratören kan lägga till eller ändra roll på projektmedlemmar i en projektgrupp.}

\emph{Starttillstånd:} Administratören A är inloggad, sidan för att redigera projekgrupper visas.

\emph{Sluttillstånd:} A är inloggad.

\begin{enumerate}
\item 
\begin{enumerate}
\item A väljer att ''redigera projektmedlemmar''.
\item A får ett felmeddelande som sägar att inga projektmedlemmar/projektgrupper finns, A omdirigeras tillbaka till huvudsidan.
\end{enumerate}
\item A får upp en lista med samtliga projektgrupper och projektmedlemmar och en lista med samtliga användare i systemet som inte är projektmedlemmar överhuvudtaget.

\item A kan nu flytta över projektmedlemmar och icke-projektmedlemmar mellan olika projektgrupper. A kan även ange vilken roll projektmedlemmen ska ha.
\item 
\begin{enumerate}
\item A bekräftar förändringarna med att klicka på ''ok''. A dirigeras då till en sida med den uppdaterade informationen och en bekräftelse på förändringen.
\item A får ett felmeddelande då den försökt flytta en projektledare som är ensam projektledare i sin projektgrupp. A bes då tillsätta en ny projektledare i gruppen innan den gamla flyttas. A omdirigeras till steg 3.
\item A får ett felmeddelande ifall den försöker lägga till fler projektmedlemmar i en grupp som har max antal projektmedlemmar redan. A omdirigeras till steg 3.
\item A får ett felmeddelande ifall A försöker tilldela en projektmedlem rollen som projektledare när max antalet projektledare för den projektgruppen redan är uppnått. A omdirigeras till steg 3. Analogt uppkommer felmeddelande om någon annan roll tilldelats där det redan finns 6 st gruppmedlemmar med den rollen.
\end{enumerate}
\end{enumerate}

\item
\textbf{Administratören kan ta bort projektmedlemmar eller en projektgrupp.}

\emph{Starttillstånd:} Administratören A är inloggad, sidan för att redigera projekgrupper visas.

\emph{Sluttillstånd:} A är inloggad.

\begin{enumerate}
\item 
\begin{enumerate}
\item A väljer ta bort projektgrupp/projektmedlemmar.
\item A får ett felmeddelande som sägar att inga projektmedlemmar/projektgrupper finns, A omdirigeras tillbaka till huvudsidan.
\end{enumerate}
\item A får upp en lista med samtliga projektgrupper och projektmedlemmar.
\item 
\begin{enumerate}
\item A markerar projektmedlemmar för borttagning och trycker därefter ''ok''. Bekräftelseruta specifierad i krav 6.1.14 i SRS visas och A trycker ''Ja''.
\item A trycker "nej" istället för ''ja'' och omdirigeras då tillbaka till steg 2.
\item A får ett felmeddelande om A försöker ta bort en projektledare som är enda projektledare i en grupp med minst en annan medlem. A ombeds utse en ny projektledare och försöka igen, A omdirigeras till steg 2.
\end{enumerate}
\item Kontrollera att borttagningen genomfördes som den skulle.
\item Kontrollera att A dirigeras till en sida med uppdaterad information om projektgrupper och deras medlemmar.
\end{enumerate}

\end{ST}




%++++++++++++++++++++++++%
%SLUT på ST:autentisering%
%++++++++++++++++++++++++%



%++++++++++++++++++++++++++++%
%BÖRJAN på ST:tidrapportering%
%++++++++++++++++++++++++++++%
%Johan
\subsection{Tidrapportering}
\subsubsection{Projektmedlem}

\begin{ST}




\item
\textbf{Genomför scenario 6.3.1 (Dokumentera arbetstimmar i systemet)}

\emph{Starttillstånd:} Projektmedlem A är inloggad, inne vid huvudmenysidan. Ingen tidigare tidrapport för denne användare finns.

\emph{Sluttillstånd:} Projektmedlem A är inloggad, inne vid tidrapportfunktionalitetssidan. Tidrapport skapad för vecka 51.

\begin{enumerate}
\item A skriver in URL för funktionalitetssidan för tidrapportering och får tillgång till sidan.
\item Kontrollera att under fältet I där man skriver in veckonummer framgår det att ingen tidigare tidrapport har skapats.
\item A skriver in veckonummer för veckan som ska tidrapporteras i det visade fältet. A skriver in:
\begin{itemize}
\item[] (a) ??=/
\item[] (b) 100
\item[] (c) Ingenting
\end{itemize}
\item För varje fall i steg 3a-c trycker A på ``OK''
\item Kontrollera att ingen tidrapport genereras.
\item Kontrollera att ett felmeddelande genereras om att veckonumret är otillåtet.
\item Kontrollera att A skickas tillbaka till sidan där man skriver in veckonummer.
\item A skriver in 51 i I och trycer på ``OK''.
\item Kontrollera att en ny tidrapport genereras med veckonumret 51 samt dagens datum ifyllt.
\item A skriver in i godtycklig ruta:
\begin{itemize}
\item[] (a) ??=/
\item[] (b) 123456
\end{itemize}
\item Kontrollera att A kan skriva in tidinformationen i 10a-b
\item För varje fall i steg 10a-b trycker A på ``Skicka''. 
\item Kontrollera att ingen tidrapport sparas och att sidan för specificerad tidrapport laddas om.
\item A skriver in 30 i godtycklig ruta och 150 i en annan.
\item A trycker på ``Skicka''.
\item Kontrollera att tidrapporten är skapad och att A får en bekräftelse om att den är sparad i databasen.
\item Kontrollera att användaren kan se totaltiden från tidrapporten.
\item A skriver in URL för funktionalitetssidan för tidrapportering.
\item Kontrollera att A kan se, under fältet I, information om att tidrapporten för vecka 51 var senast sparad samt dagens datum.
\item A skriver in 51 i fältet I och trycker på ``OK''.
\item Kontrollera att en ny tidrapport inte skapas.
\item Kontrollera att denna rapport hämtas från databasen och visas för A.
\item Kontrollera att den visade rapporten kan redigeras.

\end{enumerate}

\item
\textbf{Genomför scenario 6.3.2 (Ta bort/redigera arbetstimmar i systemet)}

\emph{Starttillstånd:} Projektmedlem A är inloggad, inne vid huvudmenysidan. Två tidrapporter finns. En signerad, för vecka 12. En osignerad för vecka 51.\\
\emph{Sluttillstånd:} Projektmedlem A är inloggad, inne vid tidrapportfunktionalitetssidan. En tidrapport finns. En signerad, för vecka 12.\\

\begin{enumerate}
\item A skriver in URL för funktionalitetssidan för tidrapportering och får tillgång till sidan.
\item A väljer ``Ta bort/redigera tidrapport''.
\item Kontrollera att A får upp en lista med tidrapporter.
\item A väljer tidrapporten för vecka 12.
\item Kontrollera att funktionaliten för redigering eller radering av tidrapporten inte finns.\\
\item A skriver in URL för ``Ta bort/redigera tidrapporter''.
\item A väljer tidrapporten för vecka 51.
\item Kontrollera att A kan välja både ``Redigera'' och ``Ta bort''.
\item A trycker på ``Redigera''.
\item Användaren skriver in i godtycklig tom ruta:
\begin{itemize}
\item[] (a) Text som inte är ascii
\item[] (b) 123456
\end{itemize}
\item För varje fall i steg 12a-b trycker A på ``Skicka''. 
\item Kontrollera att ingen tidrapport sparas och att sidan för specificerad tidrapport laddas om.
\item A skriver in 100 i godtycklig tom ruta och trycker på ``OK''.
\item Kontrollera att tidrapporten är uppdaterad.
\item A skriver in URL för ``Ta bort/redigera tidrapporter''.
\item A väljer vecka 51.
\item A trycker på ``Ta bort''.
\item Kontrollera att en bekräftelseruta visas.
\item A trycker på ``Nej''.
\item Kontrollera att tidrapporten inte är borttagen och att sidan där tidrapporten listas visas.
\item A väljer vecka 51.
\item A trycker på ``Ta bort''.
\item A trycker på ``Ja''.
\item Kontrollera att tidrapport är borttagen.
\end{enumerate}

\item
\textbf{Systemet stödjer stegen i figur 5 (Ref. 2)} 

\emph{Starttillstånd:} Projektmedlem A är inte inloggad. Signerad tidrapport y finns i systemet. \\
\emph{Sluttillstånd:} Projektmedlem A är inte inloggad. Osignerad tidrapport x och signerad tidrapport y finns i systemet.\\

\begin{enumerate}

\item A skriver in URL för inloggningssidan.
\item A loggar in, fel lösenord rätt användarnamn. Misslyckas, sidan laddas om.
\item A skriver in rätt lösenord och rätt användarnamn, lyckas, huvudsidan visas.
\item Kontrollera att A kan se projektmedlemsfunktionaliteter på funktionalitetssidan I.\\
\item A väljer ``Lista medlemmar''.
\item Kontrollera att alla projektmedlemmar i As grupp listas.
\item A går tillbaka till I.
\item A väljer ``Tidrapportering''.
\item A väljer ``Ny tidrapport''
\item Fyller i tidrapport, 50 i godtycklig ruta, trycker på ``Skicka'', lyckas. Tidrapport x skapad.
\item A går tillbaka till I.
\item A väljer "Redigera tidrapport".
\item Kontrollera att gamla tidrapporter listas.
\item Väljer tidrapport y, kan inte redigera.
\item A går tillbaka till ``Redigera tidrapport''.
\item A väljer tidrapport x.
\item A ändrar/tar bort rapporten.
\item A väljer ``Visa statistik''.
\item Kontrollera att olika sorters statistik listas (statistik per användare, per roll etc.).\\
\item A genererar statistik per roll, alla veckor.
\item A loggar ut.
\item Kontrollera att A är utloggad.

\end {enumerate}



\end{ST}

\subsubsection{Projektledare}

%MAX 100 RAPPORTER KRAV INTE MED
\begin{ST}
\item 
\textbf{Systemet stödjer stegen i figur 6 (Ref. 2)}\\
\emph{Starttillstånd:} Projektledare A inte inloggad. Projektmedlem B har rollen t1, en osignerad tidrapport x.\\
\emph{Sluttillstånd:} Projektledare A inte inloggad. Projektmedlem B har rollen t2, en signerad tidrapport x.\\

\begin{enumerate}

\item A skriver in URL till inloggninssidan.
\item A skriver in fel lösenord, rätt användarnamn. Misslyckas, sidan laddas om.
\item A skriver in rätt lösenord och rätt användarnamn. Lyckas logga in.
\item Kontrollera att A har tillgång till projektledarfuntionaliteter.
\item A väljer ``Lista medlemmar''.
\item Kontrollera att A ser alla medlemmar och dess roller.
\item A byter roll på medlem B till t2.
\item A klickar på menyn.
\item A väljer ``Tidrapportering''.
\item A väljer ``Ny tidrapport''.
\item A fyller in ny tidrapport och trycker på ``Skicka''.
\item Kontrollera att tidrapport är skapad.
\item A skriver in URL för ``Tidrapportering''.
\item A trycker ``Redigera tidrapport''.
\item Kontrollera att en lista med tidrapporter kan ses.
\item A försöker redigera/ta bort sin signerade rapport, går inte.
\item A ändrar/tar bort sin osignerade rapport.
\item A skriver URL ``Tidrapportering''.
\item A väljer ``Visa statistik''.
\item Kontrollera att A kan se en lista med statistik (per användare, per roll etc.)
\item A väljer statistik per användare för alla veckor.
\item A trycker ``Generera''.
\item Kontrollera att A kan se statistiken.
\item A skriver URL ``Tidrapportering''.
\item A väljer ``Visa alla tidrapporter''.
\item A väljer projektmedlem B, sätter tidrapport x till godkänd.
\item A loggar ut.
\item Kontrollera att A är utloggad.



\end{enumerate}

\end{ST}
%++++++++++++++++++++++++++%
%SLUT på ST:tidrapportering%
%++++++++++++++++++++++++++%




%+++++++++++++++++++++++++++%
%BÖRJAN på ST:Administration%
%+++++++++++++++++++++++++++%
\subsection{Administration}
\invisiblesubsubsection{Övergripande}

\subsubsection{Projektledare}
\begin{ST}

\item
\textbf{Genomför scenario 6.4.1 i SRS som projektledare}

\emph{Starttillstånd:} Projektledare PL inloggad.

\emph{Sluttillstånd:} Projekledare PL inloggad.

\begin{enumerate}
\item
PL väljer "Generera statistik" i menyn
\item
PL väljer [typ av statistik]
\item 
PL väljer [typ av rapport]
\item
Kontrollera att rapport av typ [typ av rapport] visas.
\item
Kontrollera att statistik av typ [typ av statistik] visas.
\end{enumerate}

\item
\textbf{Genomför scenario 6.4.2 i SRS som projektledare}

\emph{Starttillstånd:} Projektledare PL inloggad. En icke godkänd tidsrapport finns i systemet.

\emph{Sluttillstånd:} Projekledare PL inloggad. En godkänd tidsrapport finns i systemet.

\begin{enumerate}

\item
PL väljer att lista tidrapporter
\item
PL väljer en icke godkänd tidrapport ur listan
\item
Kontrollera att PL ser rapporten
\item
PL godkänner rapporten
\item
Kontrollera att dialogruta visades
\item
Kontrollera att rapporten är godkänd

\end{enumerate}

\item
\textbf{Genomför scenario 6.4.3 i SRS som projektledare}

\item
\textbf{Genomför scenario 6.4.4 i SRS som projektledare}

\emph{Starttillstånd:} Projektledare L som är projektledare för grupp G är inloggad och befinner sig på sidan ``Visa statistik''. Det finns tre projektmedlemmar i G utöver L i systemet, M1, M2 och M3. M1 heter ``axelg'', M2 heter ``axelu'', M3 heter ``johan''. De har två tidsrapporteringar var för vecka 1 och 2. M1 har en tidrapport T1a som innehåller 10 minuter registrerat på möte vecka 1 och en tidsrapport T1b som innehåller 15 minuter registrerat på möte vecka 2. M2 har en tidrapport T2a som innehåller 20 minuter registrerat på möte vecka 1 och en tidsrapport T2b som innehåller 25 minuter registrerat på möte vecka 2. M3 har en tidrapport T3a som innehåller 30 minuter registrerat på möte vecka 1 och en tidsrapport T3b som innehåller 35 minuter registrerat på möte vecka 2. Alla tidrapporter för vecka 1 är signerade, alla tidrapporter för vecka 2 är osignerade.

\emph{Sluttillstånd:} Oförändrat.

\begin{enumerate}
\item Tryck på ``Sortera alla tidrapporter efter namn, stigande ordning''.
\item Kontrollera att det är sorterat på ordningen M1, M2 sen M3.
\item Tryck på ``Sortera alla tidrapporter efter vecka, stigande ordning''.
\item Kontrollera att det är sorterat på ordningen vecka 1 sen vecka 2.
\item Tryck på ``Sortera alla tidrapporter efter om rapporten är godkänd eller ej, stigande ordning''.
\item Kontrollera att det är sorterat på ordningen osignerat sen signerat.
\item Tryck på ``Sortera alla tidrapporter efter namn, fallande ordning''.
\item Kontrollera att det är sorterat på ordningen M3, M2 sen M1.
\item Tryck på ``Sortera alla tidrapporter efter vecka, fallande ordning''.
\item Kontrollera att det är sorterat på ordningen vecka 2 sen vecka 1.
\item Tryck på ``Sortera alla tidrapporter efter om rapporten är godkänd eller ej, fallande ordning''.
\item Kontrollera att det är sorterat på ordningen signerat sen osignerat.
\end{enumerate}

\end{ST}

\subsubsection{Administratör}

\begin{ST}

\item
\textbf{Genomför scenario 6.4.1 i SRS som administratör}

\emph{Starttillstånd:} Administratör A inloggad.

\emph{Sluttillstånd:} Administratör A inloggad.

\begin{enumerate}
\item
A väljer "Generera statistik" i menyn
\item
A väljer [typ av statistik]
\item 
A väljer [typ av rapport]
\item
Kontrollera att rapport av typ [typ av rapport] visas.
\item
Kontrollera att statistik av typ [typ av statistik] visas.
\end{enumerate}

\item
\textbf{Genomför scenario 6.4.2 i SRS som administratör}

\emph{Starttillstånd:} Administratör A inloggad. En icke godkänd tidsrapport finns i systemet.

\emph{Sluttillstånd:} Administratör A inloggad. En godkänd tidsrapport finns i systemet.

\begin{enumerate}

\item
A väljer att lista tidrapporter
\item
A väljer en icke godkänd tidrapport ur listan
\item
Kontrollera att A ser rapporten
\item
A godkänner rapporten
\item
Kontrollera att dialogruta visades
\item
Kontrollera att rapporten är godkänd

\end{enumerate}

\item
\textbf{Genomför scenario 6.4.3 i SRS som administratör}

\emph{Starttillstånd:} Projektledare A inloggad. Inne på sidan för tidrapportering. Rapport x godkänd.\\
\emph{Sluttillstånd:} Projektledare A inloggad. Inne på sidan för tidrapportering. Rapport x inte godkänd.\\


\begin{enumerate}
\item A listar alla tidrapporter
\item A väljer rapport x, trycker på "Ej godkänd"
\item Kontrollera att en dialogruta kommer fram som bekräftar att x inte är godkänd.
\item Kontrollera att x inte är godkänd.
\item Kontrollera att A dirigeras tillbaka till sidan över alla tidrapporter.
\end{enumerate}

\item
\textbf{Genomför scenario 6.4.4 i SRS som administratör}

\emph{Starttillstånd:} Administratören A är inloggad och befinner sig på sidan ``Visa statistik''. Det finns tre projektmedlemmar i projektgruppen G i systemet, M1, M2 och M3. M1 heter ``axelg'', M2 heter ``axelu'', M3 heter ``johan''. De har två tidsrapporteringar var för vecka 1 och 2. M1 har en tidrapport T1a som innehåller 10 minuter registrerat på möte vecka 1 och en tidsrapport T1b som innehåller 15 minuter registrerat på möte vecka 2. M2 har en tidrapport T2a som innehåller 20 minuter registrerat på möte vecka 1 och en tidsrapport T2b som innehåller 25 minuter registrerat på möte vecka 2. M3 har en tidrapport T3a som innehåller 30 minuter registrerat på möte vecka 1 och en tidsrapport T3b som innehåller 35 minuter registrerat på möte vecka 2. Alla tidrapporter för vecka 1 är signerade, alla tidrapporter för vecka 2 är osignerade.

\emph{Sluttillstånd:} Oförändrat.

\begin{enumerate}
\item Tryck på ``Sortera alla tidrapporter efter namn, stigande ordning''.
\item Kontrollera att det är sorterat på ordningen M1, M2 sen M3.
\item Tryck på ``Sortera alla tidrapporter efter vecka, stigande ordning''.
\item Kontrollera att det är sorterat på ordningen vecka 1 sen vecka 2.
\item Tryck på ``Sortera alla tidrapporter efter om rapporten är godkänd eller ej, stigande ordning''.
\item Kontrollera att det är sorterat på ordningen osignerat sen signerat.
\item Tryck på ``Sortera alla tidrapporter efter namn, fallande ordning''.
\item Kontrollera att det är sorterat på ordningen M3, M2 sen M1.
\item Tryck på ``Sortera alla tidrapporter efter vecka, fallande ordning''.
\item Kontrollera att det är sorterat på ordningen vecka 2 sen vecka 1.
\item Tryck på ``Sortera alla tidrapporter efter om rapporten är godkänd eller ej, fallande ordning''.
\item Kontrollera att det är sorterat på ordningen signerat sen osignerat.
\end{enumerate}

\item \textbf{Administratören har tillgång administratörsfunktionaliteter} 

\emph{Utgår}
\end{ST}



%+++++++++++++++++++++++++++%
%SLUTET på ST:Administration%
%+++++++++++++++++++++++++++%




%+++++++++++++++++++++++++++%
%BÖRJAN på ST:Kvalitetstest+%
%+++++++++++++++++++++++++++%
\subsection{Kvalitetskrav}

\subsubsection{Prestanda}

\begin{ST}
\item
\textbf{Försök logga in med fler än 50 användare samtidigt.}

\emph{Starttillstånd:} Användare user1-user51 finns registrerade med lösenord pass. Inga användare inloggade.

\emph{Sluttillstånd:} Användare user1-user50 är inloggade. User51 är inte inloggad

\begin{enumerate}

\item
Försök logga in användare user1-user51.
\item
Kontrollera att user1-user50 är inloggade.
\item
Kontrollera att user51 inte är inloggad.
\end{enumerate}

\item
\textbf{Logga in med 50 användare}

\emph{Starttillstånd:} Användare user1-user50 finns registrerade med lösenord pass. Inga användare inloggade.

\emph{Sluttillstånd:} Användare user1-user50 är inloggade.

\begin{enumerate}

\item
Försök logga in användare user1-user50.
\item
Kontrollera att user1-user50 är inloggade.
\end{enumerate}



\item
\textbf{Svaret på en godtycklig förfrågan från en dator i E-huset kommer i 95\% av fallen tillbaka
inom en sekund.}

\emph{Starttillstånd:} Administratör inloggad på dator i E-huset.

\emph{Sluttillstånd:} Administratör inloggad på dator i E-huset.

\begin{enumerate}

\item Administratören försöker generera statistik 40 gånger. Tiden det tar innan servern svarar mäts varje gång.

\item Den uppmätta tiden utvärderas och bör inte vara högre än en (1) sekund fler än två (2) gånger.


\end{enumerate}

\end{ST}

%+++++++++++++++++++++++++++%
%SLUTET på ST:Kvalitetstest+%
%+++++++++++++++++++++++++++%
%\end{ST}
\subsection{Regressionstest}

Alla test ska köras två gånger i veckan.

När något ändras ska helst alla tester köras igen. Om så inte är möjligt ska åtminstone de
generella kraven och de tester i det område som förändringen påverkade köras.
Om något ändras inom dessa områden måste följande testfall regressionstestas. Vid varje
ändring ska de test rörande de generella kraven också testas.
Områden innefattar:

\begin{itemize}

\item
Generella krav

\item
Autentisering

\item
Tidrapportering

\item
Administration

\end{itemize}


\end{document}
