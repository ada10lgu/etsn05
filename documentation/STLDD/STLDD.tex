\documentclass[a4paper]{article}
\usepackage[pdftex]{graphicx}
\usepackage{anysize}
\marginsize{3cm}{3cm}{3cm}{3cm}
\linespread{1.2}
\usepackage[utf8]{inputenc}
\usepackage[T1]{fontenc}       
\usepackage[swedish]{babel}      
\usepackage{epstopdf}     % För svensk avstavning och svenska
\usepackage[osf]{mathpazo} % Palatino with smallcaps and oldstyle numbers
\usepackage[scaled]{helvet}
\usepackage{morefloats} %så att man kan ha fler float-bilder
%\restylefloat{table}
\usepackage{etoolbox}
\usepackage{hanging}
\usepackage{listings}
\usepackage{graphicx}
\usepackage[width=.75\textwidth]{caption} %så all bildtext blir smalare

\lstset{language=SQL}

\newcommand\getcurrentref[1]{%
 \ifnumequal{\value{#1}}{0}
  {??}
  {\the\value{#1}}%
}  
\newcommand\requirement[2]{
	\numberedrow{Krav}{#1}{#2}
}
\newcommand\scenario[2] {
	\numberedrow{Scenario}{#1}{#2}
}
\newcommand\numberedrow[3]{
	\noindent
	\textbf{#1 \getcurrentref{section}.\getcurrentref{subsection}.#2.} #3
	
}

\usepackage{fancyhdr}
\fancyhf{}
\fancyhead[L]{Ansvarig: SG}

\fancyhead[R]{Datum: \today | Version: 1.1 | Dokumentnummer: PUSS144404}


\title{STLDD - Software Top Level Design Document: NewPussSystem}                  	
\author{Systemarkitektgruppen \\ Nina Khayyami | Johan Rönnåker |Martin Lichota | Marcel Tovar Rascon}
\date{}

\begin{document}

\maketitle
\thispagestyle{fancy}
\tableofcontents
\newpage

\section*{Dokumenthistorik}

\begin{tabular}{ l l l p{8.5cm} }
Ver. & Datum & Ansv. & Beskrivning \\\hline
0.1 & 30 september 2014 & SG & Struktur för dokumentet\\
0.2 & 2 oktober 2014 & SG & Lagt in klass-diagram, ER-diagram, sekvens-diagram samt lagt in all text. Färdigt för informell granskning.\\
0.3 & 7 oktober 2014 & SG & Uppdaterat klassbeskrivningar, klassmetoder, klassdiagram, SQL-frågor, sessionsinformation, sekvensdiagram samt figurbeskrivningar.\\
0.4 & 7 oktober 2014 & SG & Uppdaterat enligt informella granskningsprotokollet.\\
0.5 & 7 oktober 2014 & SG & Uppdaterat klassdiagram, uppdaterat sekvensdiagram för klassen ProjectGroupAdmin, rättat stavfel och syntaxfel.\\
0.6 & 7 oktober 2014 & SG & Uppdaterat klassdiagram, sekvensdiagram och ett litet fel i metodbeskrivningen.\\
0.7 & 15 oktober 2014 & SG & Uppdaterat klassdiagram med två nya klasser och fixat småfel i ER-diagram.\\
0.8 & 15 oktoboer 2014 & SG & Uppdaterat sessionsvariabler.\\
0.9 & 16 oktober 2014 & SG & Lade till groupID som sessionsvariabel.\\
1.0 & 17 oktober 2014 & SG & Ändringar i klasserna och metodbeskrivningarna samt klassdiagrammet. Färdig för baseline.\\
1.1 & 22 oktober 2014 & SG & Tagit bort privata metoder, samt uppdaterat klassdiagrammet. Även lagt till en saknad metodbeskrivning i ReportGenerator. 

\end{tabular}
\newpage
\section{Inledning}       
Dokumentet beskriver högnivådesignen för NewPussSystem tidrapporteringssystem.


\section{Referensdokument}
\begin{enumerate}
\item SRS - Software Requirements Specification (Dokumentnummer PUSS144401, version 0.21)
\item STLDD - Software Top Level Design Document: BaseBlockSystem (Dokumentnummer PUSS12004, version 1.0)
\end{enumerate}



\section{Sammanfattning}
Systemet är implementerat i en Tomcat server med följande klasser och klassmetoder:

\subsection{Klasser}
\begin{hangparas}{1.5em}{1}
\textbf{class servletBase} Den här klassen är superklass åt alla servlets i systemet. För beskrivning se sidan 2 i referensdokument 2.
\end{hangparas} 

\vspace{5mm}
\begin{hangparas}{1.5em}{1}
\textbf{class Administration} Se sida 2 i referensdokument 2. En ändring har gjorts där metoden deleteUser ska returnera en boolean och inparametern ändras till (int id).
\end{hangparas}

\vspace{5mm}
\begin{hangparas}{1.5em}{1}
\textbf{class LogIn} Se sida 2 i referensdokument 2.
\end{hangparas}

\vspace{5mm}
\begin{hangparas}{1.5em}{1}
\textbf{class Access} Den här klassen använder sig av tabellen log och users i databasen för att hålla koll på vilka som är inloggade och aktiva. Har en användare varit inaktiv mer än 20 minuter, se dokumentreferens 1 krav 6.2.9, så nekar den åtkomst och loggar ut användaren. \end{hangparas}

\vspace{5mm}
\begin{hangparas}{1.5em}{1}
\textbf{class ProjectGroupAdmin} Den här servleten ärver från ServletBase. Först kontrollerar den att användaren har behörighet att besöka sidan. Om så är fallet presenteras en tabell med information om varje projektgrupp (namn, projektledare). I tabellen kan administratören välja att radera gruppen och att lägga till/radera en projektledare. Projektgruppsnamnet i tabellen länkar administratören till projektledarens ändringsmeny. Nedanför tabellen finns en meny där administratören kan välja att lägga till nya projektgrupper, gå tillbaka till startsidan för administration, gå tillbaka till den gemensamma första inloggningssidan, samt logga ut.
\end{hangparas}

\vspace{5mm}
\begin{hangparas}{1.5em}{1}
\textbf{class GroupHandling} Den här servleten ärver från ServletBase och hanterar användarna i en projektgrupp.
\end{hangparas}

\vspace{5mm}
\begin{hangparas}{1.5em}{1}
\textbf{class TimeReporting} Den här servleten ärver från ServletBase och presenterar en undermeny med menyval relaterade till tidrapportering. Förutom undermenyn kommer första sidan att vara tom. När ett val i undermenyn görs, kommer klassen att hantera den valda funktionen. Funktionerna som finns inkluderar att lista tidrapporter, visa och uppdatera tidrapporter samt skapa nya tidrapporter.
\end{hangparas}

\vspace{5mm}
\begin{hangparas}{1.5em}{1}
\textbf{class ReportGenerator} Den här statiska klassen är en nästlad klass som har metoder som tar emot referenser till databasen och ritar upp en tidrapport med relevanta data. De olika tidrapporter som kan ritas upp är: en redigerbar, ny och tom tidrapport, en redigerbar existerande tidrapport och en icke redigerbar existerande tidrapport.
\end{hangparas}

\vspace{5mm}
\begin{hangparas}{1.5em}{1}
\textbf{class ProjectLeader} Den här servleten ärver från ServletBase och visar en lista över alla användare. Härifrån kan projektledaren ändra projektmedlemmarnas roller. 
\end{hangparas}

\vspace{5mm}
\begin{hangparas}{1.5em}{1}
\textbf{class ReportHandling} Den här servleten ärver från ServletBase och visar två listor; en lista med signerade och en lista med osignerade tidrapporter om man väljer att hantera tidrapporter. Projektledaren kan här signera eller avsignera rapporter.
\end{hangparas}

\vspace{5mm}
\begin{hangparas}{1.5em}{1}
\textbf{class Statistics} Den här servleten ärver från ServletBase och visar vissa funktioner om man väljer att generera statistik. Funktioner som kan väljas är t.ex. att visa vilken vecka som det har tidrapporterats mest tid. 
\end{hangparas}

\vspace{5mm}
\begin{hangparas}{1.5em}{1}
\textbf{class Start} Den här servleten ärver från ServletBase och skapar den statiska menyn som finns tillgänglig för användaren på alla sidor i systemet.\\
\end{hangparas}

\subsection{Klassmetoder}
Nedan beskrivs alla klassmetoder till klasserna som presenterats under rubriken 3.1 Klasser.

\subsubsection{class Access}

\begin{hangparas}{1.5em}{1}
public boolean updateLog(int userID, String session)\\
Updates the log with a new timestamps for the given user and session. \\
@param userID: The requesting users id.\\
@param session: The requesting users session id.\\
@return boolean: True if the user has not been inactive for too long.
\end{hangparas}

\vspace{5mm}
\begin{hangparas}{1.5em}{1}
public boolean logInUser(int userID, String session)\\
Updates the database, sets the user as logged in, stores the current session id and current timestamp for the requesting user.\\
@param userID: The requesting users id.\\
@param session: The requesting users session id.\\
@return boolean: True if the user is not already logged in.
\end{hangparas}

\vspace{5mm}
\begin{hangparas}{1.5em}{1}
public boolean logOutUser(int userID, String session)\\
Updates the database, sets the user as logged out, removes the current session id and current timestamp for the requesting user.\\
@param userID: The requesting users id.\\
@param session: The requesting users session id.\\
@return boolean - True if the user is not already logged out.\\
\end{hangparas}



\subsubsection{class ReportGenerator}

\begin{hangparas}{1.5em}{1}
public static String viewReport(ResultSet data)\\
Prints out a time report in the right format and with the data specified by the ResultSet parameter.\\
@param data: Specifies which data to print in the time report. \\
@return String: A string containing html code.
\end{hangparas}

\begin{hangparas}{1.5em}{1}
public static String viewReport(HashMap<String, int> data)\\
Prints out a time report in the right format and with the data specified by the HashMap parameter.\\
@param data: Specifies which data to print in the time report. \\
@return String: A string containing html code.
\end{hangparas}

\vspace{5mm}
\begin{hangparas}{1.5em}{1}
public static String updateReport(ResultSet data)\\
Prints out a html-form in the right format and with the data specified by the ResultSet parameter.\\ 
@param data: Specifies which data to print in the html-form.\\
@return String: A string containing html code.
\end{hangparas}

\vspace{5mm}
\begin{hangparas}{1.5em}{1}
public static String newReport(int weekNumber)\\
Prints out a time report html-form prefilled with data.\\
@param weekNumber: Specifies for which week the data should be shown.\\
@return String: A string containing html code.\\
\end{hangparas}
 

\subsubsection{class ProjectLeader}

\begin{hangparas}{1.5em}{1}
public void showAllUsers(int groupID, PrintWriter out)\\
Shows a list of all the users in the system.\\
@param groupID: The group which members should be listed.
@param out: PrintWriter object needed for print outs.
\end{hangparas}

\vspace{5mm}
\begin{hangparas}{1.5em}{1}
public boolean changeRole(int userGroupID, String role)\\
Assigns a role to a user in a project group.\\
@param userGroupID: The user to be assigned a role, and in which group.\\
@param role: Which role to assign.\\
@return boolean: True if the user was successfully added.
\end{hangparas}


\section{Klassdiagram}
Ett klassdiagram med alla klasser visas i Figur \ref{umldiagram}.

\begin{figure}[h!]
\centering
\includegraphics[width=160mm]{UML.jpg}
\caption{Klassdiagram över alla klasser i systemet. \label{umldiagram}}
\end{figure}



\section{Databas}

ER-diagram över databasen för systemet visas i figur \ref{ER}. Databasen kan skapas från scratch med följande SQL kommandon: \\

\begin{figure}[ht!]
\centering
\includegraphics[width=120mm]{DB_14okt.jpg}
\caption{ER-diagram över databasen för systemet \label{ER}}
\end{figure}

\begin{lstlisting}
mysql> create database puss1404;

mysql> use puss1404;

mysql> CREATE TABLE IF NOT EXISTS `groups` (
    ->   `id` int(11) NOT NULL AUTO_INCREMENT,
    ->   `name` varchar(20) NOT NULL,
    ->   PRIMARY KEY (`id`),
    ->   UNIQUE KEY `name` (`name`)
    -> ) ENGINE=InnoDB DEFAULT CHARSET=utf8 AUTO_INCREMENT=1 ;

mysql> CREATE TABLE IF NOT EXISTS `log` (
    ->   `id` int(11) NOT NULL AUTO_INCREMENT,
    ->   `user_id` int(11) NOT NULL,
    ->   `time` timestamp NOT NULL DEFAULT CURRENT_TIMESTAMP ON UPDATE
    ->     CURRENT_TIMESTAMP,
    ->   `session` varchar(100) NOT NULL,
    ->   PRIMARY KEY (`id`),
    ->   KEY `user_id` (`user_id`)
    -> ) ENGINE=InnoDB DEFAULT CHARSET=utf8 AUTO_INCREMENT=1 ;

mysql> CREATE TABLE IF NOT EXISTS `reports` (
    ->   `id` int(11) NOT NULL AUTO_INCREMENT,
    ->   `user_group_id` int(11) NOT NULL,
    ->   `date` date NOT NULL,
    ->   `week` int(11) NOT NULL,
    ->   `total_time` int(11) NOT NULL,
    ->   `signed` tinyint(4) NOT NULL,
    ->   PRIMARY KEY (`id`),
    ->   KEY `user_group_id` (`user_group_id`)
    -> ) ENGINE=InnoDB DEFAULT CHARSET=utf8 AUTO_INCREMENT=1 ;
    
mysql> CREATE TABLE IF NOT EXISTS `report_times` (
    ->   `id` int(11) NOT NULL AUTO_INCREMENT,
    ->   `report_id` int(11) NOT NULL,
    ->   `SDP_U` int(11) NOT NULL,
    ->   `SDP_I` int(11) NOT NULL,
    ->   `SDP_F` int(11) NOT NULL,
    ->   `SDP_O` int(11) NOT NULL,
    ->   `SRS_U` int(11) NOT NULL,
    ->   `SRS_I` int(11) NOT NULL,
    ->   `SRS_F` int(11) NOT NULL,
    ->   `SRS_O` int(11) NOT NULL,
    ->   `SVVS_U` int(11) NOT NULL,
    ->   `SVVS_I` int(11) NOT NULL,
    ->   `SVVS_F` int(11) NOT NULL,
    ->   `SVVS_O` int(11) NOT NULL,
    ->   `STLDD_U` int(11) NOT NULL,
    ->   `STLDD_I` int(11) NOT NULL,
    ->   `STLDD_F` int(11) NOT NULL,
    ->   `STLDD_O` int(11) NOT NULL,
    ->   `SVVI_U` int(11) NOT NULL,
    ->   `SVVI_I` int(11) NOT NULL,
    ->   `SVVI_F` int(11) NOT NULL,
    ->   `SVVI_O` int(11) NOT NULL,
    ->   `SDDD_U` int(11) NOT NULL,
    ->   `SDDD_I` int(11) NOT NULL,
    ->   `SDDD_F` int(11) NOT NULL,
    ->   `SDDD_O` int(11) NOT NULL,
    ->   `SVVR_U` int(11) NOT NULL,
    ->   `SVVR_I` int(11) NOT NULL,
    ->   `SVVR_F` int(11) NOT NULL,
    ->   `SVVR_O` int(11) NOT NULL,
    ->   `SSD_U` int(11) NOT NULL,
    ->   `SSD_I` int(11) NOT NULL,
    ->   `SSD_F` int(11) NOT NULL,
    ->   `SSD_O` int(11) NOT NULL,
    ->   `slutrapport_U` int(11) NOT NULL,
    ->   `slutrapport_I` int(11) NOT NULL,
    ->   `slutrapport_F` int(11) NOT NULL,
    ->   `slutrapport_O` int(11) NOT NULL,
    ->   `funktionstest` int(11) NOT NULL,
    ->   `systemtest` int(11) NOT NULL,
    ->   `regressionstest` int(11) NOT NULL,
    ->   `meeting` int(11) NOT NULL,
    ->   `lecture` int(11) NOT NULL,
    ->   `excersice` int(11) NOT NULL,
    ->   `terminal` int(11) NOT NULL,
    ->   `study` int(11) NOT NULL,
    ->   `other` int(11) NOT NULL,
    ->   PRIMARY KEY (`id`),
    ->   KEY `report_id` (`report_id`)
    -> ) ENGINE=InnoDB DEFAULT CHARSET=utf8 AUTO_INCREMENT=1 ;

mysql> CREATE TABLE IF NOT EXISTS `users` (
    ->   `id` int(11) NOT NULL AUTO_INCREMENT,
    ->   `username` varchar(10) NOT NULL,
    ->   `password` varchar(10) NOT NULL,
    ->   `is_admin` tinyint(4) NOT NULL,
    ->   PRIMARY KEY (`id`),
    ->   UNIQUE KEY `username` (`username`)
    -> ) ENGINE=InnoDB DEFAULT CHARSET=utf8 AUTO_INCREMENT=1 ;
    
mysql> CREATE TABLE IF NOT EXISTS `user_group` (
    ->   `id` int(11) NOT NULL AUTO_INCREMENT,
    ->   `user_id` int(11) NOT NULL,
    ->   `group_id` int(11) NOT NULL,
    ->   `role` varchar(20) NOT NULL,
    ->   PRIMARY KEY (`id`),
    ->   KEY `user_id` (`user_id`),
    ->   KEY `group_id` (`group_id`)
    -> ) ENGINE=InnoDB DEFAULT CHARSET=utf8 AUTO_INCREMENT=1 ;
    
mysql> ALTER TABLE `log`
    ->   ADD CONSTRAINT `log_ibfk_1` FOREIGN KEY (`user_id`) REFERENCES `users`
    ->   (`id`);

mysql> ALTER TABLE `reports`
    ->   ADD CONSTRAINT `reports_ibfk_1` FOREIGN KEY (`user_group_id`) REFERENCES
    ->   `user_group` (`id`);

mysql> ALTER TABLE `report_times`
    ->   ADD CONSTRAINT `report_times_ibfk_1` FOREIGN KEY (`report_id`) REFERENCES
    ->   `reports` (`id`);

mysql> ALTER TABLE `user_group`
    ->   ADD CONSTRAINT `user_group_ibfk_2` FOREIGN KEY (`group_id`) REFERENCES
    ->    `groups`  (`id`),
    ->  ADD CONSTRAINT `user_group_ibfk_1` FOREIGN KEY (`user_id`) REFERENCES
    ->    `users`  (`id`);
  
\end{lstlisting}


\section{Information lagrad i sessioner}
I en pågående session sparas följande attribut i sessionen:

\vspace{5mm}
\begin{hangparas}{1.5em}{1}
\textbf{Integer state}: Används för att beskriva om användaren är inloggad eller inte. Följande två tillstånd finns:\\
\textbf{0:} Ej inloggad\\
\textbf{1:} Inloggad
\end{hangparas}

\vspace{5mm}
\begin{hangparas}{1.5em}{1}
\textbf{Session session}: Användarens sessions-id.
\end{hangparas}

\vspace{5mm}
\begin{hangparas}{1.5em}{1}
\textbf{String name}: Användarens användarnamn, e.g. 'admin'. 
\end{hangparas}

\vspace{5mm}
\begin{hangparas}{1.5em}{1}
\textbf{int userID}: Användarens id. 
\end{hangparas}

\vspace{5mm}
\begin{hangparas}{1.5em}{1}
\textbf{int groupID}: Id till gruppen som användaren är inloggad på. 
\end{hangparas}

\vspace{5mm}
\begin{hangparas}{1.5em}{1}
\textbf{int userGroupID}: Id från tabellen user\_group som är ett id som sammanlänkar användare och grupp.
\end{hangparas}

\vspace{5mm}
\begin{hangparas}{1.5em}{1}
\textbf{String role}: Användarens roll i den aktuella gruppen.
\end{hangparas}

\section{Sekvensdiagram}

Observera att inte alla meddelanden visas i följande sekvensdiagram, utan endast de meddelanden som krävs för att förstå sekvensen.

\subsection{class Administration}
Figur 2 i dokumentreferens 2 visar sekvensen för hur servleten Administration hanterar att administratören lägger till en ny användare.


\subsection{class ProjectGroupAdmin}
Figur \ref{addGroup} visar hur servleten ProjectGroupAdmin hanterar en förfrågan om att lägga till en ny projektgrupp och Figur \ref{addGroupFail} visar hur den hanteras när den misslyckas. 

\begin{figure}[h!]
\centering
\includegraphics[width=90mm]{SeqDia__addGroup_FINAL.jpg}
\caption{Sekvensdiagram som visar hur en lyckad förfrågan om att lägga till en ny projektgrupp hanteras i klassen ProjectGroupAdmin. \label{addGroup}}
\end{figure}

\begin{figure}[h!]
\centering
\includegraphics[width=90mm]{SeqDia__addGroup_Fail_FINAL.jpg}
\caption{Sekvensdiagram som visar hur en misslyckad förfrågan om att lägga till en ny projektgrupp hanteras i klassen ProjectGroupAdmin. \label{addGroupFail}}
\end{figure}

\noindent
Figur \ref{addUserToGroup} visar hur servleten ProjectGroupAdmin hanterar en förfrågan om att lägga till en användare i en projektgrupp, medan Figur \ref{addUserToGroupFail} visar hur det hanteras när den misslyckas.

\begin{figure}[h!]
\centering
\includegraphics[width=90mm]{SeqDia__addUserToGroup_FINAL.jpg}
\caption{Sekvensdiagram som visar hur klassen ProjectGroupAdmin hanterar en förfrågan om att lägga till en användare i en projektgrupp.
\label{addUserToGroup}}
\end{figure}

\begin{figure}[h!]
\centering
\includegraphics[width=90mm]{SeqDia__addUserToGroup_Fail_FINAL.jpg}
\caption{Sekvensdiagram som visar hur klassen ProjectGroupAdmin hanterar en misslyckad förfrågan om att lägga till en användare i en projektgrupp.
\label{addUserToGroupFail}}
\end{figure}

\noindent
Figur \ref{removeUserFromGroup} visar hur servleten ProjectGroupAdmin hanterar en förfrågan om att ta bort en användare från en projektgrupp, medan Figur \ref{removeUserFromGroupFail} visar hur det hanteras när den misslyckas.

\begin{figure}[h!]
\centering
\includegraphics[width=90mm]{SeqDia__removeUserFromGroup_FINAL.jpg}
\caption{Sekvensdiagram som visar hur klassen ProjectGroupAdmin hanterar en förfrågan om att ta bort en användare från en projektgrupp.
\label{removeUserFromGroup}}
\end{figure}

\begin{figure}[h!]
\centering
\includegraphics[width=90mm]{SeqDia__removeUserFromGroup_Fail_FINAL.jpg}
\caption{Sekvensdiagram som visar hur klassen ProjectGroupAdmin hanterar en misslyckad förfrågan om att ta bort en användare från en projektgrupp.
\label{removeUserFromGroupFail}}
\end{figure}

\noindent
Figur \ref{removeGroup} visar hur servleten ProjectGroupAdmin hanterar en förfrågan om att ta bort en projektgrupp.

\begin{figure}[h!]
\centering
\includegraphics[width=90mm]{SeqDia__removeGroup_FINAL.jpg}
\caption{Sekvensdiagram som visar hur klassen ProjectGroupAdmin hanterar en förfrågan om att ta bort en projektgrupp.
\label{removeGroup}}
\end{figure}

\subsection{TimeReporting}
Figur \ref{newReport} visar hur servleten TimeReporting hanterar en förfrågan om att skapa en ny tidrapport som läggs till i databasen.

\begin{figure}[h!]
\centering
\includegraphics[width=90mm]{newReport_successful.jpg}
\caption{Visar sekvensen av de grundläggande metodanrop som sker då användaren framgångsrikt skapar en ny tidrapport i klassen TimeReporting.
\label{newReport}}
\end{figure}

\noindent
Figur \ref{removeReport} visar hur servleten TimeReporting hanterar en förfrågan om att ta bort en tidrapport och Figur \ref{removeReportFail} visar hur det hanteras när den misslyckas.

\begin{figure}[h!]
\centering
\includegraphics[width=90mm]{removeReport_successful.jpg}
\caption{Visar sekvensen av de grundläggande metodanrop som sker i klassen TimeReporting då användaren framgångsrikt tar bort en tidrapport som inte är signerad.
\label{removeReport}}
\end{figure}

\begin{figure}[h!]
\centering
\includegraphics[width=90mm]{removeReport_unsuccessful.jpg}
\caption{Visar sekvensen av de grundläggande metodanrop som sker i klassen TimeReporting då användaren misslyckat försöker ta bort en tidrapport som är signerad.
\label{removeReportFail}}
\end{figure}

\noindent
Figur \ref{updateReportSigned} visar hur servleten TimeReporting hanterar en förfrågan om att uppdatera en tidrapport som redan blivit signerad och Figur \ref{updateReportUnsigned} visar hur en förfrågan om att uppdatera en tidrapport som ännu inte blivit signerad hanteras.

\begin{figure}[h!]
\centering
\includegraphics[width=90mm]{updateReport_signed.jpg}
\caption{Visar sekvensen av de grundläggande metodanrop som sker i klassen TimeReporting då användaren misslyckat försöker uppdatera en tidrapport som är signerad.
\label{updateReportSigned}}
\end{figure}

\begin{figure}[h!]
\centering
\includegraphics[width=90mm]{updateReport_unsigned.jpg}
\caption{Visar sekvensen av de grundläggande metodanrop som sker i klassen TimeReporting då användaren framgångsrikt uppdaterar en tidrapport som inte är signerad.
\label{updateReportUnsigned}}
\end{figure}

\noindent
Figur \ref{viewReportList} visar hur servleten TimeReporting hanterar en förfrågan om att skriva ut en användares alla tidrapporter, Figur \ref{viewReportListFail} visar hur den hanteras när den misslyckas.

\begin{figure}[h!]
\centering
\includegraphics[width=90mm]{viewReportList_successful.jpg}
\caption{Visar sekvensen av de grundläggande metodanrop som sker i klassen TimeReporting då användaren framgångsrikt listar alla sina befintliga tidrapporter.
\label{viewReportList}}
\end{figure}

\begin{figure}[h!]
\centering
\includegraphics[width=90mm]{viewReportList_unsuccessful.jpg}
\caption{Visar sekvensen av de grundläggande metodanrop som sker i klassen TimeReporting då användaren misslyckat försöker lista alla sina befintliga tidrapporter, då denne inte har några befintliga tidrapporter.
\label{viewReportListFail}}
\end{figure}


\subsection{ProjectLeader}
Figur \ref{assignRole} visar hur servleten ProjectLeader hanterar en förfrågan om att ändra en projektroll för en användare.

\begin{figure}[h!]
\centering
\includegraphics[width=90mm]{assignRole.jpg}
\caption{Sekvensdiagram som beskriver hur klassen ProjectLeader hanterar en förfrågan om att ändra en projektroll för en användare.
\label{assignRole}}
\end{figure}

\subsection{ReportHandling}

Figur \ref{showAllViewReport} visar hur servleten ReportHandling hanterar en förfrågan om att visa alla tidrapporter i projektgruppen, signerade samt osignerade. Figur \ref{signedReport} visar hur servleten ProjectLeader hanterar en förfrågan om att signera en rapport.

\begin{figure}[h!]
\centering
\includegraphics[width=90mm]{showAllViewReport.jpg}
\caption{Sekvensdiagram som visar hur klassen ReportHandling hanterar en förfrågan om att visa alla tidrapporter i en projektgrupp.
\label{showAllViewReport}}
\end{figure}

\begin{figure}[h!]
\centering
\includegraphics[width=90mm]{signedReport.jpg}
\caption{Sekvensdiagram som visar hur klassen ReportHandling hanterar en förfrågan om att signera en rapport.
\label{signedReport}}
\end{figure}

\subsection{Statistics}
Figur \ref{busiestWeek} visar hur servleten Statistics hanterar en förfrågan om att visa vilken vecka som har flest totala inlagda minuter.\\
Figur \ref{commonActivity} visar hur servleten Statistics hanterar en förfrågan om att visa vilken aktivitet som har flest totala inlagda minuter.\\
Figur \ref{generateStatisticsReport} visar hur servleten Statistics hanterar en förfrågan om att visa en tidrapport som sammanfattar arbetet utfört i en viss aktivitet av en viss grupp/undergrupp under en specifik tid.

\begin{figure}[h!]
\centering
\includegraphics[width=90mm]{busiestWeek.jpg}
\caption{Sekvensdiagram som visar hur en förfrågan om att visa vilken vecka som har flest totala inlagda minuter hanteras i klassen Statistics.
\label{busiestWeek}}
\end{figure}

\begin{figure}[h!]
\centering
\includegraphics[width=90mm]{commonActivity.jpg}
\caption{Sekvensdiagram som beskriver hur en förfrågan om att visa vilken aktivitet som har flest totala inlagda minuter hanteras i klassen Statistics.
\label{commonActivity}}
\end{figure}

\begin{figure}[h!]
\centering
\includegraphics[width=90mm]{generateStatisticsReport.jpg}
\caption{Sekvensdiagram som visar hur klassen Statistics hanterar en förfrågan om att visa en tidrapport som sammanfattar arbetet utfört i en viss aktivitet av en viss grupp/undergrupp under en specifik tid.
\label{generateStatisticsReport}}
\end{figure}

\end{document}